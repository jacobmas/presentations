% hack to deal with beamer/pgf bug
\RequirePackage{atbegshi}

% "draft" makes compilation faster; remove for final version
% shadow = put drop shadows behind blocks by default
% xcolor: svgnames imports lots of color names
% table allows for multi-color table rows
% pdftex is the compiler
\documentclass[shadow,xcolor=pdftex,svgnames,table,t]{beamer}

\usepackage{amsmath,amssymb,amsfonts}
\usepackage{array}
\newif\iflncs\lncstrue
% need to load tikz defs before beamer defs, for some reason
% additional libraries for tikz and pgf
\usepackage{tikz}
\usetikzlibrary{matrix}
\usetikzlibrary{calc}
\usetikzlibrary{shadows}
\usetikzlibrary{scopes}
\usetikzlibrary{chains}
\usetikzlibrary{positioning}
\usetikzlibrary{fit}
\usetikzlibrary{backgrounds}
\usetikzlibrary{decorations.pathmorphing}
\usetikzlibrary{decorations.pathreplacing}
\usetikzlibrary{shapes.geometric}
\usetikzlibrary{shapes.symbols}
\usetikzlibrary{shapes.misc}
\usetikzlibrary{trees}

% color aliases
%\colorlet{Lattice}{Green}
%\colorlet{Set}{Red!10}

% global styles
\tikzset{
  % make drop shadows a bit softer
  every shadow/.style={opacity=.4,shadow xshift=.3ex,shadow yshift=-.3ex},
  % use \& to separate cells in matrices, avoiding fragility in beamer frames
  every matrix/.style={ampersand replacement=\&},
  % edges and joins with arrows and thicker
  every edge/.append style={->,thick},
  every join/.style={->,thick},
  % distance between nodes
  % node distance=1ex and 2ex,
}

% styles for drawing in Rn, especially lattices
\tikzset{
  % lattice transform must include "cm" dimension to work properly,
  % otherwise the x and y transforms become dependent
  Ztrans/.style={x={(1.5cm,1cm)},y={(.5cm,1.5cm)}},
  Zdual/.style={x={(1.5cm,-.5cm)},y={(.5cm,1cm)}},
  latttrans/.style={x={(1.8cm,.5cm)},y={(0.4cm,1.3cm)}},
  dualtrans/.style={x={(1.3cm,-.4cm)},y={(-.5cm,1.8cm)}},
  lattB/.style={x={(1.8cm,.5cm)},y={(0.4cm,1.4cm)}},
  dualB/.style={x={(1.4cm,-.4cm)},y={(-.5cm,1.8cm)}},
  shortB/.style={x={(.5cm,.2cm)},y={(-.1cm,1.7cm)}},
  latt2B/.style={x={(1cm,.1cm)},y={(.1cm,1.2cm)}},
  ulattB/.style={x={(2.0cm,.5cm)},y={(2.1cm,-.2cm)}},
  openball/.style={fill=structure.fg!20},
  ball/.style={openball,thin,draw=structure.fg},
  latt/.style={Lattice,fill},
  dist/.style={Brown,densely dashed,font=\footnotesize},
  offlabel/.style={Brown,inner sep=2pt,font=\tiny},
  axes/.style={->,Gray,style=very thin},
  fundamental/.style={Gray,fill opacity=.2},
  target/.style={fill,Brown},
  project/.style={thin,dashed},
  help lines/.style={Gray,very thin,step=.5cm},
  % expanding waves representing noise
  noise/.style={
    Lattice,
    opacity=.5,
    decorate,decoration={
      expanding waves,
      segment length=5pt,angle=10,
      post=moveto,post length=3pt
    },
  },
}

% lattice that can be dropped into any path.
% it would be sensible to set a default argument here, but
% commands with optional args don't parse correctly inside a
% tikz path
\newcommand*{\latt}[1]{%
  \foreach \a in {-4, ..., 4} {
    \foreach \b in {-4, ..., 4} {
      (\a,\b) circle (#1)
    }
  }
}

% style for functions mapping one point to another
\tikzset{map/.style={->,semithick}}

% style for an offset rounded box
\tikzset{showoff/.style={draw,fill=White,rounded corners,drop
    shadow,inner sep=6pt}}

% style for all algorithms
\tikzset{algorithm/.style={
    draw=#1!50!Black!60,solid,
    top color=White,bottom color=#1!50!Black!15,
    very thick,rounded corners,drop shadow,
    minimum size=.9cm,inner sep=6pt,outer sep=4pt
  }
}

% style for honest, malicious, trusted parties
\tikzset{honest/.style={algorithm=Green}}
\tikzset{oracle/.style={algorithm=Green}}
\tikzset{malicious/.style={algorithm=Red}}
\tikzset{trusted/.style={algorithm=Purple,inner sep=10pt}}

% style for notions
\tikzset{notion/.style={
    draw=#1!50!Black!60,solid,
    top color=White,bottom color=#1!50!Black!15,
    very thick,rounded corners,drop shadow,
    minimum height=6.5ex,inner sep=6pt,
  }
}

\tikzset{app/.style={notion=Blue}}
\tikzset{prim/.style={notion=Green}}
\tikzset{math/.style={notion=Red}}

% % style for a common reference string
\tikzset{crs/.style={draw,fill=White,cloud,cloud puffs=13,cloud ignores aspect,drop shadow}}

%% BOILERPLATE PRESENTATION STUFF

% an underlining package
\usepackage{ulem}
% use normal italics for emphasis
\normalem

% my custom theme
\usetheme{CJP}

% dingbats
\usepackage{pifont}

\renewcommand{\Check}{\ding{52}}
\newcommand{\Cross}{\ding{55}}
\newcommand{\Hand}{\ding{43}}
\newcommand{\GreenCheck}{{\color{Green}\Check}}
\newcommand{\RedCross}{{\color{Red}\Cross}}
\newcommand{\Orange}[1]{{\color{Orange} #1}}

% alias for citations
\newcommand{\citationsize}{\footnotesize}

\subject{Theoretical Computer Science}

% suppress ugly QEDs in proofs
\def\qedsymbol{}


% common stuff
% basic notation

\newcommand{\lamperp}{\Lambda^{\perp}}
\newcommand{\lam}{\Lambda}

% lattice
\newcommand{\lat}{\mathcal{L}}
% fundamental region
\newcommand{\piped}{\mathcal{P}}
% smoothing parameter
\newcommand{\smooth}{\eta}
% smoothing w/ epsilon
\newcommand{\smootheps}{\smooth_{\epsilon}}
% ball
\newcommand{\ball}{\mathcal{B}}
% cube
\newcommand{\cube}{\mathcal{C}}
% covering radius symbol
\newcommand{\cover}{\ensuremath{\mu}}
% Gram-Schmidt
\newcommand{\gs}[1]{\ensuremath{\widetilde{#1}}}
% GS minimum
\newcommand{\gsmin}{\ensuremath{\tilde{bl}}}
% volume operation
\DeclareMathOperator{\vol}{vol}
% Hermite normal form
\DeclareMathOperator{\hnf}{HNF}
% rank
\DeclareMathOperator{\rank}{rank}
% distance operator
\DeclareMathOperator{\dist}{dist}
% span operator
\DeclareMathOperator{\spn}{span}
% error function
\DeclareMathOperator{\erf}{erf}

% quantities that show up regularly
\newcommand{\wsln}{\ensuremath{\omega(\sqrt{\log n})}}
\newcommand{\wslm}{\ensuremath{\omega(\sqrt{\log m})}}

% support algorithms
\newcommand{\sampleZ}{\algo{Sample}\Z}
\newcommand{\sampleD}{\algo{SampleD}}
\newcommand{\tobasis}{\algo{ToBasis}}
\newcommand{\invert}{\algo{Invert}}
\newcommand{\genbasis}{\algo{GenBasis}}
\newcommand{\extbasis}{\algo{ExtBasis}}
\newcommand{\extlattice}{\algo{ExtLattice}}
\newcommand{\randbasis}{\algo{RandBasis}}
\newcommand{\gentrap}{\algo{GenTrap}}
\newcommand{\deltrap}{\algo{DelTrap}}
\newcommand{\noisy}{\algo{Noisy}}

% problems related to lattices
\newcommand{\svp}{\problem{SVP}}
\newcommand{\gapsvp}{\problem{GapSVP}}
\newcommand{\cogapsvp}{\problem{coGapSVP}}
\newcommand{\usvp}{\problem{uSVP}}
\newcommand{\sivp}{\problem{SIVP}}
\newcommand{\gapsivp}{\problem{GapSIVP}}
\newcommand{\cvp}{\problem{CVP}}
\newcommand{\gapcvp}{\problem{GapCVP}}
\newcommand{\cvpp}{\problem{CVPP}}
\newcommand{\gapcvpp}{\problem{GapCVPP}}
\newcommand{\bdd}{\problem{BDD}}
\newcommand{\gdd}{\problem{GDD}}
\newcommand{\add}{\problem{ADD}}
\newcommand{\incgdd}{\problem{IncGDD}}
\newcommand{\incivd}{\problem{IncIVD}}
\newcommand{\crp}{\problem{CRP}}
\newcommand{\gapcrp}{\problem{GapCRP}}
% problems on ideal lattices
\newcommand{\igvp}{\problem{IGVP}}
\newcommand{\incigvp}{\problem{IncIGVP}}
% avg-case stuff
\newcommand{\sis}{\problem{SIS}}
\newcommand{\isis}{\problem{ISIS}}
\newcommand{\ilwe}{\problem{ILWE}}
\newcommand{\lwe}{\problem{LWE}}
\newcommand{\rlwe}{\problem{RLWE}}
\newcommand{\lwr}{\problem{LWR}}
\newcommand{\rlwr}{\problem{RLWR}}
\newcommand{\dlwe}{\problem{DLWE}}
\newcommand{\lpn}{\problem{LPN}}
\newcommand{\Psibar}{\ensuremath{\bar{\Psi}}}

% macros for matrices and vectors
\newcommand{\matA}{\ensuremath{\mathbf{A}}}
\newcommand{\matB}{\ensuremath{\mathbf{B}}}
\newcommand{\matC}{\ensuremath{\mathbf{C}}}
\newcommand{\matD}{\ensuremath{\mathbf{D}}}
\newcommand{\matE}{\ensuremath{\mathbf{E}}}
\newcommand{\matF}{\ensuremath{\mathbf{F}}}
\newcommand{\matG}{\ensuremath{\mathbf{G}}}
\newcommand{\matH}{\ensuremath{\mathbf{H}}}
\newcommand{\matI}{\ensuremath{\mathbf{I}}}
\newcommand{\matJ}{\ensuremath{\mathbf{J}}}
\newcommand{\matK}{\ensuremath{\mathbf{K}}}
\newcommand{\matL}{\ensuremath{\mathbf{L}}}
\newcommand{\matM}{\ensuremath{\mathbf{M}}}
\newcommand{\matN}{\ensuremath{\mathbf{N}}}
\newcommand{\matO}{\ensuremath{\mathbf{O}}}
\newcommand{\matP}{\ensuremath{\mathbf{P}}}
\newcommand{\matQ}{\ensuremath{\mathbf{Q}}}
\newcommand{\matR}{\ensuremath{\mathbf{R}}}
\newcommand{\matS}{\ensuremath{\mathbf{S}}}
\newcommand{\matT}{\ensuremath{\mathbf{T}}}
\newcommand{\matU}{\ensuremath{\mathbf{U}}}
\newcommand{\matV}{\ensuremath{\mathbf{V}}}
\newcommand{\matW}{\ensuremath{\mathbf{W}}}
\newcommand{\matX}{\ensuremath{\mathbf{X}}}
\newcommand{\matY}{\ensuremath{\mathbf{Y}}}
\newcommand{\matZ}{\ensuremath{\mathbf{Z}}}
\newcommand{\matzero}{\ensuremath{\mathbf{0}}}

\newcommand{\veca}{\ensuremath{\mathbf{a}}}
\newcommand{\vecalpha}{\ensuremath{\mathbf{\alpha}}}
\newcommand{\vecb}{\ensuremath{\mathbf{b}}}
\newcommand{\vecc}{\ensuremath{\mathbf{c}}}
\newcommand{\vecd}{\ensuremath{\mathbf{d}}}
\newcommand{\vece}{\ensuremath{\mathbf{e}}}
\newcommand{\vecf}{\ensuremath{\mathbf{f}}}
\newcommand{\vecg}{\ensuremath{\mathbf{g}}}
\newcommand{\vech}{\ensuremath{\mathbf{h}}}
\newcommand{\veck}{\ensuremath{\mathbf{k}}}
\newcommand{\vecm}{\ensuremath{\mathbf{m}}}
\newcommand{\vecp}{\ensuremath{\mathbf{p}}}
\newcommand{\vecq}{\ensuremath{\mathbf{q}}}
\newcommand{\vecr}{\ensuremath{\mathbf{r}}}
\newcommand{\vecs}{\ensuremath{\mathbf{s}}}
\newcommand{\vect}{\ensuremath{\mathbf{t}}}
\newcommand{\vecu}{\ensuremath{\mathbf{u}}}
\newcommand{\vecv}{\ensuremath{\mathbf{v}}}
\newcommand{\vecw}{\ensuremath{\mathbf{w}}}
\newcommand{\vecx}{\ensuremath{\mathbf{x}}}
\newcommand{\vecy}{\ensuremath{\mathbf{y}}}
\newcommand{\vecz}{\ensuremath{\mathbf{z}}}
\newcommand{\veczero}{\ensuremath{\mathbf{0}}}

\DeclareMathOperator{\diag}{diag}
\renewcommand{\O}{\mathcal{O}}

% GENERAL COMPUTING STUFF

\newcommand{\bit}{\ensuremath{\set{0,1}}}
\newcommand{\pmone}{\ensuremath{\set{-1,1}}}

% asymptotic stuff
\DeclareMathOperator{\poly}{poly}
\DeclareMathOperator{\polylog}{polylog}
\DeclareMathOperator{\negl}{negl}
\newcommand{\Otil}{\ensuremath{\tilde{O}}}

% probability/distribution stuff
\DeclareMathOperator*{\E}{\mathbb{E}}
\DeclareMathOperator*{\Var}{Var}

\DeclareMathOperator{\trace}{Tr}
\DeclareMathOperator{\lcm}{lcm}

% assorted
\DeclareMathOperator*{\wt}{wt}

% hash functions
\newcommand{\calH}{\ensuremath{\mathcal{H}}}
\newcommand{\calX}{\ensuremath{\mathcal{X}}}
\newcommand{\calY}{\ensuremath{\mathcal{Y}}}

\newcommand{\compind}{\ensuremath{\stackrel{c}{\approx}}}
\newcommand{\statind}{\ensuremath{\stackrel{s}{\approx}}}
\newcommand{\perfind}{\ensuremath{\equiv}}

% font for general-purpose algorithms
\newcommand{\algo}[1]{\ensuremath{\mathsf{#1}}\xspace}

% font for general-purpose computational problems
\iflncs
\renewcommand{\problem}[1]{\ensuremath{\mathsf{#1}}\xspace}
\else
\newcommand{\problem}[1]{\ensuremath{\mathsf{#1}}\xspace}
\fi

% font for complexity classes
\newcommand{\class}[1]{\ensuremath{\mathsf{#1}}\xspace}

% complexity classes and languages
\renewcommand{\P}{\class{P}}
\newcommand{\BPP}{\class{BPP}}
\newcommand{\NP}{\class{NP}}
\newcommand{\coNP}{\class{coNP}}
\newcommand{\AM}{\class{AM}}
\newcommand{\coAM}{\class{coAM}}
\newcommand{\IP}{\class{IP}}
\newcommand{\SZK}{\class{SZK}}
\newcommand{\NISZK}{\class{NISZK}}
\newcommand{\NICZK}{\class{NICZK}}
\newcommand{\PPP}{\class{PPP}}
\newcommand{\PPAD}{\class{PPAD}}

\newcommand{\yes}{\ensuremath{\text{YES}}}
\newcommand{\no}{\ensuremath{\text{NO}}}

\newcommand{\Piyes}{\ensuremath{\Pi^{\yes}}}
\newcommand{\Pino}{\ensuremath{\Pi^{\no}}}

%\DeclareMathOperator{\trace}{Tr}
%\DeclareMathOperator{\lcm}{lcm}
\DeclareMathOperator{\msb}{msb}
\DeclareMathOperator{\lsb}{lsb}
\DeclareMathOperator{\rad}{rad}

\renewcommand{\O}{\mathcal{O}}
\newcommand{\ord}{\ensuremath{\text{ord}}}

\newcommand{\calA}{\ensuremath{\mathcal{A}}}
\newcommand{\calB}{\ensuremath{\mathcal{B}}}
\newcommand{\calC}{\ensuremath{\mathcal{C}}}
\newcommand{\calD}{\ensuremath{\mathcal{D}}}
\newcommand{\calE}{\ensuremath{\mathcal{E}}}
\newcommand{\calF}{\ensuremath{\mathcal{F}}}
\newcommand{\calG}{\ensuremath{\mathcal{G}}}
\newcommand{\calI}{\ensuremath{\mathcal{I}}}
\newcommand{\calJ}{\ensuremath{\mathcal{J}}}
\newcommand{\calK}{\ensuremath{\mathcal{K}}}
\newcommand{\calL}{\ensuremath{\mathcal{L}}}
\newcommand{\calM}{\ensuremath{\mathcal{M}}}
\newcommand{\calN}{\ensuremath{\mathcal{N}}}
\newcommand{\calO}{\ensuremath{\mathcal{O}}}
\newcommand{\calP}{\ensuremath{\mathcal{P}}}
\newcommand{\calQ}{\ensuremath{\mathcal{Q}}}
\newcommand{\calR}{\ensuremath{\mathcal{R}}}
\newcommand{\calS}{\ensuremath{\mathcal{S}}}
\newcommand{\calT}{\ensuremath{\mathcal{T}}}
\newcommand{\calU}{\ensuremath{\mathcal{U}}}
\newcommand{\calV}{\ensuremath{\mathcal{V}}}
\newcommand{\calW}{\ensuremath{\mathcal{W}}}
\newcommand{\calZ}{\ensuremath{\mathcal{Z}}}

\newcommand{\frakd}{\mathfrak{d}}
\newcommand{\frakp}{\mathfrak{p}}


%\algnotext{EndFor}
%\algnotext{EndIf}
%\algnotext{EndWhile}

% fix weird spacing of \pmod, and introduce \pmod* command; see
% http://tex.stackexchange.com/questions/39221/removing-extra-space-with-pmod-command
\makeatletter
\renewcommand{\pod}[1]{\mathchoice
  {\allowbreak \if@display \mkern 18mu\else \mkern 8mu\fi (#1)}
  {\allowbreak \if@display \mkern 18mu\else \mkern 8mu\fi (#1)}
  {\mkern4mu(#1)}
  {\mkern4mu(#1)}
}
\makeatletter
\let\@@pmod\pmod
\DeclareRobustCommand{\pmod}{\@ifstar\@pmods\@@pmod}
\def\@pmods#1{\mkern4mu({\operator@font mod}\mkern 6mu#1)}
\makeatother
% crypto-related notation

% KEYS AND RELATED

\newcommand{\mm}[1]{\ensuremath{#1}}

\newcommand{\pp}{\mm{pp}}      % public params
\newcommand{\pk}{\mm{pk}}
\newcommand{\vk}{\mm{vk}}
\newcommand{\sk}{\mm{sk}}
\newcommand{\mpk}{\mm{mpk}}
\newcommand{\msk}{\mm{msk}}
\newcommand{\fk}{\mm{fk}}
\newcommand{\id}{id}
\newcommand{\msgspace}{\ensuremath{\mathcal{M}}}
\newcommand{\idspace}{\ensuremath{\mathcal{ID}}}

\newcommand{\concat}{\ensuremath{\|}}

% GAMES STUFF

% advantage
\newcommand{\advan}{\ensuremath{\mathbf{Adv}}}

% different attack models
\newcommand{\attack}[1]{\ensuremath{\text{#1}}}

\newcommand{\atk}{\attack{atk}} % dummy attack
\newcommand{\indcpa}{\attack{ind-cpa}}
\newcommand{\indcca}{\attack{ind-cca}}
\newcommand{\anocpa}{\attack{ano-cpa}} % anonymous
\newcommand{\anocca}{\attack{ano-cca}}
\newcommand{\sidindcpa}{\attack{sid-ind-cpa}} % selective-id
\newcommand{\aidindcpa}{\attack{aid-ind-cpa}} % adaptive-id
\newcommand{\sidindcca}{\attack{sid-ind-cca}}
\newcommand{\aidindcca}{\attack{aid-ind-cca}}
\newcommand{\euacma}{\attack{eu-acma}} % forgery: adaptive chosen-message
\newcommand{\euscma}{\attack{eu-scma}} % forgery: static chosen-message
\newcommand{\suacma}{\attack{su-acma}} % strong forgery: adaptive chosen-message
\newcommand{\suscma}{\attack{su-scma}} % strong forgery: static chosen-message

% ADVERSARIES
\newcommand{\attacker}[1]{\ensuremath{\mathcal{#1}}}

\newcommand{\Adv}{\attacker{A}}
\newcommand{\Sim}{\attacker{S}}
\newcommand{\Ora}{\attacker{O}}
\newcommand{\Inv}{\attacker{I}}
\newcommand{\For}{\attacker{F}}

% CRYPTO SCHEMES

\newcommand{\scheme}[1]{\ensuremath{\text{#1}}}

% public-key cryptosystem
\newcommand{\pkc}{\scheme{PKC}}
\newcommand{\pkcsetup}{\algo{Setup}}
\newcommand{\pkcgen}{\algo{Gen}}
\newcommand{\pkcenc}{\algo{Enc}} % can also use \kemenc and \kemdec
\newcommand{\pkcdec}{\algo{Dec}}
\newcommand{\msgsp}{\mathcal{M}}

% extra deniable algorithms for pkc
\newcommand{\biden}{\scheme{BI\mbox{-}DEN}}
\newcommand{\den}{\scheme{DEN}}
\newcommand{\dengen}{\algo{DenGen}} 
\newcommand{\denenc}{\algo{DenEnc}} % can also use \kemenc and \kemdec
\newcommand{\rfake}{\algo{RecFake}}
\newcommand{\sfake}{\algo{SendFake}}

% plan ahead deniability 
\newcommand{\pden}{\scheme{PL\mbox{-}DEN}}
\newcommand{\pdenenc}{\algo{PADenEnc}} % can also use \kemenc and \kemdec
\newcommand{\prfake}{\algo{PARecFake}}
\newcommand{\psfake}{\algo{PASendFake}}

% bitranslucent sets
\newcommand{\bts}{\scheme{BTS}}
\newcommand{\btsgen}{\algo{Gen}}
\newcommand{\btsdengen}{\algo{DenGen}}
\newcommand{\sampleP}{\algo{SampleP}} 
\newcommand{\sampleU}{\algo{SampleU}} 
\newcommand{\testP}{\algo{TestP}}
\newcommand{\fakeScoins}{\algo{FakeSCoins}}
\newcommand{\fakeRcoins}{\algo{FakeRCoins}}

% digital signatures
\newcommand{\sig}{\scheme{SIG}}
\newcommand{\siggen}{\algo{Gen}}
\newcommand{\sigsign}{\algo{Sign}}
\newcommand{\sigver}{\algo{Ver}}

% key-encapsulation mechanism
\newcommand{\kem}{\scheme{KEM}}
\newcommand{\kemgen}{\algo{Gen}}
\newcommand{\kemenc}{\algo{Encaps}}
\newcommand{\kemdec}{\algo{Decaps}}

% symmetric cipher
\newcommand{\sym}{\scheme{SYM}}
\newcommand{\symenc}{\algo{E}}
\newcommand{\symdec}{\algo{D}}

% identity-based encryption
\newcommand{\ibe}{\scheme{IBE}}
\newcommand{\ibesetup}{\algo{Setup}}
\newcommand{\ibeext}{\algo{Ext}}
\newcommand{\ibeenc}{\algo{Enc}}
\newcommand{\ibedec}{\algo{Dec}}

% hierarchical IBE (as key encapsulation)
\newcommand{\hibe}{\scheme{HIBE}}
\newcommand{\hibesetup}{\algo{Setup}}
\newcommand{\hibeext}{\algo{Extract}}
\newcommand{\hibeenc}{\algo{Encaps}}
\newcommand{\hibedec}{\algo{Decaps}}

% binary tree encryption (as key encapsulation)
\newcommand{\bte}{\scheme{BTE}}
\newcommand{\btesetup}{\algo{Setup}}
\newcommand{\bteext}{\algo{Extract}}
\newcommand{\bteenc}{\algo{Encaps}}
\newcommand{\btedec}{\algo{Decaps}}

% trapdoor functions
\newcommand{\tdf}{\scheme{TDF}}
\newcommand{\tdfgen}{\algo{Gen}}
\newcommand{\tdfeval}{\algo{Eval}}
\newcommand{\tdfinv}{\algo{Invert}}
\newcommand{\tdfver}{\algo{Ver}}

% preimage samplable functions
\newcommand{\psf}{\scheme{PSF}}
\newcommand{\psfgen}{\algo{Gen}}
\newcommand{\psfsamdom}{\algo{SampleDom}}
\newcommand{\psfsampre}{\algo{SamplePre}}

% COMPLEXITY MEASURES

\newcommand{\qhash}{\ensuremath{Q_{H}}}
\newcommand{\qext}{\ensuremath{Q_{E}}}
\newcommand{\qid}{\ensuremath{Q_{\text{id}}}}
\newcommand{\qsign}{\ensuremath{Q_{\text{sign}}}}

% UC STUFF

\newcommand{\func}[1]{\ensuremath{\mathcal{F}_{#1}}\xspace}
\newcommand{\prot}[1]{\ensuremath{\mathcal{\pi}_{#1}}\xspace}
\newcommand{\exec}{\ensuremath{\text{EXEC}}}
\newcommand{\ideal}{\ensuremath{\text{IDEAL}}}
% \Sim and \Adv are already defined above
\newcommand{\Env}{\attacker{Z}}
\newcommand{\sid}{\ensuremath{\mathsf{sid}}\xspace}
\newcommand{\ssid}{\ensuremath{\mathsf{ssid}}\xspace}
\newcommand{\command}[1]{\ensuremath{\mathsf{#1}}\xspace}
% "left-right" pairs of symbols

% inner product
\newcommand{\inner}[1]{\langle{#1}\rangle}
\newcommand{\innerfit}[1]{\left\langle{#1}\right\rangle}
% absolute value
\newcommand{\abs}[1]{\lvert{#1}\rvert}
\newcommand{\absfit}[1]{\left\lvert{#1}\right\rvert}
% a set
\newcommand{\set}[1]{\{{#1}\}}
\newcommand{\setfit}[1]{\left\{{#1}\right\}}
% parens
\newcommand{\parens}[1]{({#1})}
\newcommand{\parensfit}[1]{\left({#1}\right)}
% tuple, alias for parens
\newcommand{\tuple}[1]{\parens{#1}}
\newcommand{\tuplefit}[1]{\parensfit{#1}}
% square brackets
\newcommand{\bracks}[1]{[{#1}]}
\newcommand{\bracksfit}[1]{\left[{#1}\right]}
% rounding off
\newcommand{\round}[1]{\lfloor{#1}\rceil}
\newcommand{\roundfit}[1]{\left\lfloor{#1}\right\rceil}
% floor function
\newcommand{\floor}[1]{\lfloor{#1}\rfloor}
\newcommand{\floorfit}[1]{\left\lfloor{#1}\right\rfloor}
% ceiling function
\newcommand{\ceil}[1]{\lceil{#1}\rceil}
\newcommand{\ceilfit}[1]{\left\lceil{#1}\right\rceil}
% length of some vector, element
\newcommand{\length}[1]{\lVert{#1}\rVert}
\newcommand{\lengthfit}[1]{\left\lVert{#1}\right\rVert}


% blackboard symbols

\newcommand{\C}{\ensuremath{\mathbb{C}}}
\newcommand{\D}{\ensuremath{\mathbb{D}}}
\newcommand{\F}{\ensuremath{\mathbb{F}}}
\newcommand{\G}{\ensuremath{\mathbb{G}}}
\newcommand{\J}{\ensuremath{\mathbb{J}}}
\newcommand{\N}{\ensuremath{\mathbb{N}}}
\newcommand{\Q}{\ensuremath{\mathbb{Q}}}
\newcommand{\R}{\ensuremath{\mathbb{R}}}
\newcommand{\T}{\ensuremath{\mathbb{T}}}
\newcommand{\Z}{\ensuremath{\mathbb{Z}}}

\newcommand{\Zt}{\ensuremath{\Z_t}}
\newcommand{\Zp}{\ensuremath{\Z_p}}
\newcommand{\Zq}{\ensuremath{\Z_q}}
\newcommand{\ZN}{\ensuremath{\Z_N}}
\newcommand{\Zps}{\ensuremath{\Z_p^*}}
\newcommand{\ZNs}{\ensuremath{\Z_N^*}}
\newcommand{\JN}{\ensuremath{\J_N}}
\newcommand{\QRN}{\ensuremath{\mathbb{QR}_N}}

% MARGIN NOTES

\newif\ifnotes\notestrue
%\newif\ifnotes\notesfalse


\ifnotes
\usepackage{color}
\definecolor{mygrey}{gray}{0.50}
\newcommand{\noteby}[2]{{\textcolor{mygrey}{\footnotesize{\bf (#1:} {#2}{\bf ) }}}}
\newcommand{\noteswarning}{{\begin{center} {\Large WARNING: NOTES ON}\end{center}}}

\else

\newcommand{\noteby}[2]{{}}
\newcommand{\noteswarning}{{}}

\fi

\newcommand{\cnote}[1]{{\noteby{Chris}{#1}}}
\newcommand{\jnote}[1]{{\noteby{Jacob}{#1}}}
\newcommand{\dnote}[1]{{\noteby{Daniel}{#1}}}

\DeclareFontFamily{U}{mathb}{\hyphenchar\font45}
\DeclareFontShape{U}{mathb}{m}{n}{
  <5> <6> <7> <8> <9> <10> gen * mathb
  <10.95> mathb10 <12> <14.4> <17.28> <20.74> <24.88> mathb12
}{}
\DeclareSymbolFont{mathb}{U}{mathb}{m}{n}

\DeclareFontFamily{U}{mathx}{\hyphenchar\font45}
\DeclareFontShape{U}{mathx}{m}{n}{
  <5> <6> <7> <8> <9> <10>
  <10.95> <12> <14.4> <17.28> <20.74> <24.88>
  mathx10
}{}
\DeclareSymbolFont{mathx}{U}{mathx}{m}{n}

\DeclareMathSymbol{\boxplus}      {2}{mathb}{"60}
\DeclareMathSymbol{\boxminus}     {2}{mathb}{"61}
\DeclareMathSymbol{\boxtimes}     {2}{mathb}{"62}
\DeclareMathSymbol{\boxdot}       {2}{mathb}{"64}
\DeclareMathSymbol{\boxcirc}      {2}{mathb}{"65}
\DeclareMathSymbol{\boxslash}     {2}{mathb}{"6D}
\DeclareMathSymbol{\boxbackslash} {2}{mathb}{"6E}
\DeclareMathSymbol{\bigboxplus}      {1}{mathx}{"D0}
\DeclareMathSymbol{\bigboxminus}     {1}{mathx}{"D1}
\DeclareMathSymbol{\bigboxtimes}     {1}{mathx}{"D2}
\DeclareMathSymbol{\bigboxdot}       {1}{mathx}{"D4}
\DeclareMathSymbol{\bigboxcirc}      {1}{mathx}{"D5}


\newcommand{\GL}{\ensuremath{\text{GL}}}

% trace
\renewcommand{\O}{\mathcal{O}}
\newcommand{\Blue}[1]{{\color{Blue}#1}}
\newcommand{\Red}[1]{{\color{Red}#1}}
\newcommand{\Purple}[1]{{\color{Purple}#1}}
\newcommand{\Green}[1]{{\color{Green}#1}}

% fix weird spacing of \pmod, and introduce \pmod* command; see
% http://tex.stackexchange.com/questions/39221/removing-extra-space-with-pmod-command
\makeatletter
\renewcommand{\pod}[1]{\mathchoice
  {\allowbreak \if@display \mkern 18mu\else \mkern 8mu\fi (#1)}
  {\allowbreak \if@display \mkern 18mu\else \mkern 8mu\fi (#1)}
  {\mkern4mu(#1)}
  {\mkern4mu(#1)}
}
\makeatletter
\let\@@pmod\pmod
\DeclareRobustCommand{\pmod}{\@ifstar\@pmods\@@pmod}
\def\@pmods#1{\mkern4mu({\operator@font mod}\mkern 6mu#1)}
\makeatother
%%%% 

\title
{ Faster Bootstrapping with\\Polynomial Error
}

\author
{ {\Large Jacob Alperin-Sheriff}
  \and
  {\Large Chris Peikert}
}

\institute{School of Computer Science\\Georgia Tech}

\date
{
  CRYPTO 2014\\
  19 August 2014
}

% restricts presentation to specified frames, for faster compilation;
% comment out the line to get the entire presentation

% Frame labels:
% title, fhe, bootstrapping, prev, results, overview,embedwarmup,
% embedsmall, thanks



%\includeonlyframes{embedwarmup}

\begin{document}

%%% put less vertical space around displayed equations
\setlength{\abovedisplayskip}{8pt plus 2pt minus 1pt}
\setlength{\belowdisplayskip}{8pt plus 2pt minus 0pt}
\setlength{\abovedisplayshortskip}{0pt plus 2pt}
\setlength{\belowdisplayshortskip}{8pt plus 2pt minus 1pt}

\begin{frame}[label=title]
  \titlepage
\end{frame}

\jnote{Will people notice if we don't change the first two slides?}

\begin{frame}[label=fhe]{Fully Homomorphic Encryption {\citationsize
      [RAD'78,Gentry'09]}}
  \begin{itemize}
  \item<+-> FHE lets you do this:
    \begin{center}
      \begin{tikzpicture}
       \node (c) {$\fbox{$\mu$}$};
  
        \node[right=of c,algorithm=Blue] (eval)
        {$\text{Eval}\parens*{f}$};

        \node[right=of eval] (out) {$\fbox{$f(\mu)$}$};

        \path
        (c) edge (eval)
        (eval) edge (out)
        ;
      \end{tikzpicture}
    \end{center}
    % where $\abs{f(\mu)}$ and decryption time don't depend on
    % $\abs{f}$.

    \medskip
    A cryptographic ``holy grail'' with countless applications.

    \medskip First solved in {\citationsize [Gentry'09]}, followed by

    \hfill
    {\scriptsize [vDGHV'10,BV'11a,BV'11b,BGV'12,B'12,GSW'13,\ldots]}
  \end{itemize}

  \onslide<+->
  %\bigskip
  \begin{itemize}
  \item ``Naturally occurring'' schemes are \alert<.>{somewhat
      homomorphic}~(SHE):

    can only evaluate functions of an \alert<.>{\textit{a priori}
      bounded} depth.
  \end{itemize}

  \begin{center}
    \begin{tikzpicture}[node distance=.4cm]
      \node (c) {$\Green{\fbox{$\mu$}}$};
      
      \node[right=of c,algorithm=Blue] (eval)
      {$\text{Eval}\parens*{f}$};
      
      \node[right=of eval] (cprime) {$\color{Orange} \fbox{$f(\mu)$}$};

      \node[right=of cprime,algorithm=Blue] (eval2)
      {$\text{Eval}\parens*{g}$};

      \node[right=of eval2] (cdp) {$\Red{\fbox{$g(f(\mu))$}}$};

      \path
      (c) edge (eval)
      (eval) edge (cprime)
      (cprime) edge (eval2)
      (eval2) edge (cdp)
      ;
    \end{tikzpicture}
  \end{center}
  \begin{itemize}
  \item<+-> Thus far, ``\alert{bootstrapping}'' is required to
    achieve~\alert{unbounded} FHE.
  \end{itemize}
\end{frame}

\begin{frame}[label=bootstrapping]{Bootstrapping: SHE $\to$ FHE {\citationsize [Gentry'09]}}
  \begin{itemize}
  \item<+-> \alert<.>{Homomorphically evaluates the SHE decryption
      function} to ``refresh'' a ciphertext $\Red{\fbox{$\mu$}}$,
    allowing further homomorphic operations.

    \begin{center}
      \begin{tikzpicture}[every node/.style={node distance=.7cm}]
        \node (sk) {$\Green{\fbox{$sk$}}$};

        \node[right=of sk,algorithm=Blue] (eval)
        {$\text{Eval}\parens*{\; \text{Dec} \parens*{\cdot\;,
              \Red{\fbox{$\mu$}}} \;}$};

        \node[right=of eval] (out) {${\color{Orange} \fbox{$\mu$}}$};

        \path (sk) edge (eval) (eval) edge (out) ;

      \end{tikzpicture}
    \end{center}

    % \onslide<+->
    % \begin{itemize}
    % \item \alert<.>{Goal}: Efficiency!  Minimize depth $d$ and
    %   size $s$ of decryption ``circuit.''

    %   \medskip
    % \item Best SHEs {\citationsize [BGV'12]} can evaluate in time
    %   $\tilde{O}(d \cdot s \cdot \lambda)$.
    % \end{itemize}

    \smallskip
  
  \item<+-> Error growth of bootstrapping determines cryptographic
    assumptions.

    \onslide<+->
    State of the art {\citationsize [BGV'12,B'12,GSW'13]}:

    \begin{itemize}
    \item<+-> \textbf{Homom Addition:} Error grows \alert<.>{additively}.
      \smallskip
    \item<+-> \textbf{Homom Multiplication:} Error grows by
      \alert<.>{$\poly(\lambda)$} factor.
    \end{itemize}
      
    \medskip
  \item<+-> Known decryption circuits have \alert<.>{logarithmic}
    $O(\log \lambda)$ depth.

    \onslide<+-> $\Longrightarrow$ Quasi-polynomial
    \alert<.>{$\lambda^{O(\log \lambda)}$} error growth and lattice
    approx factors

    \medskip
  \item<+-> Can we do better?
  \end{itemize}
\end{frame}

\begin{frame}[label=prev1]{Bootstrapping with Polynomial Error
    {\citationsize[BrakerskiVaikuntanathan'14]}}

  \begin{itemize}
  \item<+-> Error growth for multiplication in {\citationsize
      [GSW'13]} is \alert<.>{asymmetric}:
    % 
    \[\text{Error in }\matC:=\Blue{\matC_1} \boxdot \Red{\matC_2}\text{
      is } \vece := \Blue{\vece_1} \cdot \text{poly}(\lambda) + \Blue{\mu_1}
    \cdot \Red{\vece_2}. \]
  \item<+-> Make multiplication \alert<.>{right-associative}:
    \[\matC_1 \boxdot (\cdots (\matC_{t-2} \boxdot (\matC_{t-1} \boxdot
    \matC_{t})) \cdots )\text{ has error } \textstyle\sum_{i}\vece_{i}
    \cdot \text{poly}(\lambda)\]
  \item<+-> Barrington's Theorem \vspace{-5pt}
  \end{itemize}
  
  \onslide<.->
  \begin{tikzpicture}[circuit logic US, tiny circuit symbols,
    every circuit symbol/.style={ fill=white,draw}]
    \matrix[column sep=5mm] (bobmat) { \onslide<4- >{ \node (i0) {0};}
      \pgfmatrixnextcell \pgfmatrixnextcell
      \\ \pgfmatrixnextcell  \node [and gate] (a1) {};  \pgfmatrixnextcell \\
      \onslide<4->{\node (i1) {0};} \pgfmatrixnextcell \pgfmatrixnextcell \node [or gate] (o) {};\\
      \pgfmatrixnextcell \node [nand gate] (a2) {}; \pgfmatrixnextcell \\
      \onslide<4->{\node (i2) {1};} \pgfmatrixnextcell \pgfmatrixnextcell \\
    }; \draw (i0.east) -- ++(right:3mm) |- (a1.input 1); \draw
    (i1.east) -- ++(right:3mm) |- (a1.input 2); \draw (i1.east)
    -- ++(right:3mm) |- (a2.input 1); \draw (i2.east) --
    ++(right:3mm) |- (a2.input 2); \draw (a1.output) --
    ++(right:3mm) |- (o.input 1); \draw (a2.output) --
    ++(right:3mm) |- (o.input 2); \draw (o.output) --
    ++(right:3mm);
    
    \node<3-|handout:0> (depthnode) [below=5pt of bobmat] 
    {depth $d \uncover<5->{\alert{\approx 3 \log \lambda}}$};
    
    
    \node (arrowpointright) [right=31pt of o] {};
    \node (arrowpointleft) [right=6pt of o] {};
    \node (tempnode) [right=6pt of arrowpointright] {};
    
    \draw[->,line width=5pt] (arrowpointleft) -- (arrowpointright);
    
    
    \node (P01) [above=6pt of tempnode] {\scriptsize
      $(P_{0,1})$};
    \node (P00) [below=6pt of tempnode] {\scriptsize
      $(P_{0,0})$};
    \node (P11) [right=14pt of P01] {\scriptsize
      $(P_{1,1})$};
    \node (P10) [right=14pt of P00] {\scriptsize
      $(P_{1,0})$};         
    \node (P21) [right=13pt of P11] {\scriptsize
      $\ldots$};
    \node (P20) [right=13pt of P10] {\scriptsize
      $\ldots$};  
    \node (P31) [right=13pt of P21] {\scriptsize
      $(P_{14,1})$};
    \node (P30) [right=13pt of P20] {\scriptsize
      $(P_{14,0})$};  
    \node (P41) [right=16pt of P31] {\scriptsize
      $(P_{15,1})$};
    \node (P40) [right=16pt of P30] {\scriptsize
      $(P_{15,0})$};  
    
    \node<3-|handout:0> (lengthnode) [right=100pt of depthnode]
    {length $4^d \uncover<5->{\alert{\approx \lambda^{6}}}$};
    
    \onslide<4->{
      \path[->] (P00) edge (P11)
      (P11) edge (P21) 
      (P21) edge (P30)
      (P30) edge (P40);
    }
    
  \end{tikzpicture}
  
  \begin{itemize}
  \item<5->[\RedCross] Problem: Barrington's transformation is
    \alert<5->{very inefficient}.
  \end{itemize}
\end{frame}

\begin{frame}[label=results]{Our Results}
  \begin{enumerate}
  \item<+-> \alert<.>{Faster bootstrapping} with \alert<.>{small
      polynomial} error growth

    \begin{itemize}
    \item<+-> Treats decryption as an \alert<.>{arithmetic
        function} over $\Zq$, not a circuit.
      
      \onslide<+-> \alert<.>{Avoids Barrington's Theorem} -- but still
      uses permutation matrices!
      
      \medskip
    \item<+-> \uline{Key Idea}: Embed additive group $(\Zq,+)$ into
      small symmetric group
      
      \medskip
      
      \onslide<+->
      
      \medskip
      \renewcommand{\arraystretch}{1.5}
      \begin{tabular}{|c|c|c|}
        \hline
        \textbf{Reference} & \textbf{\# Homom Ops} & \textbf{Noise Growth}\\
        \hline
        {\citationsize [GHS'12,AP'13]} (packing) & $\tilde{O}(1)$ \GreenCheck &
        $\lambda^{O(\log \lambda)}$ \\ \hline \hline
        {\citationsize [BV'14]} & $\tilde{O}(\lambda^6)$ & large
        $\poly(\lambda)$ \\
        \hline
        This work & $\tilde{O}(\lambda)$ \GreenCheck & $\tilde{O}(\lambda^{2})$\\
        \hline
      \end{tabular}
      
      % \medskip
      % \item<+-> \alert<.>{Small polynomial} approx factors
      %   \alert<.>{without ``dimension leveraging.''}
    \end{itemize}
    
    
    \medskip
  \item<+-> Variant of {\citationsize [GSW'13]} encryption scheme
    \begin{itemize}
    \item<+-> Very simple description and error analysis
      
      \medskip
    \item<+-> Enjoys \alert<.>{full re-randomization} of error as a
      natural side effect
      
      \smallskip Cf.~{\citationsize [BV'14]}: partial
      re-randomization, using extra key material
    \end{itemize}
  \end{enumerate}
\end{frame}

\begin{frame}[label=gsw]{Simpler GSW Variant}
  \begin{itemize}
  \item<+-> ``Gadget'' $\Zq$-matrix $\matG$ {\citationsize [MP'12]}:
    for any $\Zq$-matrix $\matA$,
    \[ \Red{\matG^{-1}(\matA)} \text{ is \Red{short}} \qquad
    \text{and} \qquad \matG \cdot \Red{\matG^{-1}(\matA)} = \matA \pmod*{q}. \]

  \item<+-> Ciphertext encrypting $\mu \in \bit$ under $\Blue{\vecs}$
    is a \alert<.>{$\Zq$-matrix} $\matC$ satisfying
    \[ \Blue{\vecs} \matC =
    \mu \cdot \Blue{\vecs} \matG + \Red{\vece} \pmod{q}. \]
    
  \item<+-> Homomorphic multiplication: $\matC_1 \boxdot \matC_2 :=
    \matC_1 \cdot \Red{\matG^{-1}(\matC_2)}$.
    \begin{align*}
      \Blue{\vecs} \matC_1 \cdot \Red{\matG^{-1}(\matC_2)} &=
      (\mu_{1} \cdot \Blue{\vecs} \matG + \Red{\vece_{1}})
      \cdot \Red{\matG^{-1}(\matC_{2})} \\
      \onslide<.->{
        &= \mu_{1} \cdot \Blue{\vecs} \matC_{2} + \Red{\vece_{1}
          \cdot \matG^{-1}(\matC_{2})}} \\
      \onslide<.->{
        &= \mu_1 \mu_2 \cdot \Blue{\vecs} \matG + \underbrace{\mu_1
          \cdot \Red{\vece_2} + \Red{\vece_1 \cdot
            \matG^{-1}(\matC_2)}}_{\text{new error}}.}
    \end{align*}
    
  \item<+-> Old method {\citationsize [GSW'13]}:
    $\matG^{-1}$ is \alert<.>{deterministic bit
      decomposition}.

    \smallskip
  \item<+-> New: $\matG^{-1}$ samples a \alert<.>{(random) subgaussian
      preimage}.
    \begin{itemize}
    \item[$\Rightarrow$] Tight $O(\sqrt{n})$ error growth, full
      rerandomization of error
    \end{itemize}
  \end{itemize}
\end{frame}

\newcommand{\mathbox}[1]{\fbox{$#1$}}

\begin{frame}[label=overview]{Overview of Our Bootstrapping Algorithm}
  \vspace{-3pt} \onslide<+->
  \begin{block}{}
    \begin{itemize}
    \item Decryption in LWE-based schemes can be expressed as
      \vspace{-5pt}
      \begin{equation*}
        \text{Dec}_{\vecs}(\vecc) := \round{\inner{\Blue{\vecs}, \vecc}}_2 \in
        \{0,1\}\text{ with } \Blue{\vecs} \in \Zq^{n},\; \vecc \in \bit^n
      \end{equation*}

    \end{itemize}
  \end{block}
  \vspace{-4pt} \jnote{Something about how at the surface it's similar
    to other depth-3 schemes?}
  \begin{enumerate}
  \item<+-> \alert<.-.(1)>{Prepare}: Encrypt each $\Blue{s_j} \in \Zq$
    under a certain group embedding.

  \end{enumerate}
  \onslide<+-> \alert<.->{Bootstrapping} procedure uses two homomorphic
  algorithms:
  \[ \mathbox{a} \boxplus \mathbox{b}=\mathbox{a+b} \quad \text{and}
  \quad
  \mathsf{Equals}(\mathbox{v},z)=
  \begin{cases}
    \scriptsize\mathbox{1} & \text{if } v=z \\
    \scriptsize \mathbox{0} & \text{otherwise}
  \end{cases} \]

  \smallskip
  \onslide<+-> Given ciphertext $\vecc \in \bit^{n}$ and encryptions
  \mathbox{\Blue{s_j}}, evaluate:

  \smallskip
  \begin{enumerate}
    \setcounter{enumi}{1}
    
  \item<.-> \alert<.->{Inner Product}: compute $\displaystyle
    \mathbox{v} := \inner{\mathbox{\Blue{\vecs}}, \vecc} =
    \bigboxplus_{j:\; c_j=1} \mathbox{\Blue{s_j}}$
    
    \medskip
  \item<+-> \alert<.->{Round}: compute $\displaystyle
    \mathbox{\round{v}_2} := \bigboxplus_{z:\; \round{z}_2=1}
    \mathsf{Equals}(\mathbox{v},z) $
  \end{enumerate}
  
  \begin{itemize}
  \item<+-> Remains to implement $\bigboxplus$ and $\mathsf{Equals}$
    for plaintext space $\Zq$.
  \end{itemize}
\end{frame}

\begin{frame}[label=embedwarmup]{Warmup: Embedding $(\Zq,+)$ into
    $(S_q,\cdot)$}
  % {\tiny CJP: there need to be 0s with boxes around them in the
  %   encrypted matrices.  The latex is
  %   unreadable/unmaintainable... consider using tikz's matrix
  %   environment to avoid all these mode changes.}

  \onslide<+->
  \begin{center}
    % \renewcommand{\arraystretch}{.5}
    \begin{tabular}{ccccc}
      % \renewcommand{\arraystretch}{1.0}
      $\Zq$&0&1&\ldots&$q-1$\\
      \noalign{\medskip}

      $S_q$&$\arraycolsep=3pt\scriptsize \begin{pmatrix}\only<1 |
        handout:0>{1}\only<2->{\Red{\mathbox{1}}}&\invisible{\mathbox{1}}&&\invisible{\mathbox{1}}\\
        \uncover<2->{\Red{\mathbox{0}}}&1&&\\
        \uncover<2->{\Red{\vdots}}&&\ddots&\\
        \uncover<2->{\Red{\mathbox{0}}}&&&1\\
      \end{pmatrix}$&


      $\arraycolsep=3pt\scriptsize \begin{pmatrix}\uncover<2->{\Red{\mathbox{0}}}&&\invisible{\mathbox{1}}&1\\
        \only<1|handout:0>{1}\only<2->{\Red{\mathbox{1}}}&&&\invisible{\mathbox{1}}\\
        \uncover<2->{\Red{\vdots}}&\ddots&&\\
        \uncover<2->{\Red{\mathbox{0}}}&&1&
      \end{pmatrix}$&

      \ldots&

      $\arraycolsep=3pt\scriptsize \begin{pmatrix}\uncover<2->{\Red{\mathbox{0}}}&1&&\invisible{\mathbox{1}}\\
        \uncover<2->{\Red{\vdots}}&\invisible{\mathbox{1}}&\ddots&\\
        \uncover<2->{\Red{\mathbox{0}}}&&&1\\
        \only<1|handout:0>{1}\only<2->{\Red{\mathbox{1}}}&\invisible{\mathbox{1}}&&
      \end{pmatrix}$\\
      \noalign{\medskip}
      \uncover<2->{&$\mathbox{P_0}$&$\mathbox{P_1}$&$\ldots$&$\mathbox{P_{q-1}}$}\\
    \end{tabular}
  \end{center}
  \vspace{-3pt} \medskip \onslide<+->
  \begin{itemize}
  \item<+-> Addition: $\mathbox{a} \boxplus \mathbox{b}$ implemented
    as $\mathbox{P_{a}} \boxdot \mathbox{P_b} = \mathbox{P_{a} \cdot
      P_{b}}$

    \begin{itemize}
    \item<.-> Recall: Right-associative multiplication yields
      polynomial error growth.
    \end{itemize}

  \item<+-> Equality test: $\mathsf{Equals}(\mathbox{a},b)$: take
    $b$th entry from first column of $\mathbox{P_a}$.

    \medskip
  \item<+-> Bottom line: $\tilde{O}(\lambda^3)$ homomorphic operations
    to bootstrap.
  \end{itemize}
\end{frame}

\begin{frame}[label=embedsmall]{Embedding $(\Zq,+)$ into Smaller Symmetric Groups}
  \begin{itemize}
  \item<+->Let $q=p_1 \cdots p_t = \tilde{O}(\lambda)$ for distinct
    prime $p_i$.
    \begin{itemize}
      \smallskip
    \item<.-> Prime Number Theorem allows $p_i, t=O(\log \lambda)$.
    \end{itemize}

    \onslide<+-> Chinese Remainder Theorem: $\Z_{q} \cong \Z_{p_1}
    \times \cdots \times \Z_{p_t}$

  \item<+-> New embedding:
    \begin{align*}
      \Zq &\to S_{p_1} \times \cdots \times S_{p_t} \\
      x &\mapsto (P_{x \bmod{p_1}}, \ldots, P_{x \bmod{p_t}})\\
    \end{align*}
    \vspace{-30pt}

  \item<+-> Addition: same as in warmup, but component-wise

    \medskip
  \item<+-> Equality test:
    \[\mathsf{Equals}_q(\mathbox{a},b)=\bigboxdot_i\mathsf{Equals}_{p_i}(\mathbox{a_{i}},b
    \bmod p_i)\]
  \item<+-> Bottom line: $\tilde{O}(\lambda)$ homomorphic operations
    to bootstrap.
  \end{itemize}
\end{frame}

\begin{frame}[label=thanks]{Open Problems}
  \begin{itemize}
  \item<+-> Can we bootstrap in \alert<.>{sublinear homom ops} with
    \alert<.>{polynomial error}?
    \begin{itemize}
    \item<.-> Barrier in {\citationsize [GSW'13]}: single-bit
      encryption (no ``packing'')
    \end{itemize}

    \medskip
  \item<+-> \alert<.>{Circular security} for unbounded FHE?
    \begin{itemize}
    \item<.-> Does our representation help or hurt security?
    \end{itemize}
  \end{itemize}

  \medskip
  \onslide<+->
  \begin{center}\Large
    Thanks!
  \end{center}
\end{frame}


\end{document}

%%% Local Variables: 
%%% mode: latex
%%% TeX-master: t
%%% End: 
