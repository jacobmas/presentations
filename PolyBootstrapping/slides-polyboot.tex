% hack to deal with beamer/pgf bug
\RequirePackage{atbegshi}

% "draft" makes compilation faster; remove for final version
% shadow = put drop shadows behind blocks by default
% xcolor: svgnames imports lots of color names
% table allows for multi-color table rows
% pdftex is the compiler
\documentclass[shadow,xcolor=pdftex,svgnames,table,t]{beamer}

\usepackage{amsmath,amssymb,amsfonts}

% need to load tikz defs before beamer defs, for some reason
\input{tikzhead}
%% BOILERPLATE PRESENTATION STUFF

% an underlining package
\usepackage{ulem}
% use normal italics for emphasis
\normalem


% dingbats
\usepackage{pifont}

\renewcommand{\Check}{\ding{52}}
\newcommand{\Cross}{\ding{55}}
\newcommand{\Hand}{\ding{43}}
\newcommand{\GreenCheck}{{\textcolor{green}\Check}}
\newcommand{\RedCross}{{\textcolor{red}\Cross}}

% alias for citations
\newcommand{\citationsize}{\footnotesize}

\subject{Theoretical Computer Science}

% suppress ugly QEDs in proofs
\def\qedsymbol{}


% common stuff
% basic notation

\newcommand{\lamperp}{\Lambda^{\perp}}
\newcommand{\lam}{\Lambda}

% lattice
\newcommand{\lat}{\mathcal{L}}
% fundamental region
\newcommand{\piped}{\mathcal{P}}
% smoothing parameter
\newcommand{\smooth}{\eta}
% smoothing w/ epsilon
\newcommand{\smootheps}{\smooth_{\epsilon}}
% ball
\newcommand{\ball}{\mathcal{B}}
% cube
\newcommand{\cube}{\mathcal{C}}
% covering radius symbol
\newcommand{\cover}{\ensuremath{\mu}}
% Gram-Schmidt
\newcommand{\gs}[1]{\ensuremath{\widetilde{#1}}}
% GS minimum
\newcommand{\gsmin}{\ensuremath{\tilde{bl}}}
% volume operation
\DeclareMathOperator{\vol}{vol}
% Hermite normal form
\DeclareMathOperator{\hnf}{HNF}
% rank
\DeclareMathOperator{\rank}{rank}
% distance operator
\DeclareMathOperator{\dist}{dist}
% span operator
\DeclareMathOperator{\spn}{span}
% error function
\DeclareMathOperator{\erf}{erf}

% quantities that show up regularly
\newcommand{\wsln}{\ensuremath{\omega(\sqrt{\log n})}}
\newcommand{\wslm}{\ensuremath{\omega(\sqrt{\log m})}}

% support algorithms
\newcommand{\sampleZ}{\algo{Sample}\Z}
\newcommand{\sampleD}{\algo{SampleD}}
\newcommand{\tobasis}{\algo{ToBasis}}
\newcommand{\invert}{\algo{Invert}}
\newcommand{\genbasis}{\algo{GenBasis}}
\newcommand{\extbasis}{\algo{ExtBasis}}
\newcommand{\extlattice}{\algo{ExtLattice}}
\newcommand{\randbasis}{\algo{RandBasis}}
\newcommand{\gentrap}{\algo{GenTrap}}
\newcommand{\deltrap}{\algo{DelTrap}}
\newcommand{\noisy}{\algo{Noisy}}

% problems related to lattices
\newcommand{\svp}{\problem{SVP}}
\newcommand{\gapsvp}{\problem{GapSVP}}
\newcommand{\cogapsvp}{\problem{coGapSVP}}
\newcommand{\usvp}{\problem{uSVP}}
\newcommand{\sivp}{\problem{SIVP}}
\newcommand{\gapsivp}{\problem{GapSIVP}}
\newcommand{\cvp}{\problem{CVP}}
\newcommand{\gapcvp}{\problem{GapCVP}}
\newcommand{\cvpp}{\problem{CVPP}}
\newcommand{\gapcvpp}{\problem{GapCVPP}}
\newcommand{\bdd}{\problem{BDD}}
\newcommand{\gdd}{\problem{GDD}}
\newcommand{\add}{\problem{ADD}}
\newcommand{\incgdd}{\problem{IncGDD}}
\newcommand{\incivd}{\problem{IncIVD}}
\newcommand{\crp}{\problem{CRP}}
\newcommand{\gapcrp}{\problem{GapCRP}}
% problems on ideal lattices
\newcommand{\igvp}{\problem{IGVP}}
\newcommand{\incigvp}{\problem{IncIGVP}}
% avg-case stuff
\newcommand{\sis}{\problem{SIS}}
\newcommand{\isis}{\problem{ISIS}}
\newcommand{\ilwe}{\problem{ILWE}}
\newcommand{\lwe}{\problem{LWE}}
\newcommand{\rlwe}{\problem{RLWE}}
\newcommand{\lwr}{\problem{LWR}}
\newcommand{\rlwr}{\problem{RLWR}}
\newcommand{\dlwe}{\problem{DLWE}}
\newcommand{\lpn}{\problem{LPN}}
\newcommand{\Psibar}{\ensuremath{\bar{\Psi}}}

\newcommand{\extlwe}{\problem{Ext \text{\textendash} LWE}}
\newcommand{\extrlwe}{\problem{Ext \text{\textendash} RLWE}}

% Module-LWE and variants
\newcommand{\mlwe}{\problem{M \text{\textendash} LWE}}
\newcommand{\extmlwe}{\problem{Ext \text{\textendash}  M \text{\textendash} LWE}}
\newcommand{\mlwr}{\problem{M \text{\textendash}  LWR}}
\input{../vecmathead}
% "left-right" pairs of symbols

\usepackage{mathtools}

% inner product
\DeclarePairedDelimiter\inner{\langle}{\rangle}
% absolute value
\DeclarePairedDelimiter\abs{\lvert}{\rvert}
% a set
\DeclarePairedDelimiter\set{\{}{\}}
% parens
\DeclarePairedDelimiter\parens{(}{)}
% tuple, alias for parens
\DeclarePairedDelimiter\tuple{(}{)}
% square brackets
\DeclarePairedDelimiter\bracks{[}{]}
% rounding off
\DeclarePairedDelimiter\round{\lfloor}{\rceil}
% floor function
\DeclarePairedDelimiter\floor{\lfloor}{\rfloor}
% ceiling function
\DeclarePairedDelimiter\ceil{\lceil}{\rceil}
% length of some vector, element
\DeclarePairedDelimiter\length{\lVert}{\rVert}
% "lifting" of a residue class
\DeclarePairedDelimiter\lift{\llbracket}{\rrbracket}

\input{../bbhead}
% trace
\DeclareMathOperator{\trace}{Tr}
\DeclareMathOperator{\lcm}{lcm}
\DeclareMathOperator{\msb}{msb}
\renewcommand{\O}{\mathcal{O}}
\newcommand{\frakp}{\mathfrak{p}}

% fix weird spacing of \pmod, and introduce \pmod* command; see
% http://tex.stackexchange.com/questions/39221/removing-extra-space-with-pmod-command
\makeatletter
\renewcommand{\pod}[1]{\mathchoice
  {\allowbreak \if@display \mkern 18mu\else \mkern 8mu\fi (#1)}
  {\allowbreak \if@display \mkern 18mu\else \mkern 8mu\fi (#1)}
  {\mkern4mu(#1)}
  {\mkern4mu(#1)}
}
\makeatletter
\let\@@pmod\pmod
\DeclareRobustCommand{\pmod}{\@ifstar\@pmods\@@pmod}
\def\@pmods#1{\mkern4mu({\operator@font mod}\mkern 6mu#1)}
\makeatother
%%%% 

\title
{ Faster Bootstrapping with Polynomial Error
}

\author
{ {\Large Jacob Alperin-Sheriff}
  \and
  {\Large Chris Peikert}
}

\institute{School of Computer Science\\Georgia Tech}

\date
{
TBD
}

% restricts presentation to specified frames, for faster compilation;
% comment out the line to get the entire presentation

%\includeonlyframes{fhe,bootstrapping,results,unpacked-main}

\begin{document}

\begin{frame}[label=title]
  \titlepage
\end{frame}

\begin{frame}[label=fhe]{Fully Homomorphic Encryption {\footnotesize
      [RAD'78,Gen'09]}}
  \begin{itemize}
  \item<+-> FHE lets you do this:
    \begin{center}
      \begin{tikzpicture}
        \node (c) {$\color{Green}\fbox{$\mu$}$};
        \node[right=of c,algorithm=Green] (eval)
        {$\text{Eval}\parens*{f\;,\; {\color{Green} \fbox{$\mu$}}}$};
        \node[right=of eval] (out) {$\color{Green} \fbox{$f(\mu)$}$};

        \path
        (c) edge (eval)
        (eval) edge (out)
        ;
      \end{tikzpicture}
    \end{center}
    where $\abs{f(\mu)}$ and decryption time don't depend on
    $\abs{f}$.

    \medskip
    A cryptographic ``holy grail'' with tons of applications.
  \end{itemize}

  \onslide<+->
  \bigskip
  \begin{itemize}
  \item Naturally occurring schemes are ``somewhat
    homomorphic''~(SHE): they can only evaluate functions of an
    \alert<.>{\textit{a priori} bounded} depth.
  \end{itemize}

  \begin{center}
    \begin{tikzpicture}[node distance=.4cm]
      \node (c) {$\color{Green}\fbox{$\mu$}$};
      
      \node[right=of c,algorithm=Green] (eval)
      {$\text{Eval}\parens*{f, {\color{Green} \fbox{$\mu$}}}$};
      
      \node[right=of eval] (cprime) {$\color{Orange} \fbox{$f(\mu)$}$};

      \node[right=of cprime,algorithm=Green] (eval2)
      {$\text{Eval}\parens*{g, {\color{Orange} \fbox{$f(\mu)$}}}$};

      \node[right=of eval2] (cdp) {$\color{Red} \fbox{$g(f(\mu))$}$};

      \path
      (c) edge (eval)
      (eval) edge (cprime)
      (cprime) edge (eval2)
      (eval2) edge (cdp)
      ;
    \end{tikzpicture}
  \end{center}
\end{frame}

\begin{frame}[label=bootstrapping]{Bootstrapping: SHE $\to$ FHE {\footnotesize [Gen'09]}}
  \begin{itemize}
  \item<+-> \alert<.>{Homomorphically evaluates the SHE decryption
      function} to ``refresh'' a ciphertext ${\color{Red}
      \fbox{$\mu$}}$, allowing further homomorphic operations.

    \begin{center}
      \begin{tikzpicture}[every node/.style={node distance=.7cm}]
        \node (sk)
        {${\color{Green} \fbox{$sk$}}$};

        \node[right=of sk,algorithm=Green] (eval)
        {$\text{Eval}\parens*{\; f(x) =
            \text{Dec}_{x}({\color{Red}\fbox{$\mu$}})\; ,\;
            {\color{Green} \fbox{$sk$}} \;}$};

        \node[right=of eval] (out)
        {${\color{Orange} \fbox{$\mu$}}$};

        \path
        (sk) edge (eval)
        (eval) edge (out)
        ;

      \end{tikzpicture}
    \end{center}

    \onslide<+->
    \begin{itemize}
    \item The only known way of obtaining \alert<.>{unbounded} FHE.

      \medskip
    \item \alert<.>{Goal}: Efficiency!  Minimize depth $d$ and
      size $s$ of decryption ``circuit.''

      \medskip
    \item Best SHEs {\footnotesize [BGV'12]} can evaluate in time
      $\tilde{O}(d \cdot s \cdot \lambda)$.
    \end{itemize}

    \smallskip

  \item<+-> Intensive study, many techniques {\footnotesize
      [G'09,GH'11a,GH'11b,GHS'12b]}, but

    \alert<.>{still very inefficient} -- the main bottleneck in FHE,
    by far.

    \medskip 
  
  \item<+-> The asymptotically most efficient methods on ``packed''
    ciphertexts {\footnotesize [GHS'12a,GHS'12b]} are very complex,
    and appear practically worse than asymptotically slower methods.

    % (No public implementation of sub-quadratic methods.)
  \end{itemize}
\end{frame}

\begin{frame}[label=great]{Milestones in Bootstrapping}
  \begin{description}
  \item<+->[[Gen'09{]}:] $\tilde{O}(\lambda^{4})$ runtime

    \medskip
  \item<+->[[BGV'12{]}:] $\tilde{O}(\lambda^{2})$ runtime, or
    $\tilde{O}(\lambda)$ amortized over $\lambda$ ciphertexts

    \onslide<+->
    \medskip
    Mainly via improved SHE homomorphic capacity.

    \medskip Amortized method requires ``exotic'' plaintext rings,
    emulating $\Z_{2}$ arithmetic in $\Z_{p}$.

    \medskip 
  \item<+->[[GHS'12b{]}:] $\tilde{O}(\lambda)$ runtime, for ``packed''
    plaintexts.  \alert<.>{Declare victory?}
  \end{description}
  %
  \onslide<+->
  \begin{center}
    \begin{tikzpicture}
      \matrix[algorithm=Green] (Dec) {
        \node {Dec circuit}; \\
        \node[algorithm=Purple] (modPhi) {${} \bmod \Phi_{m}(X)$}; \\
      };

      \node[algorithm=Black,right=of Dec,align=center] (compiler)
      {[GHS'12a]\\compiler};

      \node[algorithm=Red,right=of compiler,align=center] (boot)
      {Bootstrapping\\Procedure};

      \path
      (Dec) edge (compiler)
      (compiler) edge (boot)
      ;
    \end{tikzpicture}
  \end{center}
  %
  \begin{itemize}
  \item<+->[\RedCross] Log-depth mod-$\Phi_{m}(X)$ circuit is
    \alert<.>{complex}, w/large hidden constants.
  \item<+->[\RedCross\RedCross] {\footnotesize [GHS'12a]} compiler is
    \alert{very complex}, w/\alert{large polylog overhead} factor.
  \end{itemize}
\end{frame}

\end{document}

%%% Local Variables: 
%%% mode: latex
%%% TeX-master: t
%%% End: 
