% hack to deal with beamer/pgf bug
\RequirePackage{atbegshi}

% "draft" makes compilation faster; remove for final version
% shadow = put drop shadows behind blocks by default
% xcolor: svgnames imports lots of color names
% table allows for multi-color table rows
% pdftex is the compiler
\documentclass[shadow,xcolor=pdftex,svgnames,table,t]{beamer}

\usepackage{amsmath,amssymb,amsfonts}
\usepackage{array}

\newif\iflncs\lncstrue
% need to load tikz defs before beamer defs, for some reason
\input{tikzhead}
%% BOILERPLATE PRESENTATION STUFF

% an underlining package
\usepackage{ulem}
% use normal italics for emphasis
\normalem


% dingbats
\usepackage{pifont}

\renewcommand{\Check}{\ding{52}}
\newcommand{\Cross}{\ding{55}}
\newcommand{\Hand}{\ding{43}}
\newcommand{\GreenCheck}{{\textcolor{green}\Check}}
\newcommand{\RedCross}{{\textcolor{red}\Cross}}

% alias for citations
\newcommand{\citationsize}{\footnotesize}

\subject{Theoretical Computer Science}

% suppress ugly QEDs in proofs
\def\qedsymbol{}


% common stuff
\input{../crypthead}
% basic notation

\newcommand{\lamperp}{\Lambda^{\perp}}
\newcommand{\lam}{\Lambda}

% lattice
\newcommand{\lat}{\mathcal{L}}
% fundamental region
\newcommand{\piped}{\mathcal{P}}
% smoothing parameter
\newcommand{\smooth}{\eta}
% smoothing w/ epsilon
\newcommand{\smootheps}{\smooth_{\epsilon}}
% ball
\newcommand{\ball}{\mathcal{B}}
% cube
\newcommand{\cube}{\mathcal{C}}
% covering radius symbol
\newcommand{\cover}{\ensuremath{\mu}}
% Gram-Schmidt
\newcommand{\gs}[1]{\ensuremath{\widetilde{#1}}}
% GS minimum
\newcommand{\gsmin}{\ensuremath{\tilde{bl}}}
% volume operation
\DeclareMathOperator{\vol}{vol}
% Hermite normal form
\DeclareMathOperator{\hnf}{HNF}
% rank
\DeclareMathOperator{\rank}{rank}
% distance operator
\DeclareMathOperator{\dist}{dist}
% span operator
\DeclareMathOperator{\spn}{span}
% error function
\DeclareMathOperator{\erf}{erf}

% quantities that show up regularly
\newcommand{\wsln}{\ensuremath{\omega(\sqrt{\log n})}}
\newcommand{\wslm}{\ensuremath{\omega(\sqrt{\log m})}}

% support algorithms
\newcommand{\sampleZ}{\algo{Sample}\Z}
\newcommand{\sampleD}{\algo{SampleD}}
\newcommand{\tobasis}{\algo{ToBasis}}
\newcommand{\invert}{\algo{Invert}}
\newcommand{\genbasis}{\algo{GenBasis}}
\newcommand{\extbasis}{\algo{ExtBasis}}
\newcommand{\extlattice}{\algo{ExtLattice}}
\newcommand{\randbasis}{\algo{RandBasis}}
\newcommand{\gentrap}{\algo{GenTrap}}
\newcommand{\deltrap}{\algo{DelTrap}}
\newcommand{\noisy}{\algo{Noisy}}

% problems related to lattices
\newcommand{\svp}{\problem{SVP}}
\newcommand{\gapsvp}{\problem{GapSVP}}
\newcommand{\cogapsvp}{\problem{coGapSVP}}
\newcommand{\usvp}{\problem{uSVP}}
\newcommand{\sivp}{\problem{SIVP}}
\newcommand{\gapsivp}{\problem{GapSIVP}}
\newcommand{\cvp}{\problem{CVP}}
\newcommand{\gapcvp}{\problem{GapCVP}}
\newcommand{\cvpp}{\problem{CVPP}}
\newcommand{\gapcvpp}{\problem{GapCVPP}}
\newcommand{\bdd}{\problem{BDD}}
\newcommand{\gdd}{\problem{GDD}}
\newcommand{\add}{\problem{ADD}}
\newcommand{\incgdd}{\problem{IncGDD}}
\newcommand{\incivd}{\problem{IncIVD}}
\newcommand{\crp}{\problem{CRP}}
\newcommand{\gapcrp}{\problem{GapCRP}}
% problems on ideal lattices
\newcommand{\igvp}{\problem{IGVP}}
\newcommand{\incigvp}{\problem{IncIGVP}}
% avg-case stuff
\newcommand{\sis}{\problem{SIS}}
\newcommand{\isis}{\problem{ISIS}}
\newcommand{\ilwe}{\problem{ILWE}}
\newcommand{\lwe}{\problem{LWE}}
\newcommand{\rlwe}{\problem{RLWE}}
\newcommand{\lwr}{\problem{LWR}}
\newcommand{\rlwr}{\problem{RLWR}}
\newcommand{\dlwe}{\problem{DLWE}}
\newcommand{\lpn}{\problem{LPN}}
\newcommand{\Psibar}{\ensuremath{\bar{\Psi}}}

\newcommand{\extlwe}{\problem{Ext \text{\textendash} LWE}}
\newcommand{\extrlwe}{\problem{Ext \text{\textendash} RLWE}}

% Module-LWE and variants
\newcommand{\mlwe}{\problem{M \text{\textendash} LWE}}
\newcommand{\extmlwe}{\problem{Ext \text{\textendash}  M \text{\textendash} LWE}}
\newcommand{\mlwr}{\problem{M \text{\textendash}  LWR}}
\input{../vecmathead}
% "left-right" pairs of symbols

\usepackage{mathtools}

% inner product
\DeclarePairedDelimiter\inner{\langle}{\rangle}
% absolute value
\DeclarePairedDelimiter\abs{\lvert}{\rvert}
% a set
\DeclarePairedDelimiter\set{\{}{\}}
% parens
\DeclarePairedDelimiter\parens{(}{)}
% tuple, alias for parens
\DeclarePairedDelimiter\tuple{(}{)}
% square brackets
\DeclarePairedDelimiter\bracks{[}{]}
% rounding off
\DeclarePairedDelimiter\round{\lfloor}{\rceil}
% floor function
\DeclarePairedDelimiter\floor{\lfloor}{\rfloor}
% ceiling function
\DeclarePairedDelimiter\ceil{\lceil}{\rceil}
% length of some vector, element
\DeclarePairedDelimiter\length{\lVert}{\rVert}
% "lifting" of a residue class
\DeclarePairedDelimiter\lift{\llbracket}{\rrbracket}

\input{../bbhead}
% MARGIN NOTES

\newif\ifnotes\notestrue
%\newif\ifnotes\notesfalse


\ifnotes
\usepackage{color}
\definecolor{mygrey}{gray}{0.50}
\newcommand{\noteby}[2]{{\textcolor{mygrey}{\footnotesize{\bf (#1:} {#2}{\bf ) }}}}
\newcommand{\noteswarning}{{\begin{center} {\Large WARNING: NOTES ON}\end{center}}}

\else

\newcommand{\noteby}[2]{{}}
\newcommand{\noteswarning}{{}}

\fi

\newcommand{\cnote}[1]{{\noteby{Chris}{#1}}}
\newcommand{\jnote}[1]{{\noteby{Jacob}{#1}}}
\newcommand{\dnote}[1]{{\noteby{Daniel}{#1}}}

% GENERAL COMPUTING STUFF

\newcommand{\bit}{\ensuremath{\set{0,1}}}
\newcommand{\pmone}{\ensuremath{\set{-1,1}}}

% asymptotic stuff
\DeclareMathOperator{\poly}{poly}
\DeclareMathOperator{\polylog}{polylog}
\DeclareMathOperator{\negl}{negl}
\newcommand{\Otil}{\ensuremath{\tilde{O}}}

% probability/distribution stuff
\DeclareMathOperator*{\E}{\mathbb{E}}
\DeclareMathOperator*{\Var}{Var}

\DeclareMathOperator{\trace}{Tr}
\DeclareMathOperator{\lcm}{lcm}

% assorted
\DeclareMathOperator*{\wt}{wt}

% hash functions
\newcommand{\calH}{\ensuremath{\mathcal{H}}}
\newcommand{\calX}{\ensuremath{\mathcal{X}}}
\newcommand{\calY}{\ensuremath{\mathcal{Y}}}

\newcommand{\compind}{\ensuremath{\stackrel{c}{\approx}}}
\newcommand{\statind}{\ensuremath{\stackrel{s}{\approx}}}
\newcommand{\perfind}{\ensuremath{\equiv}}

% font for general-purpose algorithms
\newcommand{\algo}[1]{\ensuremath{\mathsf{#1}}\xspace}

% font for general-purpose computational problems
\iflncs
\renewcommand{\problem}[1]{\ensuremath{\mathsf{#1}}\xspace}
\else
\newcommand{\problem}[1]{\ensuremath{\mathsf{#1}}\xspace}
\fi

% font for complexity classes
\newcommand{\class}[1]{\ensuremath{\mathsf{#1}}\xspace}

% complexity classes and languages
\renewcommand{\P}{\class{P}}
\newcommand{\BPP}{\class{BPP}}
\newcommand{\NP}{\class{NP}}
\newcommand{\coNP}{\class{coNP}}
\newcommand{\AM}{\class{AM}}
\newcommand{\coAM}{\class{coAM}}
\newcommand{\IP}{\class{IP}}
\newcommand{\SZK}{\class{SZK}}
\newcommand{\NISZK}{\class{NISZK}}
\newcommand{\NICZK}{\class{NICZK}}
\newcommand{\PPP}{\class{PPP}}
\newcommand{\PPAD}{\class{PPAD}}

\newcommand{\yes}{\ensuremath{\text{YES}}}
\newcommand{\no}{\ensuremath{\text{NO}}}

\newcommand{\Piyes}{\ensuremath{\Pi^{\yes}}}
\newcommand{\Pino}{\ensuremath{\Pi^{\no}}}

%\DeclareMathOperator{\trace}{Tr}
%\DeclareMathOperator{\lcm}{lcm}
\DeclareMathOperator{\msb}{msb}
\DeclareMathOperator{\lsb}{lsb}
\DeclareMathOperator{\rad}{rad}

\renewcommand{\O}{\mathcal{O}}
\newcommand{\ord}{\ensuremath{\text{ord}}}

\newcommand{\calA}{\ensuremath{\mathcal{A}}}
\newcommand{\calB}{\ensuremath{\mathcal{B}}}
\newcommand{\calC}{\ensuremath{\mathcal{C}}}
\newcommand{\calD}{\ensuremath{\mathcal{D}}}
\newcommand{\calE}{\ensuremath{\mathcal{E}}}
\newcommand{\calF}{\ensuremath{\mathcal{F}}}
\newcommand{\calG}{\ensuremath{\mathcal{G}}}
\newcommand{\calI}{\ensuremath{\mathcal{I}}}
\newcommand{\calJ}{\ensuremath{\mathcal{J}}}
\newcommand{\calK}{\ensuremath{\mathcal{K}}}
\newcommand{\calL}{\ensuremath{\mathcal{L}}}
\newcommand{\calM}{\ensuremath{\mathcal{M}}}
\newcommand{\calN}{\ensuremath{\mathcal{N}}}
\newcommand{\calO}{\ensuremath{\mathcal{O}}}
\newcommand{\calP}{\ensuremath{\mathcal{P}}}
\newcommand{\calQ}{\ensuremath{\mathcal{Q}}}
\newcommand{\calR}{\ensuremath{\mathcal{R}}}
\newcommand{\calS}{\ensuremath{\mathcal{S}}}
\newcommand{\calT}{\ensuremath{\mathcal{T}}}
\newcommand{\calU}{\ensuremath{\mathcal{U}}}
\newcommand{\calV}{\ensuremath{\mathcal{V}}}
\newcommand{\calW}{\ensuremath{\mathcal{W}}}
\newcommand{\calZ}{\ensuremath{\mathcal{Z}}}

\newcommand{\frakd}{\mathfrak{d}}
\newcommand{\frakp}{\mathfrak{p}}


%\algnotext{EndFor}
%\algnotext{EndIf}
%\algnotext{EndWhile}

% fix weird spacing of \pmod, and introduce \pmod* command; see
% http://tex.stackexchange.com/questions/39221/removing-extra-space-with-pmod-command
\makeatletter
\renewcommand{\pod}[1]{\mathchoice
  {\allowbreak \if@display \mkern 18mu\else \mkern 8mu\fi (#1)}
  {\allowbreak \if@display \mkern 18mu\else \mkern 8mu\fi (#1)}
  {\mkern4mu(#1)}
  {\mkern4mu(#1)}
}
\makeatletter
\let\@@pmod\pmod
\DeclareRobustCommand{\pmod}{\@ifstar\@pmods\@@pmod}
\def\@pmods#1{\mkern4mu({\operator@font mod}\mkern 6mu#1)}
\makeatother
\DeclareFontFamily{U}{mathb}{\hyphenchar\font45}
\DeclareFontShape{U}{mathb}{m}{n}{
  <5> <6> <7> <8> <9> <10> gen * mathb
  <10.95> mathb10 <12> <14.4> <17.28> <20.74> <24.88> mathb12
}{}
\DeclareSymbolFont{mathb}{U}{mathb}{m}{n}

\DeclareFontFamily{U}{mathx}{\hyphenchar\font45}
\DeclareFontShape{U}{mathx}{m}{n}{
  <5> <6> <7> <8> <9> <10>
  <10.95> <12> <14.4> <17.28> <20.74> <24.88>
  mathx10
}{}
\DeclareSymbolFont{mathx}{U}{mathx}{m}{n}

\DeclareMathSymbol{\boxplus}      {2}{mathb}{"60}
\DeclareMathSymbol{\boxminus}     {2}{mathb}{"61}
\DeclareMathSymbol{\boxtimes}     {2}{mathb}{"62}
\DeclareMathSymbol{\boxdot}       {2}{mathb}{"64}
\DeclareMathSymbol{\boxcirc}      {2}{mathb}{"65}
\DeclareMathSymbol{\boxslash}     {2}{mathb}{"6D}
\DeclareMathSymbol{\boxbackslash} {2}{mathb}{"6E}
\DeclareMathSymbol{\bigboxplus}      {1}{mathx}{"D0}
\DeclareMathSymbol{\bigboxminus}     {1}{mathx}{"D1}
\DeclareMathSymbol{\bigboxtimes}     {1}{mathx}{"D2}
\DeclareMathSymbol{\bigboxdot}       {1}{mathx}{"D4}
\DeclareMathSymbol{\bigboxcirc}      {1}{mathx}{"D5}


\newcommand{\GL}{\ensuremath{\text{GL}}}
% trace
\DeclareMathOperator{\trace}{Tr}
\DeclareMathOperator{\lcm}{lcm}
\DeclareMathOperator{\msb}{msb}
\renewcommand{\O}{\mathcal{O}}
\newcommand{\frakp}{\mathfrak{p}}
\newcommand{\Blue}[1]{{\color{Blue}#1}}
\newcommand{\Red}[1]{{\color{Red}#1}}
\newcommand{\Purple}[1]{{\color{Purple}#1}}
\newcommand{\Green}[1]{{\color{Green}#1}}
%\newcommand{\poly}{\text{poly}}

% fix weird spacing of \pmod, and introduce \pmod* command; see
% http://tex.stackexchange.com/questions/39221/removing-extra-space-with-pmod-command
\makeatletter
\renewcommand{\pod}[1]{\mathchoice
  {\allowbreak \if@display \mkern 18mu\else \mkern 8mu\fi (#1)}
  {\allowbreak \if@display \mkern 18mu\else \mkern 8mu\fi (#1)}
  {\mkern4mu(#1)}
  {\mkern4mu(#1)}
}
\makeatletter
\let\@@pmod\pmod
\DeclareRobustCommand{\pmod}{\@ifstar\@pmods\@@pmod}
\def\@pmods#1{\mkern4mu({\operator@font mod}\mkern 6mu#1)}
\makeatother
%%%% 

\title
{ Short Signatures with Short Public Keys\\From Homomorphic Trapdoor Functions
}

\author
{ {\Large Jacob Alperin-Sheriff}
}

\institute{School of Computer Science\\Georgia Tech}

\date
{

}

% restricts presentation to specified frames, for faster compilation;
% comment out the line to get the entire presentation

%\includeonlyframes{tags,compg}
%\includeonlyframes{PHTDFtoSigs,Reduction}%PHTDF,OurPHTDF,PHTDFtoSigs}

\begin{document}

%%% put less vertical space around displayed equations
\setlength{\abovedisplayskip}{8pt plus 2pt minus 1pt}
\setlength{\belowdisplayskip}{8pt plus 2pt minus 0pt}
\setlength{\abovedisplayshortskip}{0pt plus 2pt}
\setlength{\belowdisplayshortskip}{8pt plus 2pt minus 1pt}

\begin{frame}[label=title]
  \titlepage
\end{frame}

\begin{frame}[label=stateless]{Stateless Standard-Model Signature Schemes}
\medskip
\begin{tikzpicture}
\node  (sign) at (1,1) [draw] {\textbf{Sign}};
\node (msg) [left=of sign] {\citationsize Message }; 
\node (sk) [above=4mm of sign] {\citationsize Signing Key };
\node (ver) [right=4cm of sign,draw] {\textbf{Verify}};
\node (vk) [above=4mm of ver] {\citationsize Verification Key };
\node (acc) [right=6mm of ver] {\citationsize Accept/Reject };
%\node (sig) [above=4mm] {\citationsize Signatures };
\path[->] (msg) edge (sign)
                 (sk) edge (sign)
                 (sign) edge[dotted] node[above]  {\scriptsize
                   Signature} (ver)
                 (vk) edge (ver)
                 (ver) edge (acc);
                 

\end{tikzpicture}
\medskip
\begin{itemize}

\item<+-> Want \alert<.>{short} public key, secret key, signatures
\smallskip
\item<+-> Under classical number-theoretic assumptions [\citationsize
  Wat'09,HW'09]:
\begin{itemize}
\smallskip
\item<+-> \alert<.>{Constant} number of group elements in sk, sigs, vk
\smallskip
\item<.-> \alert<.>{Linear} size in security parameter $\lambda$
\end{itemize}
\smallskip
\item<+-> Under lattice-based assumptions (in ring setting)
\begin{itemize}
\item<+-> Trapdoors and pre-images already have \alert<.>{quasilinear} ($\tilde{O}(\lambda)$) size
\smallskip
\item<+-> Best schemes require:
\smallskip
\begin{itemize}
\item<.-> \alert<.>{Logarithmic} number of pre-images in signatures [\citationsize BHJ+14]
\smallskip
\item<.-> \alert<.>{Logarithmic} number of trapdoors in the public key [\citationsize DM14]
\end{itemize}
\smallskip
\item<+-> Can we do better?
\end{itemize}
\end{itemize}
\end{frame}
\begin{frame}[label=results]{Our Results}
\begin{itemize}
\item<+-> \alert<.>{Constant} number of trapdoors in public key, \alert<.>{short
  signatures}
\medskip
\item<+-> Starting Point: DM14 signature scheme
\smallskip
\begin{itemize}
\item<+-> DM14 used \alert<.>{linear homomorphisms} over trapdoor functions
\smallskip
\item<+-> \alert<.>{Our idea: }Use \alert<.>{full homomorphisms} over
  trapdoor functions
\smallskip

\end{itemize}
\item<+-> Comparison to previous work ($d=\omega(\log{\log{n}})$)
\smallskip
%\end{itemize}
\begin{table}
%\citationsize
\begin{tabular}{|l|cccc|}
\hline
Scheme&pk &sk &Sig. &SIS param \\
&$R_q^{1 \times k}$ mat.&$R_q^{k \times k}$ mat.&$R_q^{k}$
vec.&$\beta$\\
\hline
Boy10,MP12&$n$&$n$&$1$&$\tilde{\Omega}(n^{5/2})$\\
BHJ+14&$1$&$1$&$d$&$\tilde{\Omega}(n^{5/2})$\\
DM14&$d$&$1$&$1$&$\tilde{\Omega}(n^{7/2})$\\
\hline
This work&1&1&1&$\tilde{\Omega}(d^{2d}\cdot
n^{11/2})$\\
\hline
\end{tabular}
\end{table}
\smallskip
%\begin{itemize}
%\item<+-> Scheme secure against adversary making at most $2^{2^d}$ queries.
\item<+-> SIS param can be (large) poly-sized if we set $d=O(\log{n}/\log{\log{n}})$
\end{itemize}
% Discuss the basic concrete results, i.e. constant number of matrices,
% short signatures, assumption, basic idea of using all homomorphism
% instead of just linear or additive
\end{frame}
\begin{frame}[label=outline]{Construction Outline}
\begin{center}
\begin{tikzpicture}[old/.style={align=flush center, text
    width=7em,draw,minimum height=1.5cm},
newstuff/.style={align=flush center, text width=7em,draw,minimum height=1.5cm, fill=yellow!20}
]
\onslide<2->{
\node (homtrap) at (0,0) [rectangle,old] {Homomorphic Trapdoor Funcs
  {[\citationsize BGG+14,GVW14]}};
\node (PHTDF) at (2,-2.5) [rectangle,newstuff] {Puncturable
  Homomorphic Trapdoor Funcs};

\node (thiswork) at (4,-5) [rectangle,newstuff] {Our Signature Scheme};
}
\node (punctrap) at (4,0) [rectangle,old] {Puncturable Trapdoor Funcs
  {[\citationsize Boy10,MP12]}};

\node (conguess) at (8,0) [rectangle,old] {Confined Guessing
  {[\citationsize BHJ+14]}};

\node (DM) at (6,-2.5) [rectangle,old] {DM Signature Scheme
  {[\citationsize DM14]}};

\path[->]
(punctrap) edge (DM)
(conguess) edge (DM);

\onslide<2->{
\path[->] 
(homtrap) edge (PHTDF)
(punctrap) edge (PHTDF)
(conguess) edge [bend left=60] (thiswork) 
(PHTDF) edge (thiswork);

};

\end{tikzpicture}
\end{center}
%\onslide<+->
\end{frame}
\begin{frame}[label=PHTDF]{Puncturable Homorphic Trapdoor Functions (PHTDF)}
%\begin{center}
\begin{tikzpicture}[bob/.style={align=flush center, text width=10em},
eval/.style={draw,rectangle,fill=green!20,align=flush center}
]
\node (basenode) {};

\onslide<6|handout:0>{
\draw[red, line width=4pt,rounded corners,opacity=1] (4.95cm,2.10cm) rectangle (9.7cm,-.35cm);
\draw[red, line width=4pt,opacity=1] (4.99cm,-0.32cm) -- (9.66cm,2.07cm);
}
\onslide<7-8>{
\node (ai) [above=-.45cm of basenode] {\small $(a_1,\ldots,a_n)$};
\node (ri) [above=1.15cm of basenode] {\small $\Red{(r_1,\ldots,r_n)}$};


\node (evaltd) [right=1.5cm of ri,eval] {$\text{Eval}_{pk}^{td}$};
\node (evalfn) [right=1.5cm of ai,eval] {$\text{Eval}_{pk}^{fn}$};
\node (tempnode) [below=.3cm of evaltd] {};

\node (g) [left=.9cm of tempnode] {$g$};

\path[->]
  (ri) edge (evaltd)
  (ai) edge [bend left] (evaltd)
  (ai) edge (evalfn)
  (g) edge (evaltd)
  (g) edge (evalfn);
}


%\onslide<1-5 
\node (f1) [right=5cm of basenode] 
  {\large $f_{pk,a,x}(u)$};
\node (v) [right=2cm of f1] {\large $v$};

\onslide<1,4->{
\path[->,dashed]
   (f1) edge (v);
}

\node (t) [above=.2cm of f1,align=left,minimum width=10em] {\only<2-4|handout:0>{$t={\tilde{t}}^{-1}$}\only<5-6|handout:0>{$t=0$}\only<8>{$t=g(\cdot)$}};

\onslide<3-4,8|handout:1>{
\path[->,dashed]
   (v) edge [bend right] node (r) [above,sloped] {\large $\Red{r}$} (f1);
}
% \onslide<4-5>{
% \node (t) [above=.2cm of f1] {$t=0$};
% }
\onslide<6|handout:0>{
\node (f2) [above=1cm of f1] {\large $f_{pk,a,x'}(u')$};

\path[->]
   (f2) edge[dashed] (v);
}
\onslide<8>{
\path[->]
  (evalfn) edge (f1)
  (evaltd) edge [bend left=20] (r);

\path[->,dotted]
  (evalfn) edge [bend left=45]  (t);
}

% Add a crossout for slide 3 to signify collision resistance
\end{tikzpicture}
%\end{center}
\begin{itemize}
\item<+-> Trapdoor functions $a$ with associated (hidden) \alert<.>{tag} $t \in \calT$.
\smallskip
\item<+-> Tag $t$ is \alert<.>{invertible}: \onslide<+-> can invert $f$ with trapdoor $r$.
\smallskip
\item<+-> Distributional Equivalence:
\[(r,u \gets \mathcal{U}, v \gets f_{pk,a,x}(u)) \statind (r,u \gets f^{-1}_{pk,a,x}(v),v \gets \mathcal{V})\]
\item<+-> Tag $t=0$: trapdoor function $a$ is ``punctured:''
\begin{itemize}
\smallskip
\item<+-> $f_{pk,a,\cdot}(\cdot)$ becomes \alert<.>{collision
    resistant}.
\end{itemize}
\smallskip
\item<+-> Homomorphic Properties
\begin{itemize}
\smallskip
\item<.-> Can evaluate funcs $g$ over tags $t_i$ associated with $a_i$
\smallskip
\item<+-> Yields new trapdoor function $a$ with tag $t$, trapdoor $r$
\end{itemize}
\end{itemize}
\end{frame}
\begin{frame}[label=OurPHTDF]{Lattice-Based Construction of PHTDFs}
\begin{tikzpicture}[bob/.style={align=flush center, text width=10em},
eval/.style={draw,rectangle,fill=green!20,align=flush center}]
\node (basenode) {};

\node (f1) [right=1cm of basenode] 
  {\large \only<1-5|handout:1>{$f_{(\matA,\matB),-\matA\Green{\matR}+t\matG,\vecx}(\Green{\vecu})$}\only<6-|handout:0>{$f_{(\matA,\matB),-\matA\Orange{\matR^*}+t\matG,\vecx}(\Orange{\vecu})$}};
%}\only<3>{$f_{(\matA,\matB),-\matA\Green{\matR},\vecx}(\Green{\vecu})$}};
\node (v) [right=1.5cm of f1] {\small \only<1-3|handout:1>{$\vecv:=[\matA  \mid -\matA\Green{\matR}+t\matG]\Green{\vecu}+\matB\vecx$}\only<4-|handout:0>{$\vecv$}};

%\only<3>{$\vecv$}


\path[->,dashed]
   (f1) edge (v);

\onslide<2,4->{
\path[->,dashed]
   (v) edge [bend right] node (r) [above,sloped] {\only<2,4-5|handout:1>{$\Green{\matR}$}\only<6-|handout:0>{$\Orange{\matR^*}$}} (f1);
}


% \onslide<3>{
% \node (f2) [above=.5cm of f1] {\large $f_{(\matA,\matB),-\matA\Green{\matR}}(\vecx',\Green{\vecu'})$};
% \path[->,dashed]
%   (f2) edge (v);

% \draw[red, line width=4pt] (3.5cm,.55cm) ellipse (3.2cm and 1.6cm);
% \draw[red, line width=4pt] (1.5cm,-0.6989cm) -- (5.5cm,1.8489cm);
% }
% \onslide<4-5>{
% \node (t) [above=.2cm of f1] {$t=0$};
% }


% Add a crossout for slide 3 to signify collision resistance
\end{tikzpicture}
\begin{itemize}
\item<+-> Construction itself is simple extension of MP12 trapdoors
\smallskip
\item<+-> \alert<.>{Short} $\Green{\matR}$ lets us sample \alert<.>{short}
  preimage $\Green{\vecu}$ for a given $\vecv-\matB\vecx$.
\smallskip
\item<+-> \alert<.>{Collision resistance} when punctured follows from
  SIS.
\item<+-> Tags may be arbitrary $n \times n$ \alert<.>{matrices}
\begin{itemize}
\smallskip
\item<+-> For trapdoor multiplication, at least one must be scalar multiple of
  $\matI$.
\smallskip
\end{itemize} 
%\smallskip
\item<+-> Trapdoor growth from homomorphic computations:
\smallskip
\begin{itemize}
    \item<.-> \textbf{Homom Addition:} Trapdoor grows \alert<.>{additively}.
      \smallskip
    \item<+-> \textbf{Homom Multiplication:} Trapdoor grows \alert<.>{asymmetrically} in $\Green{\matR_1},\Green{\matR_2}$
\vspace{-5pt}
\[\Orange{\matR^*} := \Green{\matR_1}\cdot\Green{\poly(n)}+t_1\Green{\matR_2}\]
%\smallskip
\end{itemize}
\vspace{-20pt}
\item<+-> Larger Trapdoors $\to$ larger pre-images, larger SIS solutions.
%\item<+-> Goal: Compute desired function with minimal trapdoor growth 
\end{itemize}
\end{frame}
\begin{frame}[label=PHTDFtoSigs]{Signatures from PHTDFs}
\begin{tikzpicture}[bob/.style={align=flush center, text width=10em},
eval/.style={draw,rectangle,fill=green!20,align=flush center}
]
\node (basenode) {};

\node (ai) [above=-.45cm of basenode] {\small $(a_1,\ldots,a_n)$};
\visible<1-2,5-|handout:0>{
\node (ri) [above=1.15cm of basenode] {\small $\Red{(r_1,\ldots,r_n)}$};
}

\node (evaltd) [right=1.5cm of ri,eval] {$\text{Eval}_{pk}^{td}$};
\node (evalfn) [right=1.5cm of ai,eval] {$\text{Eval}_{pk}^{fn}$};
\node (tempnode) [below=.3cm of evaltd] {};

\onslide<1-2,5-6|handout:0>{
\path[->]
  (ri) edge (evaltd)
  (ai) edge [bend left] (evaltd);
}

\path[->]
  (ai) edge (evalfn);

\onslide<2-|handout:1>{
\node (g) [left=.9cm of tempnode] {\only<2-6|handout:1>{$g$}\only<7|handout:0>{$g^*$}};
\path[->]
  (g) edge (evalfn);
}



\node (f1) [right=5cm of basenode] 
  {\large \only<2->{$f_{pk,a,x}(\Green{u})$}};


\node (v) [right=1.7cm of f1] {$v$};
\visible<6-7|handout:0>{
\node (v2) [right=-0.15cm of v] {$=f_{pk,\bar{a},\bar{x}}(\Green{\bar{u}})$};
}

\onslide<7|handout:0>{
\node(f2) [above=1cm of f1] {\large $f_{pk,\bar{a},\bar{x}}(\Green{\bar{u}})$};
\path[->,dashed]
    (f2) edge [] (v);

}


\onslide<2,5-6|handout:0>{
\path[->]
   (v) edge [bend right,dashed] node (r) [above,sloped] {\large
     $\Red{r}$} (f1)
  (g) edge (evaltd)

  (evaltd) edge [bend left=20] (r);
}
\onslide<2-|handout:1>{ 

\node (t) [above=.2cm of f1,align=left,minimum width=10em] {\only<1-6|handout:1>{$t=g(\cdot)$}\only<7|handout:0>{$t=0$}};

\path[->]
  (evalfn) edge (f1);


\path[->,dotted]
  (evalfn) edge [bend left=45]  (t);

}

% \onslide<2>{
%}

\onslide<3,7|handout:0>{
\path[->,dashed]
   (f1) edge [] node (quest) [above] {\small \only<3>{?}} (v);
}

% Add a crossout for slide 3 to signify collision resistance
\end{tikzpicture}
\begin{block}{Signature Scheme}
\vspace{2pt}
\begin{description}
\item<+->[\textbf{Gen}$(1^\lambda)$:] Choose $vk=(pk,a_1,\ldots,a_n,v)$,
  $\Red{sk}=\Red{(r_1,\ldots, r_n)}$
\item<+->[\textbf{Sign}$(x)$:] Sample $g \gets \mathcal{G}$. Invert 
  to valid $\Green{u}$. Output $(\Green{u},g)$
\item<+->[\textbf{Ver}$(x,(\Green{u},g))$:] Verify that $\Green{u}$
  valid, $f_{pk,a,x}(\Green{u})=v$.
\vspace{2pt}
\end{description}
\end{block}
%\vspace{-2pt}
\begin{itemize}
\item<+-> Scheme security/correctness depend on properties of
  sampled $g \gets \mathcal{G}$
\smallskip
\item<+-> Actual Scheme: $t$ must always be \alert<.>{invertible.}
\medskip
\item<+-> Security reduction against $\Adv$ (with non-negligble probability):
\smallskip
\begin{enumerate}
\smallskip
\item<.-> $t=g(\cdot)$ must be invertible for \alert<.>{all but one} of queries made by
  $\Adv$.
\medskip
\item<+-> $\Adv$ chooses $g^*$ for forgery such that $t=g^*(\cdot)=0$
\end{enumerate}
\end{itemize}
\end{frame}
\begin{frame}[label=tags]{Tags and $g$}
\begin{tikzpicture}[fred/.style={line width=1pt}]

\onslide<2-4>{
\node (t) [] at (0cm,0cm) {\LARGE $t^{(g)}=$};

\foreach \x/\xtext in
{0cm/0,0.5cm/1,1cm/0,1.5cm/1,2cm/1,2.5cm/0,3cm/0,3.5cm/1,4.2cm/$\ldots$}
{
  \node (my\x) [] at (\x+1cm,-0.1cm) {\LARGE \xtext};
}
\node (mylast) at (5.6cm,0cm) {};
}



\onslide<2>{
%$t_1^(g)$
\draw [fred] (.85cm,-.4cm) -- (0.95cm,-.4cm) -- (1.00cm,-.45cm) --
(1.05cm,-.4cm) node
[very near start,below=.35cm] {\scriptsize $t_1^{(g)}$} -- (1.15cm,-.4cm);
\draw [line width=1pt] (.85cm,-.4cm) --(.85cm,-.35cm); 
\draw [line width=1pt] (1.15cm,-.4cm) --(1.15cm,-.35cm);


% $t_2^{g}$
\draw [fred] (.85cm,-.45cm) -- (.85cm,-.5cm) -- (1.35cm,-.5cm) --
(1.40cm,-.55cm) -- (1.45cm,-.5cm) node [very near
end,below=.3cm] {\scriptsize $t_2^{(g)}$} -- (1.65cm,-.5cm) --
(1.65cm,-.45cm);
}

\onslide<2-4>{
%t_3^{(g)}
\draw [fred] (.85cm,-.55cm) -- (.85cm,-.6cm) -- (2cm,-.6cm) -- 
(2.05cm,-.65cm) -- (2.1cm,-.6cm) node (t3) [very near end,below=.20cm]
{\scriptsize $t_3^{(g)}$} -- (2.65cm,-.6cm) -- (2.65cm,-.55cm);

\onslide<2>{
%t_4^{(g)}
\draw [fred] (.85cm,-.65cm) -- (.85cm,-.7cm) -- (4cm,-.7cm) --
(4.05cm,-.75cm) -- (4.1cm,-.7cm) node [very near end,below=.10cm]
{\scriptsize $t_4^{(g)}$} -- (4.65cm,-.7cm) -- (4.65cm,-.65cm);
}
}

\onslide<3-4>{
\node (poly) [right=6cm of t] {\LARGE $x+x^3$};
\path[->]
 (t3) edge [bend right=15] (poly);

}

\onslide<5->{

\node (a0) [] {\large $a_0$};

\node (tistar) [below=.5cm of a0] {$\hat{t}_{i^*}$};

\path
 (a0) edge [dotted,-] (tistar);

}


\end{tikzpicture}
\begin{itemize}
\item<+-> Each $g \in \mathcal{G}$ is uniquely specified by a tag $t^{(g)}
  \in \bit^{n}$
\smallskip
\item<+-> Tags decompose into prefixes $t_i^{(g)}$ of
  length $c_{i}=2^{i}$
\begin{itemize}
\item<+-> Prefixes embed into $\text{GF}(2^n)$ (higher coefficients
  set to 0)
\smallskip
\item<+-> \alert<.>{Key Point}: Embedding of $t^{(g)}-t^{(g')}$
  is invertible for $g \neq g'$.
\end{itemize}
%\smallskip
\item<+-> Trapdoor $a_{0}$ in public key has random
  tag prefix of length $c_{i^*}$.
\smallskip

%\smallskip
\item<+-> Choice of $i^*$ via \alert<.>{confined guessing}
  {\citationsize [BHJ+14,DM14]}
% use words  to describe 
 \begin{itemize}
 \item<+->  $\Adv$ makes $Q$ queries, succeeds with probability
   $\epsilon$
\smallskip 
 \item<.-> Choose smallest $i^*$ such that $2Q^2/\epsilon \leq
   2^{c_{i^*}}$
\smallskip 
\item<+-> $Q$ random tags will have distinct length
  $c_i^*$ prefixes with prob $\epsilon/2$
 \end{itemize}
\item<+-> $i^*$ must be kept secret from $\Adv$ 
\smallskip
 \end{itemize}
 \end{frame}

 \begin{frame}[label=compg]{Computing $g$ Homomorphically}
\begin{tikzpicture}
\onslide<1->{

\node (a0) [] {\large $a_0$};

\node (tistar) [below=.5cm of a0] {$\hat{t}_{i^*}$};

\path
 (a0) edge [dotted,-] (tistar);

}
\onslide<3,6->{

 \node (a1) [right=2cm of a0] {\large $a_1$ }; 
 \node (a2) [right=1cm of a1] {\large $a_2$ };
 \node (ldots1) [right=0.25cm of a2] {\large $\ldots$};
 \node (ai) [right=0.25cm of ldots1] {\large $a_{i^*}$};
 \node (ldots2) [right=0.25cm of ai] {\large $\ldots$};
 \node (ad) [right=0.25cm of ldots2] {\large $a_{d}$};
 
 \node (t1) [below=.6cm of a1] {$0$};
 \node (t2) [below=.6cm of a2] {$0$};
 %\node (tldots1) [below=.7cm of ldots1] {$0$};
 \node (ti) [below=.6cm of ai] {$1$};
% \node (tldots2) [below=.7cm of ldots2] { $0$};
\node (td) [below=.6cm of ad] {$0$};

\path[dotted,-]
  (a1) edge[-] (t1)
  (ai) edge[-] (ti)
 (a2) edge[-] (t2)
 % (ldots1) edge[-] (tldots1)
  %(ldots2) edge[-] (tldots2)
  (ad) edge[-] (td);
 
}

\onslide<4-> {
\node (b) [right=.4cm of a0] {\large $b$};
\node (istar) [below=.55cm of b] {$i^*$};

\path[dotted]
  (b) edge[-] (istar);
}

\onslide<6->{

\path
  (istar) edge [bend right] node [above, sloped] {\scriptsize $i^*\overset{?}{=} 1$} (t1)
  (istar) edge [bend right] node [above, near end, sloped] {\scriptsize
    $i^*\overset{?}{=} 2$} (t2)
  (istar) edge [bend right=30] node [above, very near end, sloped]
  {\scriptsize $i^*\overset{?}{=} i^*$} (ti)
 (istar) edge [bend right=30] node [above, very near end, sloped] 
 {\scriptsize $i^*\overset{?}{=} d$} (td);

}

\end{tikzpicture}
 \begin{itemize}
\item<+-> Evaluation of $g$:  Subtract $i^*$th \alert<.>{prefix} of
  $t^{(g)}$ from $\hat{t}_{i^*}$.
 \smallskip

\item<+-> Selection of prefix of $t^{(g)}$ needs to be done
  \alert<.>{homomorphically}
\smallskip
\item<+-> Old way: {\citationsize [DM14]} Length $d$ indicator vector
  for $i^*$ in pk
\smallskip
\item<+-> Our way: Store $i^*$ as tag in pk
\smallskip
\begin{itemize}
\item<+-> Degree $d-1$ interpolation polynomial computes
  $i\overset{?}{=}i^*$ 
\smallskip 
\item<+-> Can compute vector
  homomorphically from $b$ alone.
\smallskip
\end{itemize}
\item<+-> Asymmetric trapdoor growth in multiplication is key
\begin{itemize}
\smallskip
\item<+-> Yields $d^d\poly(n)$ growth instead of $d^dn^{\log{d}}$ growth
\end{itemize}
\end{itemize}
\end{frame}
\begin{frame}[label=conclusions]{Concluding Thoughts}
  \begin{itemize}
  \item<+-> Open problems:
\begin{itemize}
  \item<+-> Short signatures with short public keys with less trapdoor
    growth?
\smallskip
  \item<+-> Can techniques be extended to achieve fully secure IBE?
  \end{itemize}
\end{itemize}
  \bigskip
  \onslide<+->
  \begin{center}
    {\Large Hiring? Talk to me!}
  \end{center}
  
\end{frame}
\end{document}
