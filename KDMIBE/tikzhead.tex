% additional libraries for tikz and pgf
\usepackage{tikz}
\usetikzlibrary{matrix}
\usetikzlibrary{calc}
\usetikzlibrary{shadows}
\usetikzlibrary{scopes}
\usetikzlibrary{chains}
\usetikzlibrary{positioning}
\usetikzlibrary{fit}
\usetikzlibrary{backgrounds}
\usetikzlibrary{decorations.pathmorphing}
\usetikzlibrary{decorations.pathreplacing}
\usetikzlibrary{shapes.geometric}
\usetikzlibrary{shapes.symbols}
\usetikzlibrary{shapes.misc}

% color aliases
%\colorlet{Lattice}{Green}
%\colorlet{Set}{Red!10}

% global styles
\tikzset{
  % make drop shadows a bit softer
  every shadow/.style={opacity=.4,shadow xshift=.3ex,shadow yshift=-.3ex},
  % use \& to separate cells in matrices, avoiding fragility in beamer frames
  every matrix/.style={ampersand replacement=\&},
  % edges and joins with arrows and thicker
  every edge/.append style={->,thick},
  every join/.style={->,thick},
  % distance between nodes
  % node distance=1ex and 2ex,
}

% styles for drawing in Rn, especially lattices
\tikzset{
  % lattice transform must include "cm" dimension to work properly,
  % otherwise the x and y transforms become dependent
  Ztrans/.style={x={(1.5cm,1cm)},y={(.5cm,1.5cm)}},
  Zdual/.style={x={(1.5cm,-.5cm)},y={(.5cm,1cm)}},
  latttrans/.style={x={(1.8cm,.5cm)},y={(0.4cm,1.3cm)}},
  dualtrans/.style={x={(1.3cm,-.4cm)},y={(-.5cm,1.8cm)}},
  lattB/.style={x={(1.8cm,.5cm)},y={(0.4cm,1.4cm)}},
  dualB/.style={x={(1.4cm,-.4cm)},y={(-.5cm,1.8cm)}},
  shortB/.style={x={(.5cm,.2cm)},y={(-.1cm,1.7cm)}},
  latt2B/.style={x={(1cm,.1cm)},y={(.1cm,1.2cm)}},
  ulattB/.style={x={(2.0cm,.5cm)},y={(2.1cm,-.2cm)}},
  openball/.style={fill=structure.fg!20},
  ball/.style={openball,thin,draw=structure.fg},
  latt/.style={Lattice,fill},
  dist/.style={Brown,densely dashed,font=\footnotesize},
  offlabel/.style={Brown,inner sep=2pt,font=\tiny},
  axes/.style={->,Gray,style=very thin},
  fundamental/.style={Gray,fill opacity=.2},
  target/.style={fill,Brown},
  project/.style={thin,dashed},
  help lines/.style={Gray,very thin,step=.5cm},
  % expanding waves representing noise
  noise/.style={
    Lattice,
    opacity=.5,
    decorate,decoration={
      expanding waves,
      segment length=5pt,angle=10,
      post=moveto,post length=3pt
    },
  },
}

% lattice that can be dropped into any path.
% it would be sensible to set a default argument here, but
% commands with optional args don't parse correctly inside a
% tikz path
\newcommand*{\latt}[1]{%
  \foreach \a in {-4, ..., 4} {
    \foreach \b in {-4, ..., 4} {
      (\a,\b) circle (#1)
    }
  }
}

% style for functions mapping one point to another
\tikzset{map/.style={->,semithick}}

% style for an offset rounded box
\tikzset{showoff/.style={draw,fill=White,rounded corners,drop
    shadow,inner sep=6pt}}

% style for all algorithms
\tikzset{algorithm/.style={
    draw=#1!50!Black!60,solid,
    top color=White,bottom color=#1!50!Black!15,
    very thick,rounded corners,drop shadow,
    minimum size=.9cm,inner sep=6pt,outer sep=4pt
  }
}

% style for honest, malicious, trusted parties
\tikzset{honest/.style={algorithm=Green}}
\tikzset{oracle/.style={algorithm=Green}}
\tikzset{malicious/.style={algorithm=Red}}
\tikzset{trusted/.style={algorithm=Purple,inner sep=10pt}}

% style for notions
\tikzset{notion/.style={
    draw=#1!50!Black!60,solid,
    top color=White,bottom color=#1!50!Black!15,
    very thick,rounded corners,drop shadow,
    minimum height=6.5ex,inner sep=6pt,
  }
}

\tikzset{app/.style={notion=Blue}}
\tikzset{prim/.style={notion=Green}}
\tikzset{math/.style={notion=Red}}

% % style for a common reference string
\tikzset{crs/.style={draw,fill=White,cloud,cloud puffs=13,cloud ignores aspect,drop shadow}}
