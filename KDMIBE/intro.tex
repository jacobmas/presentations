\section{Introduction}
\label{sec:intro}

Traditional notions of secure encryption, starting with semantic (or
IND-CPA) security~\cite{DBLP:journals/jcss/GoldwasserM84}, assume that
the plaintext messages do not depend on the secret decryption key
(except perhaps indirectly, via the public key or other ciphertexts).
In many settings, this may fail to be the case.  One obvious scenario
is, of course, careless or improper key management: for example, some
disk encryption systems end up encrypting the symmetric secret key
itself (or a derivative) and storing it on disk.  However, there are
also situations in which key-dependent messages are used as an
integral part of an cryptosystem.  For example, in their anonymous
credential system, Camenisch and
Lysyanskaya~\cite{DBLP:conf/eurocrypt/CamenischL01} use a cycle of
key-dependent messages to discourage users from delegating their
secret keys.  More recently, Gentry's ``bootstrapping'' technique for
obtaining a fully homomorphic
cryptosystem~\cite{DBLP:conf/stoc/Gentry09} encrypts a secret key
under the corresponding public key in order to support unbounded
homomorphism; the cryptosystem therefore needs to be ``circular
secure.''  More generally, a system that potentially reveals
encryptions of any party's secret key under any user's public key
needs to be ``clique'' or ``key-dependent message'' (KDM) secure.
This notion allows for proving formal symbolic soundness of
cryptosystems having complexity-based security
proofs~\cite{DBLP:conf/esorics/AdaoBHS05}.

Since Boneh \etal's breakthrough
work~\cite{DBLP:conf/crypto/BonehHHO08} giving a KDM-secure encryption
scheme, in the standard model, from the Decision Diffie-Hellman
assumption, a large number of results (mostly positive) have been
obtained regarding circular-\jnote{Should the dash be there? I'm not
  sure}\cnote{yep that's usually the way it's done} and KDM-secure
encryption~\cite{DBLP:conf/tcc/HaitnerH09,DBLP:conf/crypto/ApplebaumCPS09,DBLP:conf/eurocrypt/BarakHHI10,DBLP:conf/crypto/BrakerskiG10,DBLP:conf/eurocrypt/Applebaum11,DBLP:conf/eurocrypt/MalkinTY11,DBLP:conf/tcc/BrakerskiGK11,DBLP:conf/crypto/BrakerskiV11}.
However, all these works have dealt only with the symmetric or
public-key settings; in particular, the question of circular/KDM
security for \emph{identity-based} cryptography has not yet been
considered.  Recall that in identity-based
encryption~\cite{DBLP:conf/crypto/Shamir84}, any string can serve as a
public key, and the corresponding secret keys are generated and
administered by a trusted Private Key Generator (PKG).

\paragraph{Circular security for IBE.}

In this work we define and construct a circular/KDM-secure
identity-based encryption (IBE) scheme.  KDM security is
well-motivated by some natural usage scenarios for IBE, as we now
explain.

Recall that identity-based encryption gives a natural and lightweight
solution to revocation, via expiring keys.  The lifetime of the
cryptosystem is divided into time periods, or ``epochs.''  An identity
string consists of a user's ``true'' identity (e.g., name)
concatenated with an epoch; when encrypting, one uses the identity for
the current epoch.  To support revocation, the PKG gives out a user's
secret key only for the current epoch, and only if the user is still
authorized to be part of the system.  Therefore, a user's privileges
can be revoked by simply refusing to give out his secret key in future
epochs; in particular, this revocation is transparent to the
encrypter.

The above framework makes it necessary for users to periodically get
new secret keys from the PKG, confidentially.  The most lightweight
method, which eliminates the need for the user to prove his identity
every time, is simply for the PKG to encrypt the new secret key under
the user's identity for the previous epoch.  This can be proved
secure, assuming the underlying IBE is CPA-secure, \emph{as long as
  there are no cycles of encrypted keys}.  However, if a user deletes
or loses an old secret key and wants to decrypt a ciphertext from the
corresponding epoch, it would be natural for the authority to provide
the old secret key encrypted under the user's identity for the current
epoch.  But because the current secret key has also been encrypted
(perhaps via a chain of encryptions) under the old identity, this may
be unsafe unless the IBE is KDM-secure.

%A similar scenario resulting in encrypted key cycles arises when using
%an IBE for \emph{forward security}.  To keep old messages secure in
%the event of an exposed secret key, the user would actively delete his
%old secret keys from previous epochs at the start of each new epoch;
%of course, then he can no longer read old messages on his own.  The
%most natural solution is for PKG to provide the user's old secret key
%encrypted under the identity for the current epoch, and for the user
%to destroy it after its use.  Again, without circular security it is
%not clear whether this solution is safe---even if the adversary never
%actually compromises a secret key!  More generally, any application in
%which the PKG or users encrypt secret keys under various identities in
%the system, without a strictly enforced acyclic structure, may have
%the possibility of publishing a key cycle and would need KDM security.

% NOT sure that this forward-security story makes enough sense.  If
% the adversary breaks in and can then request old secret keys, he
% gets to decrypt old messages too.

% KDM security can also be important when using an IBE for \emph{forward
%   security}.  Recall that forward security is a way of minimizing the
% damage caused by exposure of a secret key.  A forward-secure
% cryptosystem has different keys for each epoch, and guarantees that
% even if an adversary learns a secret key for some epoch, all messages
% encrypted in previous epochs remain secret (assuming the old secret
% keys have been reliably destroyed).

% Forward-secure encryption has
% been realized for symmetric
% encryption~\cite{DBLP:conf/ctrsa/BellareY03}, public-key encryption
% (using identity-based techniques)~\cite{DBLP:journals/joc/CanettiHK07}
% and (hierarchical) identity-based
% encryption~\cite{DBLP:conf/ccs/YaoFDL04}.

% Forward-secure encryption come with a significant downside in
% usability, however: once the old secret keys have been deleted, it is
% impossible to decrypt messages from previous epochs.  If a user wishes
% to be able to read old messages, she will either have to store them in
% the clear after the initial decryption, or keep them encrypted under
% some longer-lived secret key; either method negates the benefits of
% forward-secure encryption.  IBE can provide a useful compromise
% between forward security and the ability to read old messages, by
% having the PKG store the old secret keys and send them to users upon
% request.  Again, the most lightweight method would encrypt the old
% secret key under the user

% It has long been recognized that identity-based encryption can provide
% a useful compromise between forward-security and the ability to read
% old messages.  Recall the approach described by Boneh and
% Franklin~\cite{DBLP:journals/siamcomp/BonehF03} for user revocation,
% which builds key expiration into an IBE system via epoch-dependent
% identities that include the current epoch along with the user's
% ``true'' identity (e.g., name or e-mail address).  With each new
% epoch, the Private Key Generator (PKG) issues each user a new secret
% key; in order to a revoke a user, the PKG simply stops issuing new
% secret keys to that user.  Of course, it is vital that each new secret
% key be transmitted confidentially so that only the intended user can
% read it.  An elegant, lightweight solution is simply to encrypt the
% new secret key under the previous epoch's identity, and deliver it
% over a public channel.  (For initialization and periodic refreshes, a
% costlier process that verifies the user's identity and delivers the
% key over a secure channel can be used.)

% If a user never wants to read old messages, then as long as she
% securely deletes her old secret keys, it is easy to verify that the
% above-described techniques provide forward security.  To read old
% messages, the user can get old secret keys from the PKG again.  Just
% as with key updates, a natural and lightweight solution is simply for
% the user to request (over a public channel) an encryption of the old
% secret key under her identity for the current epoch.  (After
% decrypting the old message, she can delete the old key.)  However,
% notice that the system has now revealed an encrypted key cycle!
% Without circular security, it is unclear whether this natural usage
% scenario is actually secure.\footnote{Note that the system has also
%   somewhat weakened the forward-security guarantees in the event of a
%   break-in, because the attacker will be able to decrypt the old
%   secret key and all subsequent (but not prior) ones.  However, this
%   seems inherent to any system that brings old secret keys ``back from
%   the dead.''  Also, only those old keys (and subsequent ones) that
%   were encrypted under the \emph{current} epoch's identity are at
%   risk.}


\subsection{Our Contributions}
\label{sec:contributions}

As already mentioned, in this work we define a form of circular/KDM
security for identity-based encryption, and give a standard-model
construction based on the learning with errors (\lwe) problem, hence
on worst-case lattice problems via the reductions
of~\cite{DBLP:journals/jacm/Regev09,DBLP:conf/stoc/Peikert09}.

As in prior positive results on circular
security~\cite{DBLP:conf/crypto/BonehHHO08,DBLP:conf/crypto/ApplebaumCPS09,DBLP:conf/crypto/BrakerskiG10},
our definition allows the adversary to obtain encrypted ``key
cliques'' for affine functions of the users' secret keys.  More
precisely, for any tuple of identities $(id_{1}, \ldots, id_{d})$, the
attacker may adaptively query encryptions of $f(sk_{id_{1}}, \ldots,
sk_{id_{d}})$ under any of the identities $id_{j}$, where $f$ is any
affine function over the message space, and each $sk_{id_{i}}$ is a
secret key for identity $id_{i}$.  (This obviously specializes to
encryptions of a single secret key.)  Our attack model is in the style
of a ``selective identity'' attack, wherein the adversary must declare
the target identities $id_{1}, \ldots, id_{d}$ (but not the functions
$f$) before seeing the public parameters of the scheme.  While this is
not the strongest security notion we might hope for, it appears to at
least capture the main security requirements of the scenarios
described above, because encrypted key cycles are only ever published
for the same ``real-world'' identity at different time epochs.
Therefore, just as in a standard selective-identity attack for IBE,
the adversary is actually limited to attacking just a single
real-world identity, on a set of $d$ epochs (which could, for example,
include all valid epochs).  We also point out that by a routine hybrid
argument, security also holds when attacking a \emph{disjoint}
collection of identity cliques (that are named before seeing the
public parameters).

Our IBE construction is built from two components, both of which
involve some novel techniques.  First, we give an \lwe-based
\emph{public-key} cryptosystem that is clique secure (even for an
\emph{unbounded} number of users) for affine functions, and is
suitable for embedding into an IBE like the one
of~\cite{DBLP:conf/stoc/GentryPV08}.  Second, we construct a
lattice-based ``all-but-$d$'' trapdoor function that serves as the
main foundation of the IBE.  We elaborate on these two contributions
next.

\paragraph{Clique-secure public-key cryptosystem.}

We first recall that Applebaum
\etal~\cite{DBLP:conf/crypto/ApplebaumCPS09} showed that a variant of
Regev's so-called ``primal'' \lwe
cryptosystem~\cite{DBLP:journals/jacm/Regev09} is clique secure for
affine functions.  Unfortunately, this primal-type system does not
seem suitable as the foundation for identity-based encryption using
the tools of~\cite{DBLP:conf/stoc/GentryPV08}.  The first reason is
that the proof of clique security
from~\cite{DBLP:conf/crypto/ApplebaumCPS09} needs the users' public
keys to be completely independent, rather than incorporating a shared
random string (like the public parameters in an IBE system).  The
second reason is a bit more technical, and is already described
in~\cite{DBLP:conf/stoc/GentryPV08}: in primal-style systems, the
user-specific public keys are exponentially sparse pseudorandom values
(with unique secret keys), and it is difficult to design an
appropriate mapping from identities to valid public keys that actually
admit usable secret keys.

Therefore, we first need to obtain clique security for a so-called
``dual''-type cryptosystem (using the terminology
from~\cite{DBLP:conf/stoc/GentryPV08}), in which \emph{every}
syntactically valid public key has a functional secret key (actually,
many such secret keys) that can be extracted by the PKG.  It turns out
that achieving this goal is quite a bit more technically challenging
than it was for the ``primal''-style schemes.  This is primarily
because the KDM-secure scheme
from~\cite{DBLP:conf/crypto/ApplebaumCPS09} (like the earlier one
from~\cite{DBLP:conf/crypto/BonehHHO08}) has the nice property that
given the public key alone, one can efficiently generate
\emph{statistically well-distributed} encryptions of the secret key
(without knowing the corresponding encryption randomness).  This
immediately implies circular security for ``self-loops,'' and clique
security follows from a couple of other related techniques.

Unfortunately, this nice statistical property on ciphertexts does not
seem attainable for dual-style \lwe encryption, because now valid
ciphertexts are exponentially sparse and hard to generate without
knowing the underlying encryption randomness.  In addition, because
the adversary may obtain an \emph{unbounded} number of key-dependent
ciphertexts, we also cannot rely on any statistical entropy of the
secret key conditioned on the public key, as is common in the security
proofs of most dual-style cryptosystems.

We resolve the above issues by relying on computational assumptions
twice in our security proof, first when changing the way that
challenge ciphertexts are produced (i.e., by using knowledge of the
secret key), and then again when changing the form of the public key.
Notably, unlike prior works
(e.g.,~\cite{DBLP:conf/innovations/GoldwasserKPV10,DBLP:conf/tcc/DodisGKPV10})
in which ciphertexts in intermediate games are created by ``encrypting
with the (possibly information theoretically revealed) secret key,''
we are able to avoid the use of super-polynomially large noise to
``overwhelm'' the slight statistical difference between the two ways
of generating ciphertexts.  This lets us prove security under fully
polynomial lattice/\lwe assumptions, i.e., those involving a
polynomially bounded modulus $q$ and inverse error rate for the \lwe
problem, and therefore polynomial approximation factors for worst-case
lattice problems.  We do so by proving the hardness of a version of
the \emph{extended}-\lwe problem, as defined and left open by the
recent work of~\cite{DBLP:conf/crypto/ONeillPW11}.  We believe that
this result should be useful in several other contexts as well.

\paragraph{All-but-$d$ trapdoor functions.}

We use the clique-secure cryptosystem described above as the
foundation for a clique-secure IBE.  To make the cryptosystem
identity-based, as in~\cite{DBLP:conf/stoc/GentryPV08} we need to
embed a ``strong'' trapdoor into the public parameters so that the PKG
can extract a secret key for any identity.  Here we use the ideas
behind the tag-based algebraic construction
of~\cite{DBLP:conf/eurocrypt/AgrawalBB10}, and follow the somewhat
simpler and more efficient realization recently due
to~\cite{DBLP:conf/eurocrypt/MicciancioP12}.  We remark that these
trapdoor constructions are well-suited to security definitions in
which the adversary attacks a \emph{single} tag, because the trapdoor
can be ``punctured'' (i.e., made useless for extracting secret keys,
and useful for embedding an \lwe challenge) at exactly one
predetermined tag.  Unfortunately, this does not appear to be
sufficient for our purposes, because in the clique security game, the
adversary is attacking $d$ identities at once and can obtain challenge
ciphertexts under all of them.

To resolve the insufficiency of a single puncture, we extend the
trapdoor constructions
of~\cite{DBLP:conf/eurocrypt/AgrawalBB10,DBLP:conf/eurocrypt/MicciancioP12}
so that it is possible to puncture the trapdoor at up to $d$
arbitrary, prespecified tags.  To accomplish this, we show how to
statistically hide in the public key a degree-$d$ polynomial
$f(\cdot)$ over a certain ring~$\mathcal{R}$, so that $f(id_{i})=0$
for all the targeted tags (identities) $id_{i}$, while $f(id)$ is a
unit in $\mathcal{R}$ (i.e., is invertible) for all other identities.
The $d$ components of the public key can be combined so as to
homomorphically evaluate~$f$ on any desired tag.  The underlying
trapdoor is punctured exactly on tags $id$ where $f(id)=0$, and is
effective for inversion whenever $f(id)$ is a unit in $\mathcal{R}$.
Our construction is analogous to the one
of~\cite{DBLP:conf/pkc/ChatterjeeS06} in the setting of prime-order
groups with bilinear pairings (where arithmetic ``in the exponent''
happens in a field), and the all-but-$d$ lossy trapdoor functions
of~\cite{DBLP:conf/asiacrypt/HemenwayLOV11}.  However, since
lattice-based constructions do not work with fields or rings like
$\Z_{N}$, we instead use techniques from the literature on secret
sharing over groups and modules,
e.g.,~\cite{DBLP:journals/siamdm/DesmedtF94,Fehr98}.

We remark that, for technical reasons relating to the number of
``hints'' for which we can prove the hardness of the extended-\lwe
problem, we have not been able to prove the KDM-security of our IBE
under fully polynomial assumptions (as we were able to do for our
basic public-key cryptosystem).  We instead rely on the conjectured
hardness of \lwe for a slightly super-polynomial modulus $q$ and
inverse error rate $1/\alpha$, which translates via known
reductions~\cite{DBLP:journals/jacm/Regev09,DBLP:conf/stoc/Peikert09}
to the conjectured hardness of worst-case lattice problems for
slightly super-polynomial approximation factors, against slightly
super-polynomial-time algorithms.  Known lattice algorithms are very
far from disproving such conjectures.

\subsection{Open Problems}
\label{sec:open-probs}

Our work suggests several interesting problems.  Currently, our
all-but-$d$ trapdoor function has a key size that grows at least
linearly with $d$.  Finding a more efficient construction, or an
``all-but-many'' trapdoor function (which is punctured on
superpolynomially many tags), such as the one
of~\cite{DBLP:conf/eurocrypt/Hofheinz12}, would be particularly
useful.  Another natural extension would be to construct a KDM-secure
IBE scheme which can be proved fully secure, i.e., under an adaptive
choice of target identities.

It would also be interesting to construct a KDM-secure IBE from other
mathematical structures that support IBE, i.e., bilinear pairings and
quadratic residues.  Both have been used to build KDM-secure
encryption alone (in~\cite{DBLP:conf/crypto/BonehHHO08}
and~\cite{DBLP:conf/crypto/BrakerskiG10}, respectively), but by
imposing a special structure on secret keys; it is unclear to us
whether this structure can be incorporated into IBE schemes.

A less obvious extension would be to construct an IBE scheme that
satisfies notions of KDM security for the \emph{master} secret key,
instead of the user secret keys.  Under a sufficiently strong security
notion, such a scheme could be used to construct a KDM-CCA secure
scheme using the transformation
from~\cite{DBLP:journals/siamcomp/BonehCHK07}.  We note that in a
concurrent and independent work, Galindo
\etal~\cite{cryptoeprint:2012:142} solved (using bilinear pairings)
variants of these problems in which the number of challenge
ciphertexts is a priori bounded.

%%% Local Variables: 
%%% mode: latex
%%% TeX-master: "kdm-ibe"
%%% End: 
