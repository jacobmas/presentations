\section{Circular-Secure IBE}
\label{sec:Construction}

Our IBE scheme is a generalization of the efficient IBE scheme of
Agrawal \etal~\cite{DBLP:conf/eurocrypt/AgrawalBB10}.  Other than some
minor changes in the parameters, the main difference is the use of the
all-but-$d$ trapdoor construction, which allows us to ``puncture'' the
master public key at up to $d$ identities in the security proof.  The
scheme has parameters modulus $q$, message space $\Zp$ for some $p <
q$, dimension $m$, and Gaussian parameters $r$ and $\gamma$.  We give
example instantiations after describing the scheme.

\subsection{Construction}

The identity space for the scheme is $U\setminus \set{0} \subset
\calR$, where $U$, $\calR$ are constructed as in
Section~\ref{sec:all-but-d}.
\begin{itemize}
\item$\ibesetup(1^n, d)$: On input security parameter $1^n$ and secret
  key clique size $d$:
  \begin{enumerate}
  \item Sample $\matR \gets D_{\Z, \omega(\sqrt{\log{n}})}^{m{d}\times
      w}$, and for $i=0, \ldots, d-1$, choose uniformly random
    $\matA_i \gets \Zqmat{n}{md}$, $\vecy_i \gets \Zqvec{n}$ and let
    $\tildeA_i=-\matA_i\matR \in \Zqmat{n}{w}$.  (Note that this is
    simply calling the all-but-$d$ trapdoor construction from
    Section~\ref{sec:all-but-d} with an empty set of punctured tags.)
    Let \[ \matA
    := \begin{bmatrix}\matA_0\\\vdots\\\matA_{d-1}\end{bmatrix}, \quad
    \tildeA
    := \begin{bmatrix}\tildeA_0\\\vdots\\\tildeA_{d-1}\end{bmatrix} =
    -\matA \matR, \quad \vecy
    := \begin{bmatrix}\vecy_0\\\vdots\\\vecy_{d-1}\end{bmatrix}. \]
  \item The public key is $mpk=(\matA, \tildeA, \vecy)$. The master
    secret key is $\msk=(\matR)$.
  \end{enumerate}

\item $\ibeext(mpk, \msk, u)$ On input $mpk, \msk$ and $u \in U
  \setminus \{0\} \subseteq \calR$:
  \begin{enumerate}
  \item Let $\vec{u}^t := (u^0, u^1, \ldots, u^{d-1})$, $\barA_{u} =
    \vec{u}^t \cdot \matA$, $\vecy_{u} = \vec{u}^t \cdot \vecy$ and
    $\matA_{u} = [\vec{u}^{t} \cdot \matA \mid u^{d}\matG - 
    \vec{u}^{t}\cdot \tildeA] = 
    [\barA_{u} \mid u^d\matG-\barA_{u}\matR]$, as in
    Section~\ref{sec:all-but-d}.
  \item Sample $\vecz_{0} \gets D_{\Z,r}^{n}$, $\vecz_{1} \gets
    D_{\Lambda_{\vecz_{0}-\vecy_{u}}^{\perp}(\matA_{u}), r}$ using the
    preimage sampling algorithm (Lemma~\ref{lem:trapgen-sample}), so
    that $\vecy_{u} = \vecz_{0}-\matA_{u} \vecz_{1}$ (as in the
    public-key cryptosystem from Section~\ref{sec:KDMCPAscheme}).
    Output $sk_{u} := \vecz_{1}$.
    
    Note that the above is possible because $u^{d} \in \calR$ is a
    unit, and by our choice of $r$ below, because $s_{1}(\matR) =
    O(\sqrt{md} + \sqrt{w}) \cdot \wsln = O(\sqrt{md}) \cdot \wsln$
    with all but $\negl(n)$ probability by Lemma~\ref{Gaussianlemmas}.
  \end{enumerate}
  
\item $\ibeenc(mpk, u, \mu)$: On input master public key, identity $u
  \in U \setminus \{0\}$, and message $\mu \in \Zp$ do:
  \begin{enumerate}
  \item Let $\vec{u}^t := (u^0, u^1, \ldots, u^{d-1})$, $\matA_{u} =
    [\vec{u}^t \cdot \matA \mid u^d\matG+\vec{u}^t\cdot \tildeA] \in
    \Zqmat{n}{md+w}$, and $\vecy_{u} = \vec{u}^t \cdot \vecy$.
  \item Choose $\vecx_0 \gets D_{\Z, r}^{n}$, $\vecx_1^{(1)} \gets D_{\Z,
      r}^{md}, \vecx_1^{(2)} \gets D_{\Z, \gamma}^{w}$, $x_2 \gets D_{\Z,
      r}$. Let $\vecx_1^t = [(\vecx_1^{(1)})^t \mid (\vecx_1^{(2)})^t]$.
  \item Output the ciphertext $\vecc^t = \vecx_0^t[\matA_{u} \mid
    \vecy_{u}]+[\vecx_1^t \mid x_2] + [\veczero \mid p\cdot \mu]$.
  \end{enumerate}

\item $\ibedec(mpk, sk_{u} = \vecz_{1}, \vecc)$: output the $\mu \in
  \Zp$ such that $\vecc^t \left[
    \begin{smallmatrix}
      \vecz_{1} \\ 1
    \end{smallmatrix} \right]$ is closest to $p \cdot \mu$ modulo $q$.
\end{itemize}

\subsection{Parameters and Correctness}
\label{subsec:IBEparams}

We need most of the parameters to match the parameters from the
public-key encryption scheme, with the additional constraint that $r$
must be large enough that we can run the preimage sampling algorithm
(Lemma~\ref{lem:trapgen-sample}) in $\ibeext$.  Thus, we choose a
sufficiently large $m=\Theta(n \log q)$, $r = O(\sqrt{md}) \cdot
\wsln^{2}$, $\gamma=n^{\omega(1)}$ slightly superpolynomial in $n$,
$p=\gamma \cdot \poly(n)$ for a sufficiently large $\poly(n)$ term to
ensure correctness, and $q=p^2$.  We need $m = \Theta(n \log{q})$ so
that the stacked matrix $\matA \in \Zq^{nd \times md}$ is wide enough
so that $(\matA, \tildeA=\matA\matR)$ is statistically close to
uniform (by Lemma~\ref{Gaussianlemmas}) over our choice of $\matA$ and
$\matR$.

We now prove correctness. Let $\vecc^t \gets \ibeenc(mpk, u, \mu)$ be
a properly generated encryption of $\mu$ under identity $u$, and let
$\vecz_{1} \gets \ibeext(mpk, msk, u)$.  Then we have
\[ \vecc^t \left[
  \begin{smallmatrix}
    \vecz_{1} \\ 1
  \end{smallmatrix} \right] = \vecx_0^t \matA_{u} \vecz_{1} + \vecx_0^t
\vecy_{u} + \inner{\vecx_1, \vecz_{1}} + x_2 + p\cdot\mu = \inner{\vecx_0,
  \vecz_{0}} + \inner{\vecx_1, \vecz_{1}} + x_2 + p\cdot\mu.\] Thus,
decryption will be correct whenever $\abs{\inner{\vecx_0,
    \vecz_{0}}+\inner{\vecx_1, \vecz_{1}}+x_2} < p/2$.  By
Cauchy-Schwarz and Lemma~\ref{Gaussianlemmas}, this bound holds except
with probability negligible in $n$ (over the choice of all the random
variables), as required.

\subsection{Proof of Security}
\label{subsec:securityproof}

\begin{theorem}
  \label{thm:kdm-cpa-ibe}
  For the above parameters, the above IBE scheme is selective identity
  KDM-CPA secure with respect to the set of affine functions over
  $\Zp$, under the $\lwe_{q,\chi}$ assumption for $\chi=D_{\Z,r}$, and
  the KDM-CPA security of the system from
  Section~\ref{sec:KDMCPAscheme}.
\end{theorem}

\begin{proof}
  Our proof of security proceeds as follows. Game 0 is the actual
  attack game. In Game 1, we use the all-but-$d$ trapdoors
  construction from Section~\ref{sec:all-but-d} to construct the
  master public key, ``puncturing'' it at the targeted identities.
  Finally, in Game 2, we play the KDM-CPA security game against a
  challenger running the public-key encryption scheme from
  Section~\ref{sec:KDMCPAscheme} and use the outputs of the challenger
  to simulate Game 1. This requires some care because the IBE secret
  keys and ciphertexts have larger dimension by an additive term
  of~$w$ (the width of $\matG$).  To address this, we fill in the
  missing dimensions of the secret keys by choosing them ourselves,
  and use knowledge of the master secret key to fill in the missing
  dimensions of the ciphertexts (here is where we use the fact that
  noise with parameter $\gamma$ ``overwhelms'' noise with parameter
  $r$).  Selective identity KDM-CPA security then follows from the
  KDM-CPA security of the public-key encryption scheme.
  
  \cnote{TODO: make the notation in the proof match what's above.}
  \paragraph{Game 0.} This is the actual security game from
  Section~\ref{subsec:KDM}. For bit $\beta$, we respond to KDM queries
  $(f_{\matV_0, w_0} = \sum_{j \in
    [d]}\inner{\vecv_{j,0},\vecz_{j,1}}+w_{0}, f_{\matV_1, w_1} =
  \sum_{j \in [d]}\inner{\vecv_{j,1}, \vecz_{j,1}}+w_1, i)$ by
  encrypting $f_{\matV_{\beta}, w_{\beta}}$ under identity $u_{i}$. We
  respond with ciphertext $\vecc$, where (for $\vec{u}_i^t = (u_i^0,
  u_i^1, \ldots, u_i^{d-1})$)
  \[\vecc^t = \vecx_0^t[\vec{u}_i^t \cdot \matA \mid \vec{u}_i^t \cdot
  \tildeA \mid \vec{u}_i^t \cdot \vecy]+[\vecx_1^t \mid x_2] + [\veczero
  \mid p \cdot f_{\matV_{\beta}, w_{\beta}}].\]

  \paragraph{Game 1.} In this game, we use the all-but-$d$ trapdoor
  construction from Section~\ref{sec:all-but-d} to ``puncture" the
  public key at each of the $d$ challenge identities in a
  statistically indistinguishable manner.

  We first choose $d$ uniform random matrices $\matA_i^* \in \Zq^{n
    \times md}$ and master secret key $\matR \gets D_{\Z,
    \omega(\sqrt{\log{n}})}^{m{d}\times w}$. In order to successfully
  simulate the security game, we still need to know a secret key for
  each of the $d$ challenge identities. So, for $i \in [d]$, we choose
  the secret key for identity $u_i$ to be $\vecz_{i,1} \gets
  D_{\Z,r}^{md+w}$ and choose error $\vecz_{i,0} \gets
  D_{\Z,r}^{n}$. We then set $\vecy_i^* = \vecz_{i,0} - [\matA_i^*
  \mid -\matA_i^*\matR]\vecz_{i,1}$.

  Lemma~\ref{Gaussianlemmas} implies that this is statistically close
  to choosing the $\vecy_i^*$ uniformly at random and then sampling
  $\vecz_{i,1} \gets D_{\lamperp_{\vecz_{i,0}-\vecy_i^*}([\matA_i^*
    \mid -\matA_i^*\matR]), r}$.  Recalling that as a result of the
  all-but-$d$ construction, we will have that $[\matA_i^* \mid
  -\matA_i^*\matR] = [\vec{u_i}^t \cdot \matA \mid u_i^d\matG
  -\vec{u_i}^t \cdot \matA\matR]$ and that $\vecy_i^* = \vec{u_i}^t
  \cdot \vecy$, we see that the master public key and the secret keys
  for the challenge identities have been generated in a manner
  statistically indistinguishable from how they were generated in Game
  0.
    
  So, we invoke the all-but-$d$ trapdoor construction on $\matA_i^*,
  \vecy_i^*$, $\matR$ and identities $u_i$ (that we received from the
  adversary), and receive back $\matA_i, \tildeA_i =
  -\matA_i\matR+c_i\matG, \vecy_i$ for $i =0, \ldots, d-1$ which we
  make the master public key and give to the adversary.

  Secret key extractions and responses to KDM queries proceed normally
  (using the now ``punctured'' public key), and so responses to KDM
  queries are now of the form
  \begin{equation}\vecc^t = \vecx_0^t[\matA_i^{*} \mid
    -\matA_i^{*}\matR \mid \vecy_{i}^{*}]+[\vecx_1^t \mid x_2] +
    [\veczero \mid p \cdot
    f_{\matV_{\beta}, w_{\beta}}].
  \end{equation}
    
  \paragraph{Game 2.} In this game, we attack the KDM-CPA secure
  scheme described in Section~\ref{sec:KDMCPAscheme} and use it to
  simulate Game 1. The secret keys in that construction only
  correspond to the ``top parts'' (dimension $md$) of the secret keys
  in the IBE scheme (denoted below as $\vecz_{i}^{(1)}$). The ``bottom
  parts'' (dimension $w$) will be denoted $\vecz_{i}^{(2)}$, with
  similar notation for syndromes $\vecy_i^{(1)}, \vecy_i^{(2)}$.  We
  will be choosing the ``bottom parts'' of the keys on our own, as
  described below.
   
  \emph{Answering KDM queries.}  After receiving the $d$ challenge
  identities from the adversary, we begin playing the KDM-CPA security
  game (for the same values of $m$ and $d$) against a challenger.  The
  challenger sends us each user's public key as $(\matA_i^*,
  (\vecy_i^{*})^{(1)}=\vecz_{i,0}-\matA_i^*\vecz_{i,1}^{(1)})$ for $i
  \in [d]$, where the $\vecz_{i, 0}$ and $\vecz_{i,1}^{(1)}$ are
  secret.  To construct the full syndrome $\vecy_i^*$ for each
  challenge identity $u_i$ in the IBE scheme, we sample the ``bottom
  part'' $\vecz_{i,1}^{(2)} \gets D_{\Z, r}^{w}$, compute
  $(\vecy_i^{*})^{(2)} = \matA_i^{*}\matR\vecz_{i,1}^{(2)}$, and then
  set $\vecy_i^* = (\vecy_i^{*})^{(1)}+(\vecy_i^{*})^{(2)}$. Note that
  as in the previous game, the $\matA_i^*$, $\vecy_i^*$ remain
  statistically close to uniform by Lemma~\ref{Gaussianlemmas}.

  We then sample $\matR \gets D_{\Z,
    \omega(\sqrt{\log{n}})}^{m{d}\times w}$ as usual, and use
  $(\matA_i^*, \matR, \vecy_i^*, u_i)$ as the input to our all-but-$d$
  trapdoor construction, receiving back $\matA_i$, $\tildeA_i =
  -\matA_i\matR+c_i\matG$, and $\vecy_i$ for $i = 0, \ldots, d-1$,
  which we make the master public key.
    
  We continue to respond to secret key extraction queries as in Game
  1, using our all-but-$d$ trapdoor to extract secret keys for all
  non-challenge identities.

  We now show how to respond to the adversary's key-dependent message
  queries with encryptions of $f_{\matV_{\beta}, w_{\beta}}$, where
  $\beta$ parameterizes the KDM-CPA security game that we are playing
  (and is unknown to us because that scheme is KDM-CPA secure). To do
  so, we need to modify the functions $f_{\matV_{0}, w_0},
  f_{\matV_{1}, w_1}$ given to us by the adversary before passing them
  along to the KDM-CPA challenger. For $k \in \bit$, our modifications
  will add to the constant part of each function ($w_{k}$) the sum of
  the inner products of the ``bottom part'' of each affine function
  vector ($\vecv_{j, k}^{(2)}$) with the ``bottom part'' of each
  secret key ($\vecz_{j, 1}^{(2)}$). Thus, instead of only being an
  encryption of a function of ``top part'' of the secret key, the
  ciphertext we receive back from the KDM-CPA challenger will in fact
  be an encryption of $f_{\matV_{k}, w_{k}}$, which is exactly what we
  need.

  Concretely, to respond to key-dependent message queries
  $(f_{\matV_0, w_0}, f_{\matV_1, w_1}, i)$, we first
  let $\vecv'_{j, k} = \vecv_{j,k}^{(1)}$, $w'_{k} = w_{k}+\sum_{j \in
    [d]}\inner{\vecv_{j,k}^{(2)}, \vecz_{j,1}^{(2)}}$ for $k \in
  \bit$.  We then query the KDM-CPA challenger with \[(f'_ {\matV_{0},
    w_0} = \sum_{j \in [d]}\inner{\vecv'_{j, 0}, \vecz_{j,1}} + w'_0,
  f'_{\matV_1, w_1} = \sum_{j \in [d]}\inner{\vecv'_{j, 1},
    \vecz_{j,1}} + w'_1, i).\]  
    Since for $k \in \bit$, $\sum_{j \in [d]}\inner{\vecv'_{j, k}, \vecz_{j,1}^{(1)}}
    +\vecw_{k}' = \sum_{j\in [d]}\inner{\vecv_{j,k}, \vecz_{j,1}} + \vecw_{k} =
  f_{\matV_{k}, w_{k}}$, we will receive back an encryption of
  $\mu=f_{\matV_{k}, w_{k}}$ for $k=\beta$ as
  \[\vecc'^t = [\vecb_0^t \mid b_2^{(1)} + \mu],\]
  where $\vecb_0^t = \vecx_0^t\matA_i^{*}+(\vecx_1^{(1)})^t$,
  $b_2^{(1)}=\vecx_0^t(\vecy_i^{*})^{(1)}+x_2$ for $\vecx_0 \gets D_{\Z,
    r}^{n}$, $\vecx_1^{(1)} \gets D_{\Z, r}^{md}$ and $x_2 \gets D_{\Z, r}$,
  and $\mu = p \cdot f_{\matV_{\beta}, w_{\beta}} = p \cdot 
  (\sum_{j \in [d]}\inner{\vecv_{j, \beta},
    \vecz_{j,1}} + w_{\beta})$, as in the actual scheme from
  Section ~\ref{sec:KDMCPAscheme}.

  We then compute $\vecb_1^t = -\vecb_0^t\matR+(\vecx_1^{(2)})^t$ and
  $b_2^{(2)} = \vecb_0^t\matR\vecz_{i,1}^{(2)}$, where as in the actual
  scheme, $\vecx_1^{(1)} \gets D_{\Z,\gamma}^{w}$. Letting $b_2 =
  b_2^{(1)}+b_2^{(2)}$, we finally output as our response
  \[\vecc^t = [\vecb_0^t \mid \vecb_1^t \mid b_2+\mu].\]

  To complete the proof, we need to show that these ciphertexts are
  statistically indistinguishable from the ciphertexts in Game 1. The
  left-most part ($\vecb_0^t$) is generated by the KDM-CPA scheme with
  exactly the same distribution we used to generate $\vecb_0^t =
  \vecx_0^t\matA_i^{*}+(\vecx_1^{(1)})^t$ in Game 1.

  In Game 1, the center part of the ciphertexts was generated as
  $-\vecx_0^t\matA_i^{*}\matR+(\vecx_1^{(2)})^t$, while in this game, the center
  part is generated as
  $\vecb_1^t=-\vecb_0^t\matR+(\vecx_1^{(2)})^t=-\vecx_0^t\matA_i^{*}\matR+(\vecx_1^{(2)})^t+(\vecx_1^{(1)})^t\matR$. Lemma~\ref{Gaussianlemmas}
  gives that $\length{(\vecx_1^{(1)})^t\matR} \leq \poly(n)$ (except with
  negligible probability), so that by Lemma~\ref{sumplusgaussian}, the
  statistical distance between $(\vecx_1^{(2)})^t$ and
  $(\vecx_1^{(2)})^t+(\vecx_1^{(1)})^t\matR$ is at most $\poly(n)/\gamma =
  \negl(n)$. Since the rest of the center part is distributed
  identically in Games 1 and 2, we have that the center part in Game 1
  is statistically indistinguishable from the center part in Game 2.

  Finally, in Game 1, the right part of the ciphertexts was generated
  as 
  \[\vecx_0^t\vecy_i^*+x_2+p\cdot (\sum_{j \in [d]}\inner{\vecv_{j,\beta},
    \vecz_{j,1}} + w_{\beta}).\] In this game, the right part is
  generated as
 \begin{align*}
  b_2+\mu&=\vecx_0^t(\vecy_i^*)^{(1)}+
  \vecx_0^t\matA_i^*\matR\vecz_{i,1}^{(2)}+(\vecx_1^{(1)})^t\matR\vecz_{i,1}^{(2)}+x_2+\mu\\
  &=\vecx_0^t\vecy_i^*+(\vecx_1^{(1)})^t\matR\vecz_{i,1}^{(2)}+x_2+p\cdot
  (\sum_{j \in [d]}\inner{\vecv_{j, \beta}, \vecz_{j,1}} + w_{\beta}).\\  
  \end{align*} Now, we
  have by Lemma~\ref{Gaussianlemmas} that
  $|(\vecx_1^{(1)})^t\matR\vecz_{i,1}^{(2)}| \leq \poly(n)$ except with negligible
  probability. Lemma~\ref{sumplusgaussian} then gives that the
  statistical distance between $x_2$ and
  $x_2+(\vecx_1^{(1)})^t\matR\vecz_{i,1}^{(2)}$ is at most $\poly(n)/\gamma =
  \negl(n)$, and since the rest of the right part is distributed
  identically in Games 1 and 2, we have that the right part in Game 1
  is statistically indistinguishable from the right part in Game 2.

  Therefore, the ciphertexts in Game 1 are statistically
  indistinguishable from the ciphertexts in Game 2, and so the two
  games as a whole are statistically indistinguishable. The theorem
  follows.
\end{proof}


%%% Local Variables: 
%%% mode: latex
%%% TeX-master: "kdm-ibe"
%%% End: 
