\section{KDM-CPA Secure Public-Key Scheme}
\label{sec:KDMCPAscheme}

Here we present a ``dual''-style \lwe cryptosystem that is KDM-CPA
secure for affine functions of the secret keys.  In fact, by setting
the parameters appropriately, the construction and security proof also
encompass (a slight variant of) the cryptosystem
from~\cite{DBLP:conf/ctrsa/LindnerP11}, which has somewhat smaller
keys and ciphertexts than ``primal'' or ``dual'' systems.  In
Section~\ref{sec:Construction} we build a KDM-CPA secure IBE around
this system.

\subsection{Construction}

The cryptosystem involves a few parameters: a modulus $q=p^2$ for a
prime $p$ where the message space is~$\Zp$; integer dimensions $n, m$
relating to the underlying \lwe problems; and a Gaussian parameter $r$
for key generation and encryption.  To make embedding this scheme into
our IBE more natural, $\pkcgen$ includes an additional parameter $d$,
which will be used to specify the size of identity cliques in the IBE
scheme, and outputs public keys $\matA$ that are $md$ columns wide.  In
the public-key scheme alone, the value $d$ is unrelated to the number
of public keys that the adversary can obtain in an attack (which 
will be denoted as $\ell$ below and is
unbounded), and we would just fix $d=1$.

\begin{itemize}[itemsep=0pt]
\item $\pkcgen(1^n, d)$: choose $\matA \in \Zq^{n \times md}$,
  $\vecz_{0} \gets D_{\Z,r}^{n}$, $\vecz_{1} \gets D_{\Z,
    r}^{md}$, and let $\vecy = \vecz_{0}-\matA\vecz_{1} = [\matI_{n}
  \mid -\matA] \vecz \in \Zq^{n}$ where $\vecz=(\vecz_{0}, \vecz_{1})
  \in \Z^{n+md}$.  The public key is $(\matA, \vecy)$ and the secret
  key is $\vecz_{1}$.

  (Notice that, unlike the dual-style encryption
  of~\cite{DBLP:conf/stoc/GentryPV08}, but like the scheme
  of~\cite{DBLP:conf/ctrsa/LindnerP11}, the public key component
  $\vecy$ is a \emph{perturbed} value of $-\matA \vecz_{1}$.  This
  will be important in the proof of KDM security.)
 
\item $\pkcenc(\matA, \vecy, \mu)$: to encrypt a message $\mu \in
  \Zp$, choose $\vecx_{0} \gets D_{\Z,r}^{n}$, $\vecx_{1} \gets
  D_{\Z, r}^{md}$ and $x' \gets D_{\Z, r}$.  Output the
  ciphertext $\vecc^t =\vecx_{0}^t[\matA \mid \vecy] + [\vecx_{1}^t
  \mid x'] + [\veczero \mid p\cdot\mu] \in \Zq^{1 \times (md+1)}$.

\item $\pkcdec(\vecz_{1}, \vecc)$: Compute $\mu' = \vecc^t \left[
    \begin{smallmatrix}
      \vecz_{1} \\ 1
    \end{smallmatrix} \right] \in \Zq$.  Output the $\mu \in \set{0,
    \ldots, p-1} = \Zp$ such that $\mu'$ is closest to $(p \mu)$ mod
  $q$.
\end{itemize}

\subsection{Parameters and Correctness}

For the public-key system alone, it suffices to take $m \geq n$ by our
use of the extended-\lwe assumption and its proof of hardness as in
Section~\ref{sec:extended-lwe}.  When embedding the system into an IBE
scheme, however, we will use $m=\Theta(n \log q)$ because we need the
public parameters to be statistically close to uniform over the choice
of the master secret key.  The error parameter $r$ must be small
enough (relative to $q/p$) so that decryption is correct with
overwhelming probability, but large enough to satisfy the reductions
to \lwe from worst-case lattice
problems~\cite{DBLP:journals/jacm/Regev09,DBLP:conf/stoc/Peikert09};
for the latter purpose, $r \geq 2\sqrt{n}$ suffices.  (Note that even
if part of the security proof relies on \lwe in dimension $> n$, this
problem is no easier than \lwe in dimension $n$, and so we can still
securely use $r=2\sqrt{n}$ with the larger dimension.)

Here we give some example bounds.  Let $r=2\sqrt{n}$, let \[ p = r^{2}
\sqrt{n + md} \cdot \wsln = n\sqrt{n + md} \cdot \wsln, \] and let
$q=p^2$.  Then decryption is correct except with probability
$\negl(n)$: let $(\matA, \vecy, \vecz) \gets \pkcgen(1^n, d)$.  For a
ciphertext $\vecc \gets \pkcenc(\matA, \vecy, \mu)$, we have
\[ \vecc^t \left[
  \begin{smallmatrix}
    \vecz_{1} \\ 1
  \end{smallmatrix} \right] = \vecx_{0}^t\matA\vecz_{1} +
\inner{\vecx_{1},\vecz_{1}} + \inner{\vecx_{0},\vecy} + x' + p\cdot\mu
= \inner{\vecx_{0}, \vecz_{0}} + \inner{\vecx_{1},\vecz_{1}} + x' +
p\cdot\mu \bmod q,\] so decryption is correct whenever
$\abs{\inner{\vecx_{0},\vecz_{0}}+\inner{\vecx_{1},\vecz_{1}} + x'} <
p/2$.  By known tail bounds on discrete Gaussians, this bound holds
except with probability $\negl(n)$ (over the choice of all the random
variables), as required.

\subsection{Proof of Security}

\begin{theorem}
  \label{thm:KDMCPATheorem}
  The above cryptosystem is KDM-CPA secure with respect to the set of
  affine functions over $\Zp$, under the extended-\lwe assumption for
  parameters described above.
\end{theorem}


\begin{proof}
  We proceed by a series of indistinguishable games. We begin with the
  real KDM-CPA attack game as defined in Section~\ref{subsec:KDM},
  where $\beta \in \bit$ is arbitrary. We then eventually transition
  to a game which proceeds in a manner independent of the value of
  $\beta$, thus proving computational indistinguishability between the
  attack games for $\beta=0$ and $\beta=1$.  Since we are describing
  the games for an arbitrary value of $\beta$, we let $f=f_{\beta}$
  denote the function of the secret key that is encrypted in each
  ciphertext query.

  We will be denoting affine functions over $\Z_p$ of the $\ell$
  users' secret keys $\vecz_{1,1}, \ldots, \vecz_{\ell,1}$ as
  \[f_{\matV, w}(\matZ) := \sum_{j \in [\ell]} \inner{\vecv_j, \vecz_{j,1}}+w
  \bmod p, \text{ where } \matV = \begin{bmatrix}\vecv_1 \ldots
    \vecv_{\ell}\end{bmatrix}, \matZ = \begin{bmatrix}\vecz_{1,1}, 
    \ldots, \vecz_{\ell,1}\end{bmatrix} \in \Zp^{md \times \ell}\] throughout
  the proof.

  \paragraph{Game 0.} \label{KDMCPAAttackGame} This is the actual
  attack game. We do everything normally, generating public keys
  $(\matA_{i}, \vecy_{i})$ with secret keys $\vecz_{i,1}$ for any
  number of users, and encrypt affine functions $f_{\matV,w}$ of the
  users' secret keys under the public key for user $i$ as
  \[\vecc^t =\vecx_{0}^t[\matA_i \mid \vecy_i] + [\vecx_1^t \mid x']
  + [\veczero \mid p\cdot f_{\matV, w}(\matZ)].\]

  \paragraph{Game 1.} \label{KDMCPAasLWE} In this game we rewrite the
  way we respond to each key-dependent message query $f_{\matV,w}(\matZ)$, so
  that the response has exactly the same distribution, but it is
  constructed to syntactically match the \textsf{ExptLWE} experiment
  in the (amortized, normal form) extended-$\lwe$ problem.  Looking
  ahead, in Game 2 we will switch the form of the ciphertexts to match
  the \textsf{ExptUnif} experiment.

  For each user $i$, we proceed as in \textsf{ExptLWE} with parameters
  error distribution $\chi:=D_{\Z,r}$, extra noise $\beta:=0$, and the
  same $r$ as in the cryptosystem.  That is, we choose $\matA_{i}
  \gets \Zq^{n \times md}$ and $\vecz_{i} \gets D_{\Z, r}^{n+md}$,
  parsing it as $\vecz_{i}=(\vecz_{i,0},\vecz_{i,1}) \in \Z^{n} \times
  \Z^{md}$.  We let $\vecy_{i} = [\matI_{n} \mid -\matA_{i}] \vecz_{i}
  = \vecz_{i,0}-\matA_{i} \vecz_{i,1}$, giving user $i$'s public key
  $(\matA_{i}, \vecy_{i})$ to the adversary.  (As expected,
  $\vecz_{i,1}$ is the secret key for user~$i$.)

  For handling later ciphertext queries, we also (lazily) generate an
  unbounded number of \lwe vectors $\vecb$, with hints $b'$, of the
  form \[ \vecb^{t} := \vecx_{0}^{t} \matA_i + \vecx_{1}^{t} \in
  \Z_q^{1 \times md}, \quad b' := \inner{\vecx, \vecz_{i}} \in \Z, \]
  where $\vecx = (\vecx_{0},\vecx_{1}) \gets D_{\Z,r}^{n+md}$ is
  freshly chosen for each $\vecb$, and are amortized over the same
  fixed $\matA_{i}$, $\vecz_{i}$ chosen above.
  
  To answer a query to encrypt $f_{\matV,w}$ under user $i$'s public
  key, we generate a fresh $\vecb$ with hint $b'$ as above, choose $x'
  \gets D_{\Z, r}$ as usual, and return to the adversary the
  ciphertext
  \[\vecc^t = [\vecb^t \mid -\inner{\vecb, \vecz_{i,1}} + b' + x' +
  p\cdot f_{\matV, w}(\matZ)]. \]

  We now show that the responses to the KDM queries in this game are
  distributed exactly the same as in the previous game.  To see this,
  simply observe that each ciphertext
  \begin{align*}
    \vecc^{t} &= [\vecb^t \mid -\inner{\vecb, \vecz_{i,1}} +
    b' + x' + p\cdot f_{\matV, w}(\matZ)] \\
    &= [\vecx_{0}^{t} \matA_{i} + \vecx_{1}^{t} \mid -(\vecx_{0}^{t}
    \matA_{i} + \vecx_{1}^{t}) \vecz_{i,1} + \inner{\vecx, \vecz_{i}}
    + x' + p\cdot f_{\matV, w}(\matZ)] \\
    &= [\vecx_{0}^{t} \matA_{i} + \vecx_{1}^{t} \mid \inner{\vecx_{0},
      \vecy_{i}} + x' + p\cdot f_{\matV, w}(\matZ)],
  \end{align*}
  which has the same distribution as in Game 0.

  \paragraph{Game 2.} This game is exactly like Game 1, except that
  instead of using \textsf{ExptLWE}, we instead use \textsf{ExptUnif},
  i.e., we choose the vectors $\vecb$ uniformly at random (but still
  choose the vectors $\vecx$ for generating the hints
  $b'=\inner{\vecx,\vecz_{i}}$).  Responses to key-dependent message
  queries are still generated as
  \[\vecc^t = [\vecb^t \mid -\inner{\vecb,
    \vecz_{i,1}} + x' + b' + p\cdot f_{\matV, w}(\matZ)].\] Under the
  extended-\lwe assumption, and hence under the standard $\lwe$
  assumption with appropriate parameters (by
  Theorem~\ref{thm:ext-lwe-hard}), this game is computationally
  indistinguishable from Game 1.

  \paragraph{Game 3.} In this game, we use the \lwe distribution
  transformation from~\cite{DBLP:conf/crypto/ApplebaumCPS09} (our
  Lemma~\ref{LWEtransformation}) to answer key-dependent queries
  without needing to use the secret keys $\vecz_{i,1}$ explicitly.  We
  begin with oracle access to $A_{\vect, \chi}$ (where $\vect \gets
  \Zq^{md}$, $\chi=D_{\Z, r}$).  We apply the transformation for each
  $i$, yielding oracle access to distributions $A_{\vecz_{i,1},\chi}$,
  where each $\vecz_{i,1}$ is distributed according to $\chi^{md}$.
  For each user $i$, we sample $A_{\vecz_{i,1},\chi}$ a total of $n$
  times to get $(-\matA_i \in \Zq^{n \times md}, \vecy_i =
  \vecz_{i,0}-\matA_i\vecz_{i,1})$, and set the public key for user
  $i$ to be $(\matA_i, \vecy_i)$.  (We write $-\matA_i$ instead of
  $\matA_i$ so that $\vecy_i$ has the same form with respect to
  $\matA_{i}$ as it does in the actual encryption scheme.)

  As in~\cite{DBLP:conf/crypto/ApplebaumCPS09}, we use the auxiliary
  information output by the \lwe transformation to construct for all
  users $i,j$ a linear relation between the (unknown) secret keys
  $\vecz_{i,1}$, $\vecz_{j,1}$, i.e., an invertible matrix
  $\matT_{i,j} \in \Zq^{md \times md}$ and vector $\vecw_{i,j} \in
  \Zq^{md}$ such that
  \begin{equation}
    \label{transformeq}
    \vecz_{j,1} = \matT_{i,j}^{t} \cdot \vecz_{i,1}+\vecw_{i,j} \bmod q.
  \end{equation}

  \jnote{Using the tilde for the re-written queries rather than a '
    symbol because using the prime symbol makes the transposes look
    ugly} To simplify notation below, we will use this transformation
  to re-write an arbitary affine function query $f_{\matV,w}(\matZ)$
  as a function of any 
  $\vecz_{i,1}$ alone. Letting $\tilde{\vecv}_i=\sum_{j \in
    [\ell]}(\matT_{i,j}\vecv_j) \bmod p$, $\tilde{w}_i =\sum_{j \in
    [\ell]}\inner{\vecv_j, \vecw_{i,j}}+w \bmod p$, we have that
\[f_{\matV, w}(\matZ)= \sum_{j
    \in [\ell]}\inner{\vecv_j,\vecz_{j,1}}+w = 
\sum_{j \in [\ell]}(\inner{\matT_{i,j}\vecv_j, \vecz_{i,1}} + \inner
{\vecv_{j}, \vecw_{i,j}}+w)=\inner{\tilde{\vecv}_{i}, \vecz_{i,1}}+\tilde{w}_{i}.\] 

To respond to a query for an encryption of $f_{\matV,w}(\matZ)$ under
the public key for user~$i$, we draw a fresh $(-\vecb_0, b_1) \gets
A_{\vecz_{i,1}, \chi}$.  Next, we choose a fresh
$\vecx=(\vecx_{0},\vecx_{1}) \gets D_{\Z, r}^{n+md}$.  We output the
ciphertext
  \[\vecc^t = [\vecb_0^t + (\vecx_{0}^{t}\matA_{i} + \vecx_{1}^{t})
  + p \cdot \tilde{\vecv}_{i}^{t} \mid b_1 +
  \inner{\vecx_{0}, \vecy_i} + p \cdot \tilde{w}_{i}].\]

  We claim that these ciphertexts are distributed identically to those
  in Game 2.  To show this, we let $\vecb^t = \vecb_0^t +
  (\vecx_{0}^{t}\matA_{i}+\vecx_1^{t}) + p \cdot
  \tilde{\vecv}_{i}^t$.  Since $\vecb_0$ is uniform (by the
  definition of $A_{\vecz_{i,1}, \chi}$), $\vecb$ in this game is
  uniform over the choice of $\vecb_0$ alone, just as in Game 2.
  % Now, taking the
  % inner product of $\vecb$ and $\vecz_i$, we have by
  % Equation~\eqref{transformeq} that
  % \begin{equation}
  %   \label{transformitoj}
  %   -\inner{\vecb,\vecz_i} = -\inner{\vecb_0,\vecz_i}-p
  %   \cdot\inner{\matT_{i,j}\vecv,\vecz_i} =
  %   -\inner{\vecb_0,\vecz_i}-p \cdot
  %   \inner{\vecv,\vecz_j}+p\cdot\inner{\vecv,\vecw_{i,j}}.
  % \end{equation}
  Next, since $b_{1}=-\inner{\vecb_{0}, \vecz_{i,1}}+x'$ for $x' \gets
  D_{\Z,r}$, we can rewrite the ciphertext constructed in this game as
  \begin{align*}
    \vecc^{t} &= [\vecb^{t} \mid \inner{-\vecb + (\matA_i^{t}
      \vecx_{0} + \vecx_{1}) + p \cdot \tilde{\vecv}_{i}, \vecz_{i,1}} +
    x' + \inner{\vecx_{0}, \vecy_i} + p
    \cdot \tilde{w}_{i}]\\
    &= [\vecb^{t} \mid -\inner{\vecb, \vecz_{i,1}} + x' +
    \inner{\vecx_{0}, \vecy_i+\matA_i\vecz_{i,1}} + \inner{\vecx_{1},
      \vecz_{i,1}} + p \cdot (\inner{\tilde{\vecv}_{i}, \vecz_{i,1}}+\tilde{w}_{i})]\\
    &= [\vecb^{t} \mid -\inner{\vecb, \vecz_{i,1}} + x' +
    \inner{\vecx, \vecz_i} + p \cdot f_{\matV, w}(\matZ)],
  \end{align*}
  which is distributed exactly as in Game 2.

  \paragraph{Game 4.} This game is exactly like Game 3, except instead
  of beginning with oracle access to $A_{\vect, \chi}$ for
  $\chi=D_{\Z,r}$, we begin with oracle access to $U(\Zq^{md} \times
  \Zq)$.  Computational indistinguishability from Game 3 follows
  directly from the hardness of decision-\lwe.

  In this game, the public keys $(\matA_i, \vecy_i)$ are now all
  uniformly random and independent.  Moreover, since every sample
  $(\vecb_0, b_1)$ drawn in the game is also truly uniform and
  independent of everything else in the ciphertexts, all of the
  responses to key-dependent message queries are just uniformly random
  and independent vectors.  This makes the game independent of
  $\beta$, and concludes the proof.
\end{proof}


%%% Local Variables: 
%%% mode: latex
%%% TeX-master: "kdm-ibe"
%%% End: 
