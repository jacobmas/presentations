
\section{All-But-$d$ Trapdoor Functions}
\label{sec:all-but-d}

Here we develop a technique for constructing ``all-but-$d$''
(tag-based) trapdoor functions, which, informally, are trapdoor
functions for which the trapdoor enables efficient inversion for all
but (up to) $d$ tags, which are specified at the time of key
generation.  This is the main tool we use for embedding our
KDM-CPA-secure public-key cryptosystem into an identity-based
encryption scheme.

Our construction is a generalization (to higher-degree polynomials) of
the main technique from~\cite{DBLP:conf/eurocrypt/AgrawalBB10}.  For
simplicity and somewhat better efficiency, we follow the construction
of~\cite{DBLP:conf/eurocrypt/MicciancioP12}, specifically the use of a
fixed, public ``gadget'' matrix $\matG$ as described in
Section~\ref{subsec:trapdoors}.

\subsection{Algebraic Background}

Let $n \geq 1$, $q \geq 2$, and $d = \poly(n)$
% ($d$ does not actually grow with $n$, but nevertheless must be bounded
% by a polynomial in $n$ to allow the construction to proceed in
% polynomial time)
be integers.  Let~$\calR$ denote any commutative ring (with
efficiently computable operations, including inversion of
multiplicative units) such that the additive group $\G=\Zq^{n}$ is an
$\calR$-module, and such that there are at least $d+1$ known elements
$U = \set{u_{0}=0, u_{1}, u_{2}, \ldots} \subseteq \calR$ where
$u_{i}-u_{j}$ is invertible in $\calR$ (i.e., a unit) for every $i
\neq j$.  \iflncs \else (These are the only abstract properties of the
ring we will need for our constructions.  In the next paragraph we
recall how to construct such a ring for any $n$ and $q$, where $U$ has
size at least $2^{n}$.)  \fi In particular, we have an (efficiently
computable) scalar multiplication operation $\calR \times \G \to \G$.
Note that multiplication by $u \in \calR$ is an invertible linear
transformation on $\G$ exactly when $u$ is invertible (i.e., a unit).
We extend scalar multiplication in the natural way to vectors and
matrices, i.e., $R^{a \times b} \times \G^{b \times c} \to \G^{a
  \times c}$.  To avoid confusion with vectors and matrices over
$\Zq$, we use $\vec{u}$ notation for vectors over $\calR$, and $V$
notation for matrices over $\calR$.

To construct a suitable ring, we use ideas from the literature on
secret sharing over groups and modules,
e.g.,~\cite{DBLP:journals/siamdm/DesmedtF94,Fehr98}.  We use an
extension ring $\calR=\Zq[x]/(F(x))$ for any monic, degree-$n$,
irreducible $F(x) = F_{0}+F_{1}x + \cdots + F_{n-1}x^{n-1} + x^{n} \in
\Zq[x]$.  Scalar multiplication $\calR \times \G \to \G$ is defined by
identifying each $\veca = (a_{0}, \ldots, a_{n-1})^{t} \in \G$ with
the polynomial $a(x) = a_{0} + a_{1}x + \cdots + a_{n-1}x^{n-1} \in
\calR$, multiplying in $\calR$, then mapping back to $\G$.  In other
words, scalar multiplication is defined by the linear transformation
$x \cdot (a_{0}, \ldots, a_{n-1})^{t} = (0, a_{0}, \ldots,
a_{n-2})^{t} - a_{n-1}(F_{0}, F_{1}, \ldots, F_{n-1})^{t}$.  It is
easy to check that with this scalar product, $\G$ is an
$\calR$-module.  In addition, by the Chinese remainder theorem, $r \in
\calR$ is a unit if and only if it is nonzero (as a polynomial
residue) modulo every prime integer divisor $p$ of $q$.  (This is
because $\Z_{p}[x]/(F(x))$ is a field by construction.)  Letting $p$
be the smallest such divisor of $q$, we can define the universe $U =
\set{u_{0}=0, u_{1}, u_{2}, \ldots} \subseteq \calR$ to consist of all
the polynomial residues having coefficients in $\set{0, \ldots, p-1}$.
Then $\abs{U} = p^{n} \geq 2^{n}$ and $u_{i}-u_{j}$ is a unit for all
$i \neq j$, as desired.

\subsection{Basic Construction}
\label{sec:basic-construction}

As in~\cite{DBLP:conf/eurocrypt/MicciancioP12}, we fix a universal public
``gadget'' matrix $\matG \in \Zq^{n \times w}$ for which there is an
efficient Gaussian preimage sampling algorithm for parameter $s \geq
\wsln$, i.e., an algorithm that given any $\vecu \in \Zq^{n}$ outputs
a sample from $D_{\lamperp_{\vecu}(\matG), s}$.  E.g., we can let
$\matG = \matI_{n} \otimes (1, 2, 4, \ldots, 2^{k-1}) \in \Zq^{n
  \times nk}$ for $k = \ceil{\lg q}$.

As input, the trapdoor generator takes:
\begin{itemize}[itemsep=0pt]
\item an integer $d \geq 1$ and a monic degree-$d$ polynomial $f(z) =
  c_{0} + c_{1} z + \cdots + z^{d} \in \calR[z]$,
\item a (usually uniformly random) matrix $\bar{\matA} \in \Zq^{(nd)
    \times \bar{m}}$ for some $\bar{m} \geq 1$, which is made up of
  stacked submatrices $\bar{\matA}_{i} \in \Zq^{n \times \bar{m}}$ for
  $i=0,\ldots,d-1$.
\item  a ``short'' secret
$\matR \in \Z^{\bar{m} \times w}$ chosen at random from an appropriate
distribution (typically, a discrete Gaussian) to serve as a trapdoor.
\end{itemize}
As output it produces a matrix $\matA \in \Zq^{(nd) \times
  (\bar{m}+w)}$ (which is statistically close to uniform, when the
parameters and input $\bar{\matA}$ are appropriately chosen).  In addition, for
each tag $u \in U$ there is an efficiently computable (from $\matA$)
matrix $\matA_{u} \in \Zq^{n \times (\bar{m}+w)}$ for which $\matR$
\emph{may} be a trapdoor, depending on the value of $f(u) \in \calR$.

We write the coefficients of $f(z)$ as a column vector $\vec{c} =
(c_{0}, c_{1}, \ldots, c_{d-1})^{t} \in \calR^{d}$, and define
\[ \matA'_{f} :=
\begin{bmatrix}
  \bar{\matA} & \vec{c} \otimes \matG
\end{bmatrix}
=
\begin{bmatrix}
  \bar{\matA}_{0} & c_{0} \cdot \matG \\
  %\bar{\matA}_{1} & c_{1} \cdot \matG \\
  \vdots & \vdots \\
  \bar{\matA}_{d-1} & c_{d-1} \cdot \matG
\end{bmatrix} \in \Zq^{(nd) \times (\bar{m}+w)}.
\]
To hide the polynomial $f$, we output the public key
\[ \matA := \matA'_{f} \cdot
\begin{bmatrix}
  \matI & -\matR \\ & \matI
\end{bmatrix}
=
\begin{bmatrix}
  \bar{\matA} & (\vec{c} \otimes \matG) - \bar{\matA} \matR
\end{bmatrix}.
\]
Note that as long as the distribution of $[\bar{\matA} \mid
-\bar{\matA} \matR]$ is statistically close to uniform, then so is
$\matA$ for any $f$.

The tag space for the trapdoor function is the set $U \subset \calR$.
For any tag $u \in U$, define the row vector $\vec{u}^{t} := (u^{0},
u^{1}, \cdots, u^{d-1}) \in \calR^{d}$ (where $0^{0}=1$) and the
derived matrix for tag $u$ to be
\[ \matA_{u} := \vec{u}^{t} \cdot \matA +
\begin{bmatrix}
  \matzero & u^{d} \cdot \matG
\end{bmatrix}
=
\begin{bmatrix}
  \vec{u}^{t} \cdot \bar{\matA} & f(u) \cdot \matG
\end{bmatrix} \cdot \bigT.
\]
By the condition in Lemma~\ref{lem:trapgen-sample}, $\matR$ is a
(strong) trapdoor for $\matA_{u}$ exactly when $f(u) \in \calR$ is a
unit, because $\matA_{u} \cdot \left[
  \begin{smallmatrix}
    \matR \\ \matI
  \end{smallmatrix} \right] = f(u) \cdot \matG$ and $f(u)$ represents
an invertible linear transformation when it is a unit. 

\subsection{Puncturing}
\label{sec:puncturing}

In our cryptosystems and security proofs we will need to generate
(using the above procedure) an all-but-$d$ trapdoor function that is
``punctured'' at up to $d$ tags.  More precisely, we are given as
input:
\begin{itemize}[itemsep=0pt]
\item a set of distinct tags $P = \set{u_{1}, \ldots, u_{\ell}}
  \subseteq U$ for some $\ell \leq d$,
\item uniformly random matrices $\matA^{*}_{i} \in \Zq^{n \times
    \bar{m}}$ for $i \in [\ell]$ (which often come from an $\sis$ or
  $\lwe$ challenge),
\item  a ``short'' secret
$\matR \in \Z^{\bar{m} \times w}$ chosen at random from an appropriate
distribution (typically, a discrete Gaussian) to serve as a trapdoor,
\item optionally, some uniformly random auxiliary matrices
  $\matY^{*}_{i} \in \Zq^{n \times k}$ for $i \in [\ell]$ and some $k
  \geq 0$.
\end{itemize}
As output we produce a public key $\matA \in \Zq^{(nd) \times
  \bar{m}}$ and auxiliary
matrix $\matY \in \Zq^{(nd) \times k}$ so that:
\begin{enumerate}[itemsep=0pt]
\item \label{item:punctured} Each $\matA_{u_{i}}$ matches the
  challenge matrix $\matA^{*}_{i}$, and $\matR$ is only a ``weak''
  trapdoor for~$\matA_{u_{i}}$.  More precisely, \[ \matA_{u_{i}} =
  \begin{bmatrix}
    \matA^{*}_{i} & \matzero
  \end{bmatrix}
  \cdot \bigT. \]
  
\item \label{item:trapdoor} $\matR$ is a (strong) trapdoor for
  $\matA_{u}$ for any \emph{nonzero} $u \in U \setminus P$, i.e.,
  $f(u)$ is a unit.

\item \label{item:outputs} The auxiliary matrix $\matY_{u_{i}} :=
  \vec{u_{i}}^{t} \cdot \matY$ equals the auxiliary input
  $\matY^{*}_{i}$ for each $u_{i} \in P$.
\end{enumerate}
We satisfy these criteria by invoking the above trapdoor generator
with the following inputs $f$ and $\bar{\matA}$:
\begin{enumerate}[itemsep=0pt]
\item We define the monic degree-$d$ polynomial \[ f(z) = z^{d-\ell}
  \cdot \prod_{i \in [\ell]} (z-u_{i}) \in \calR[z] \] and expand to
  compute its coefficients $c_{i} \in \calR$.  Note that $f(u_{i}) =
  0$ for every $u_{i} \in P$, and $f(u)$ is a unit for any nonzero $u
  \in U \setminus P$ because $0 \in U$ and $u_{i}-u_{j}$ is a unit for
  every distinct $u_{i},u_{j} \in U$.
\item We define $\bar{\matA}$ using interpolation: let $\matA^{*} \in
  \Zq^{(n \ell) \times \bar{m}}$ denote the stack of challenge
  matrices $\matA^{*}_{i}$, and let $V \in \calR^{\ell \times d}$ be
  the Vandermonde matrix whose rows are the vectors $\vec{u_{i}}^{t}$
  defined above.  We then let $\bar{\matA} \in \Zq^{(nd) \times
    \bar{m}}$ be a uniformly random solution to $V \cdot \bar{\matA} =
  \matA^{*}$.

  Such a solution exists, and is efficiently computable and uniformly
  random (over the uniformly random choice of $\matA^{*}$ and the
  random solution chosen).  To see this, extend $V$ to an invertible
  $d \times d$ Vandermonde matrix over $\calR$ having unit determinant
  $\prod_{i < j} (u_{j}-u_{i}) \in \calR^{*}$, by adding $d-\ell$
  additional rows $\vec{u_{j}}^{t}$ for arbitrary distinct $u_{j} \in
  U \setminus P$.  Likewise, extend $\matA^{*}$ to have dimension $(n
  d) \times \bar{m}$ by adding uniformly random rows.  Then for any
  fixed choice of the (extended) matrix $V$, the (extended) matrix
  $\matA^{*}$ and solution $\bar{\matA}$ are in bijective
  correspondence, and so the latter is uniformly random because the
  former is.
\item We also define the auxiliary matrix $\matY$ similarly using
  interpolation, as a uniformly random solution to $V \cdot \matY =
  \matY^{*}$.
\end{enumerate}

%%% Local Variables: 
%%% mode: latex
%%% TeX-master: "kdm-ibe"
%%% End: 
