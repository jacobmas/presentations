\documentclass[11pt,t,xcolor=pdftex,svgnames]{beamer}
\mode<presentation> {\usetheme{NIST}}

\usepackage{colortbl}
\usepackage{multirow}

\title{Ramstake}
\subtitle{Submission by Alan Szepieniec, KU Leuven}
\author{Jacob Alperin-Sheriff}
\institute{NIST}

\def\CC{{C\nolinebreak[4]\hspace{-.05em}\raisebox{.4ex}{\tiny\bf ++}}}


\usepackage{environ}% Required for \NewEnviron, i.e. to read the whole body of the environment
\makeatletter

\newcounter{acolumn}%  Number of current column
\newlength{\acolumnmaxheight}%   Maximum column height


% `column` replacement to measure height
\newenvironment{@acolumn}[1]{%
    \stepcounter{acolumn}%
    \begin{lrbox}{\@tempboxa}%
    \begin{minipage}{#1}%
}{%
    \end{minipage}
    \end{lrbox}
    \@tempdimc=\dimexpr\ht\@tempboxa+\dp\@tempboxa\relax
    % Save height of this column:
    \expandafter\xdef\csname acolumn@height@\roman{acolumn}\endcsname{\the\@tempdimc}%
    % Save maximum height
    \ifdim\@tempdimc>\acolumnmaxheight
        \global\acolumnmaxheight=\@tempdimc
    \fi
}

% `column` wrapper which sets the height beforehand
\newenvironment{@@acolumn}[1]{%
    \stepcounter{acolumn}%
    % The \autoheight macro contains a \vspace macro with the maximum height minus the natural column height
    \edef\autoheight{\noexpand\vspace*{\dimexpr\acolumnmaxheight-\csname acolumn@height@\roman{acolumn}\endcsname\relax}}%
    % Call original `column`:
    \orig@column{#1}%
}{%
    \endorig@column
}

% Save orignal `column` environment away
\let\orig@column\column
\let\endorig@column\endcolumn

% `columns` variant with automatic height adjustment
\NewEnviron{acolumns}[1][]{%
    % Init vars:
    \setcounter{acolumn}{0}%
    \setlength{\acolumnmaxheight}{0pt}%
    \def\autoheight{\vspace*{0pt}}%
    % Set `column` environment to special measuring environment
    \let\column\@acolumn
    \let\endcolumn\end@acolumn
    \BODY% measure heights
    % Reset counter for second processing round
    \setcounter{acolumn}{0}%
    % Set `column` environment to wrapper
    \let\column\@@acolumn
    \let\endcolumn\end@@acolumn
    % Finally process columns now for real
    \begin{columns}[#1]%
        \BODY
    \end{columns}%
}
\makeatother

\newcommand{\Red}[1]{{\color{Red}#1}}
\newcommand{\xof}{\ensuremath{\text{xof}}}

%% BOILERPLATE PRESENTATION STUFF

% an underlining package
\usepackage{ulem}
% use normal italics for emphasis
\normalem


% dingbats
\usepackage{pifont}

\renewcommand{\Check}{\ding{52}}
\newcommand{\Cross}{\ding{55}}
\newcommand{\Hand}{\ding{43}}
\newcommand{\GreenCheck}{{\textcolor{green}\Check}}
\newcommand{\RedCross}{{\textcolor{red}\Cross}}

% alias for citations
\newcommand{\citationsize}{\footnotesize}

\subject{Theoretical Computer Science}

% suppress ugly QEDs in proofs
\def\qedsymbol{}

% bring in the individual header files

\input{bbhead}
% "left-right" pairs of symbols

\usepackage{mathtools}

% inner product
\DeclarePairedDelimiter\inner{\langle}{\rangle}
% absolute value
\DeclarePairedDelimiter\abs{\lvert}{\rvert}
% a set
\DeclarePairedDelimiter\set{\{}{\}}
% parens
\DeclarePairedDelimiter\parens{(}{)}
% tuple, alias for parens
\DeclarePairedDelimiter\tuple{(}{)}
% square brackets
\DeclarePairedDelimiter\bracks{[}{]}
% rounding off
\DeclarePairedDelimiter\round{\lfloor}{\rceil}
% floor function
\DeclarePairedDelimiter\floor{\lfloor}{\rfloor}
% ceiling function
\DeclarePairedDelimiter\ceil{\lceil}{\rceil}
% length of some vector, element
\DeclarePairedDelimiter\length{\lVert}{\rVert}
% "lifting" of a residue class
\DeclarePairedDelimiter\lift{\llbracket}{\rrbracket}

\input{vecmathead}
\input{thmhead}
% GENERAL COMPUTING STUFF

\newcommand{\bit}{\ensuremath{\set{0,1}}}
\newcommand{\pmone}{\ensuremath{\set{-1,1}}}

% asymptotic stuff
\DeclareMathOperator{\poly}{poly}
\DeclareMathOperator{\polylog}{polylog}
\DeclareMathOperator{\negl}{negl}
\newcommand{\Otil}{\ensuremath{\tilde{O}}}

% probability/distribution stuff
\DeclareMathOperator*{\E}{\mathbb{E}}
\DeclareMathOperator*{\Var}{Var}

\DeclareMathOperator{\trace}{Tr}
\DeclareMathOperator{\lcm}{lcm}

% assorted
\DeclareMathOperator*{\wt}{wt}

% hash functions
\newcommand{\calH}{\ensuremath{\mathcal{H}}}
\newcommand{\calX}{\ensuremath{\mathcal{X}}}
\newcommand{\calY}{\ensuremath{\mathcal{Y}}}

\newcommand{\compind}{\ensuremath{\stackrel{c}{\approx}}}
\newcommand{\statind}{\ensuremath{\stackrel{s}{\approx}}}
\newcommand{\perfind}{\ensuremath{\equiv}}

% font for general-purpose algorithms
\newcommand{\algo}[1]{\ensuremath{\mathsf{#1}}\xspace}

% font for general-purpose computational problems
\iflncs
\renewcommand{\problem}[1]{\ensuremath{\mathsf{#1}}\xspace}
\else
\newcommand{\problem}[1]{\ensuremath{\mathsf{#1}}\xspace}
\fi

% font for complexity classes
\newcommand{\class}[1]{\ensuremath{\mathsf{#1}}\xspace}

% complexity classes and languages
\renewcommand{\P}{\class{P}}
\newcommand{\BPP}{\class{BPP}}
\newcommand{\NP}{\class{NP}}
\newcommand{\coNP}{\class{coNP}}
\newcommand{\AM}{\class{AM}}
\newcommand{\coAM}{\class{coAM}}
\newcommand{\IP}{\class{IP}}
\newcommand{\SZK}{\class{SZK}}
\newcommand{\NISZK}{\class{NISZK}}
\newcommand{\NICZK}{\class{NICZK}}
\newcommand{\PPP}{\class{PPP}}
\newcommand{\PPAD}{\class{PPAD}}

\newcommand{\yes}{\ensuremath{\text{YES}}}
\newcommand{\no}{\ensuremath{\text{NO}}}

\newcommand{\Piyes}{\ensuremath{\Pi^{\yes}}}
\newcommand{\Pino}{\ensuremath{\Pi^{\no}}}

%\DeclareMathOperator{\trace}{Tr}
%\DeclareMathOperator{\lcm}{lcm}
\DeclareMathOperator{\msb}{msb}
\DeclareMathOperator{\lsb}{lsb}
\DeclareMathOperator{\rad}{rad}

\renewcommand{\O}{\mathcal{O}}
\newcommand{\ord}{\ensuremath{\text{ord}}}

\newcommand{\calA}{\ensuremath{\mathcal{A}}}
\newcommand{\calB}{\ensuremath{\mathcal{B}}}
\newcommand{\calC}{\ensuremath{\mathcal{C}}}
\newcommand{\calD}{\ensuremath{\mathcal{D}}}
\newcommand{\calE}{\ensuremath{\mathcal{E}}}
\newcommand{\calF}{\ensuremath{\mathcal{F}}}
\newcommand{\calG}{\ensuremath{\mathcal{G}}}
\newcommand{\calI}{\ensuremath{\mathcal{I}}}
\newcommand{\calJ}{\ensuremath{\mathcal{J}}}
\newcommand{\calK}{\ensuremath{\mathcal{K}}}
\newcommand{\calL}{\ensuremath{\mathcal{L}}}
\newcommand{\calM}{\ensuremath{\mathcal{M}}}
\newcommand{\calN}{\ensuremath{\mathcal{N}}}
\newcommand{\calO}{\ensuremath{\mathcal{O}}}
\newcommand{\calP}{\ensuremath{\mathcal{P}}}
\newcommand{\calQ}{\ensuremath{\mathcal{Q}}}
\newcommand{\calR}{\ensuremath{\mathcal{R}}}
\newcommand{\calS}{\ensuremath{\mathcal{S}}}
\newcommand{\calT}{\ensuremath{\mathcal{T}}}
\newcommand{\calU}{\ensuremath{\mathcal{U}}}
\newcommand{\calV}{\ensuremath{\mathcal{V}}}
\newcommand{\calW}{\ensuremath{\mathcal{W}}}
\newcommand{\calZ}{\ensuremath{\mathcal{Z}}}

\newcommand{\frakd}{\mathfrak{d}}
\newcommand{\frakp}{\mathfrak{p}}


%\algnotext{EndFor}
%\algnotext{EndIf}
%\algnotext{EndWhile}

% fix weird spacing of \pmod, and introduce \pmod* command; see
% http://tex.stackexchange.com/questions/39221/removing-extra-space-with-pmod-command
\makeatletter
\renewcommand{\pod}[1]{\mathchoice
  {\allowbreak \if@display \mkern 18mu\else \mkern 8mu\fi (#1)}
  {\allowbreak \if@display \mkern 18mu\else \mkern 8mu\fi (#1)}
  {\mkern4mu(#1)}
  {\mkern4mu(#1)}
}
\makeatletter
\let\@@pmod\pmod
\DeclareRobustCommand{\pmod}{\@ifstar\@pmods\@@pmod}
\def\@pmods#1{\mkern4mu({\operator@font mod}\mkern 6mu#1)}
\makeatother
% basic notation

\newcommand{\lamperp}{\Lambda^{\perp}}
\newcommand{\lam}{\Lambda}

% lattice
\newcommand{\lat}{\mathcal{L}}
% fundamental region
\newcommand{\piped}{\mathcal{P}}
% smoothing parameter
\newcommand{\smooth}{\eta}
% smoothing w/ epsilon
\newcommand{\smootheps}{\smooth_{\epsilon}}
% ball
\newcommand{\ball}{\mathcal{B}}
% cube
\newcommand{\cube}{\mathcal{C}}
% covering radius symbol
\newcommand{\cover}{\ensuremath{\mu}}
% Gram-Schmidt
\newcommand{\gs}[1]{\ensuremath{\widetilde{#1}}}
% GS minimum
\newcommand{\gsmin}{\ensuremath{\tilde{bl}}}
% volume operation
\DeclareMathOperator{\vol}{vol}
% Hermite normal form
\DeclareMathOperator{\hnf}{HNF}
% rank
\DeclareMathOperator{\rank}{rank}
% distance operator
\DeclareMathOperator{\dist}{dist}
% span operator
\DeclareMathOperator{\spn}{span}
% error function
\DeclareMathOperator{\erf}{erf}

% quantities that show up regularly
\newcommand{\wsln}{\ensuremath{\omega(\sqrt{\log n})}}
\newcommand{\wslm}{\ensuremath{\omega(\sqrt{\log m})}}

% support algorithms
\newcommand{\sampleZ}{\algo{Sample}\Z}
\newcommand{\sampleD}{\algo{SampleD}}
\newcommand{\tobasis}{\algo{ToBasis}}
\newcommand{\invert}{\algo{Invert}}
\newcommand{\genbasis}{\algo{GenBasis}}
\newcommand{\extbasis}{\algo{ExtBasis}}
\newcommand{\extlattice}{\algo{ExtLattice}}
\newcommand{\randbasis}{\algo{RandBasis}}
\newcommand{\gentrap}{\algo{GenTrap}}
\newcommand{\deltrap}{\algo{DelTrap}}
\newcommand{\noisy}{\algo{Noisy}}

% problems related to lattices
\newcommand{\svp}{\problem{SVP}}
\newcommand{\gapsvp}{\problem{GapSVP}}
\newcommand{\cogapsvp}{\problem{coGapSVP}}
\newcommand{\usvp}{\problem{uSVP}}
\newcommand{\sivp}{\problem{SIVP}}
\newcommand{\gapsivp}{\problem{GapSIVP}}
\newcommand{\cvp}{\problem{CVP}}
\newcommand{\gapcvp}{\problem{GapCVP}}
\newcommand{\cvpp}{\problem{CVPP}}
\newcommand{\gapcvpp}{\problem{GapCVPP}}
\newcommand{\bdd}{\problem{BDD}}
\newcommand{\gdd}{\problem{GDD}}
\newcommand{\add}{\problem{ADD}}
\newcommand{\incgdd}{\problem{IncGDD}}
\newcommand{\incivd}{\problem{IncIVD}}
\newcommand{\crp}{\problem{CRP}}
\newcommand{\gapcrp}{\problem{GapCRP}}
% problems on ideal lattices
\newcommand{\igvp}{\problem{IGVP}}
\newcommand{\incigvp}{\problem{IncIGVP}}
% avg-case stuff
\newcommand{\sis}{\problem{SIS}}
\newcommand{\isis}{\problem{ISIS}}
\newcommand{\ilwe}{\problem{ILWE}}
\newcommand{\lwe}{\problem{LWE}}
\newcommand{\rlwe}{\problem{RLWE}}
\newcommand{\lwr}{\problem{LWR}}
\newcommand{\rlwr}{\problem{RLWR}}
\newcommand{\dlwe}{\problem{DLWE}}
\newcommand{\lpn}{\problem{LPN}}
\newcommand{\Psibar}{\ensuremath{\bar{\Psi}}}

\newcommand{\extlwe}{\problem{Ext \text{\textendash} LWE}}
\newcommand{\extrlwe}{\problem{Ext \text{\textendash} RLWE}}

% Module-LWE and variants
\newcommand{\mlwe}{\problem{M \text{\textendash} LWE}}
\newcommand{\extmlwe}{\problem{Ext \text{\textendash}  M \text{\textendash} LWE}}
\newcommand{\mlwr}{\problem{M \text{\textendash}  LWR}}
\input{crypthead}

\newcommand{\etal}{\textit{et al.}}


\begin{document}

%title page
    \setbeamertemplate{headline}[title_page]
    \setbeamertemplate{footline}[title_page]
    \csname beamer@calculateheadfoot\endcsname %recalculate head and foot dimension
        \begin{frame}
            \titlepage
        \end{frame}
%head and foot for body text    
    \setbeamertemplate{headline}[body]
    \setbeamertemplate{footline}[body]

%%%%%%%%%%%%%%%%%%%%%%%%%%%%%%%%%%%%%%%%%%
% \begin{frame}{Outline}
%     \vskip 5mm
%     {\parbox{.95\textwidth}{\tableofcontents[hideallsubsections]}}
% \end{frame}

%\begin{frame}
%    \titlepage
%\end{frame}
%
%\begin{frame}
%    \frametitle{Overview}
%
%    %\tableofcontents
%    \begin{itemize}
%        \item Scope of the project
%        \item Bayesian language models
%        \item Results
%        \item Research plan
%        \item Side projects
%        \item Reflections
%        \item Formalities
%    \end{itemize}
%
%\end{frame}
\section{Basic Idea}

\begin{frame}    
    \frametitle{Another Variation on Learning With Errors}

    \onslide<1->
    \begin{block}{(Ring)-LWE reminder}
        \begin{description}
            \item[KeyGen$(1^{\lambda})$:]
\begin{itemize}
\item Sample uniform $a \gets R$, \alert{short} (Euclidean norm)
              $\Red{s},\Red{e}$. 
\item Output $(a,b=\Red{s}\cdot a +
              \Red{e})$
\end{itemize}
            \item[Enc$(\mu)$:] 
\begin{itemize}
  \item Sample \alert{short} (Euclidean norm)
              $\Red{r},\Red{e_1}, \Red{e_2}$.
            \item Output  
              $c=(a\cdot \Red{r}+\Red{e_1}, b\cdot
              \Red{r}+\Red{e_2}+(q/2)\mu)$
\end{itemize}
             \item[Dec$(c)$:] Compute $u=c_1\Red{s}-c_2$, recover
               $\mu$ from $u$
        \end{description}
    \end{block}

    \onslide<2->
    \begin{block}{What if Ring is the Integers?}
        \begin{itemize}
          \item Modulus clearly needs be $q=2^{\Omega(\text{poly}(n))}$ to avoid
          trivial brute-force
            \item[\RedCross]  Easy to break with LLL if \alert{short} =
              Euclidean norms
            \item Alternative: \alert{short} = Hamming weight in binary
        \end{itemize}
    \end{block}
\end{frame}

\begin{frame}    
    \frametitle{Binary Hamming Weights}
\begin{itemize}
\item<.-> Hamming weight $\wt(\vecv)$is number of non-zero
  elements
\smallskip
\item<.-> View integer $v$ in $\Z$ as an (infinite length) bit-vector
\smallskip
\item<+-> $\wt(v + w) \leq \wt(v)+\wt(w)$.
\smallskip
\item<.-> $\wt(v \cdot w ) \leq \wt(v) \cdot \wt(w)$
\end{itemize}
\onslide<+->
\begin{block}{(Tight) Example}
\begin{table}
\begin{tabular}{cr|cccc}
&3&&&1&1\\
$\cdot$&5&&1&0&1\\
\hline
&15&1&1&1&1\\
\end{tabular}
\end{table}
\end{block}
\begin{itemize}
\item<+-> Important: $\wt(v \cdot w ) \leq \wt(v) \cdot
  \wt(w)$ holds \alert{over the integers}
\end{itemize}
\end{frame}
\begin{frame}
  \frametitle{Mersenne Primes}
\begin{itemize}
\item<+-> Does $\wt(v \cdot w) \leq \wt(v) \cdot \wt(w)$ hold modulo $q$?
\item<+->[\RedCross] For most $q$, no!
\item<+-> Example: $q=2^{n}+1$
\begin{itemize}
\item<.-> $\wt(2 \bmod{q})$ = 1
\item<.-> $\wt(2^{n} \bmod{q})$ = 1
\item<.->[\RedCross] $\wt((2\cdot 2^{n}) = 2^{n}-1 \bmod{q})=n$ 
\end{itemize}
\end{itemize} 
\onslide<+->
\begin{block}{Mersenne Numbers}
\begin{itemize}
\item<.->[\GreenCheck] $\wt$ relations do hold modulo $q=2^{n}-1$. 
\item<+-> Reason: rep. of $2^{k}$ mod $q$  in $[0,2^{n}-1)$ is always
$2^{\ell}$
\item<.-> $v$ mod $q$ = summing $\wt(v)$ different $\wt(1)$
  numbers mod $q$
\item<+->  $q$ needs to be a \alert{Mersenne Prime} to avoid
  decomposition attacks.
\item[\RedCross]<.-> For desired parameters, Ramstake limited to $n=216091,756839$
\end{itemize}
\end{block}
\end{frame}

\begin{frame}\frametitle{Ramstake Description -- KeyGen}
\onslide<+->
On input randommess $seed$
\begin{enumerate}
\item<+-> Expand $seed$ into necessary randomness via $\vecr \gets
  \xof(seed)$
\begin{itemize}
\smallskip
\item<.-> Reason for $\text{xof}$: to
  shorten keys (a la Kyber/NewHope/Frodo)
\smallskip
\end{itemize}
\item<+-> Sample random $seed\_a$ (via $\vecr$)
\medskip
\item<+-> Choose $a \gets \Z_q$ randomly via $\text{xof}(seed\_a)$ 
\medskip
\item<+-> Choose \alert{low-weight} $s, e \in \Z_q$ (via $\vecr$)
\medskip
\item<+-> Compute $b = s\cdot a + e \bmod{q}$. 
\medskip
\item<+-> Secret key is $sk=(s, seed)$, public key is $pk=(seed\_a, b)$
\end{enumerate}
\end{frame}

\begin{frame}\frametitle{Ramstake Description -- Encapsulation}
\onslide<+->
On input $pk=(seed\_a,b)$, message $\mu$
\begin{enumerate}
\item<+-> Compute $r=xof(\mu)$.
\begin{itemize}
\item<.-> $r$ will be used both to sample 
\medskip
\item<.-> Reason for $\text{xof}$: to
  shorten public key (a la Kyber/NewHope/Frodo)
\end{itemize}
\medskip
\item<+-> Choose \alert{low-weight} $s, e \in \Z_q$
\medskip
\item<+-> Compute $b = s\cdot a + e \bmod{q}$. 
\medskip
\item<+-> Secret key is $(s)$, public key is $(seed\_a, b)$
\end{enumerate}
\end{frame}

\begin{frame}\frametitle{Parameter Sizes}
moo
\end{frame}

\begin{frame}\frametitle{Attacks (Part 1)}
moo
\end{frame}


\end{document}
