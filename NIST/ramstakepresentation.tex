\documentclass[11pt,t]{beamer}
\mode<presentation> {\usetheme{NIST}}

\usepackage{colortbl}
\usepackage{multirow}

\title{Ramstake}
\subtitle{Submission by Alan Szepieniec, KU Leuven}
\author{Jacob Alperin-Sheriff}
\institute{NIST}

\def\CC{{C\nolinebreak[4]\hspace{-.05em}\raisebox{.4ex}{\tiny\bf ++}}}


\usepackage{environ}% Required for \NewEnviron, i.e. to read the whole body of the environment
\makeatletter

\newcounter{acolumn}%  Number of current column
\newlength{\acolumnmaxheight}%   Maximum column height


% `column` replacement to measure height
\newenvironment{@acolumn}[1]{%
    \stepcounter{acolumn}%
    \begin{lrbox}{\@tempboxa}%
    \begin{minipage}{#1}%
}{%
    \end{minipage}
    \end{lrbox}
    \@tempdimc=\dimexpr\ht\@tempboxa+\dp\@tempboxa\relax
    % Save height of this column:
    \expandafter\xdef\csname acolumn@height@\roman{acolumn}\endcsname{\the\@tempdimc}%
    % Save maximum height
    \ifdim\@tempdimc>\acolumnmaxheight
        \global\acolumnmaxheight=\@tempdimc
    \fi
}

% `column` wrapper which sets the height beforehand
\newenvironment{@@acolumn}[1]{%
    \stepcounter{acolumn}%
    % The \autoheight macro contains a \vspace macro with the maximum height minus the natural column height
    \edef\autoheight{\noexpand\vspace*{\dimexpr\acolumnmaxheight-\csname acolumn@height@\roman{acolumn}\endcsname\relax}}%
    % Call original `column`:
    \orig@column{#1}%
}{%
    \endorig@column
}

% Save orignal `column` environment away
\let\orig@column\column
\let\endorig@column\endcolumn

% `columns` variant with automatic height adjustment
\NewEnviron{acolumns}[1][]{%
    % Init vars:
    \setcounter{acolumn}{0}%
    \setlength{\acolumnmaxheight}{0pt}%
    \def\autoheight{\vspace*{0pt}}%
    % Set `column` environment to special measuring environment
    \let\column\@acolumn
    \let\endcolumn\end@acolumn
    \BODY% measure heights
    % Reset counter for second processing round
    \setcounter{acolumn}{0}%
    % Set `column` environment to wrapper
    \let\column\@@acolumn
    \let\endcolumn\end@@acolumn
    % Finally process columns now for real
    \begin{columns}[#1]%
        \BODY
    \end{columns}%
}
\makeatother

\newcommand{\Red}[1]{{\textcolor{red} #1}}

%% BOILERPLATE PRESENTATION STUFF

% an underlining package
\usepackage{ulem}
% use normal italics for emphasis
\normalem


% dingbats
\usepackage{pifont}

\renewcommand{\Check}{\ding{52}}
\newcommand{\Cross}{\ding{55}}
\newcommand{\Hand}{\ding{43}}
\newcommand{\GreenCheck}{{\textcolor{green}\Check}}
\newcommand{\RedCross}{{\textcolor{red}\Cross}}

% alias for citations
\newcommand{\citationsize}{\footnotesize}

\subject{Theoretical Computer Science}

% suppress ugly QEDs in proofs
\def\qedsymbol{}


\begin{document}

%title page
    \setbeamertemplate{headline}[title_page]
    \setbeamertemplate{footline}[title_page]
    \csname beamer@calculateheadfoot\endcsname %recalculate head and foot dimension
        \begin{frame}
            \titlepage
        \end{frame}
%head and foot for body text    
    \setbeamertemplate{headline}[body]
    \setbeamertemplate{footline}[body]

%%%%%%%%%%%%%%%%%%%%%%%%%%%%%%%%%%%%%%%%%%
% \begin{frame}{Outline}
%     \vskip 5mm
%     {\parbox{.95\textwidth}{\tableofcontents[hideallsubsections]}}
% \end{frame}

%\begin{frame}
%    \titlepage
%\end{frame}
%
%\begin{frame}
%    \frametitle{Overview}
%
%    %\tableofcontents
%    \begin{itemize}
%        \item Scope of the project
%        \item Bayesian language models
%        \item Results
%        \item Research plan
%        \item Side projects
%        \item Reflections
%        \item Formalities
%    \end{itemize}
%
%\end{frame}
\section{Basic Idea}

\begin{frame}    
    \frametitle{Another Variation on Learning With Errors}

    \onslide<1->
    \begin{block}{(Ring)-LWE reminder}
        \begin{description}
            \item[KeyGen$(1^{\lambda})$:]
\begin{itemize}
\item Sample uniform $a \gets R$, \alert{short} (Euclidean norm)
              $\Red{s},\Red{e}$. 
\item Output $(a,b=\Red{s}\cdot a +
              \Red{e})$
\end{itemize}
            \item[Enc$(\mu)$:] 
\begin{itemize}
  \item Sample \alert{short} (Euclidean norm)
              $\Red{r},\Red{e}_1, \Red{e}_2$.
            \item Output  
              $c=(a\cdot \Red{r}+\Red{e}_1, b\cdot
              \Red{r}+\Red{e}_2+(q/2)\mu)$
\end{itemize}
             \item[Dec$(c)$:] Compute $u=c_1\Red{s}-c_2$, recover
               $\mu$ from $u$
        \end{description}
    \end{block}

    \onslide<2->
    \begin{block}{What if Ring is the Integers?}
        \begin{itemize}
          \item Modulus clearly needs be $q=2^{Omega(\poly(n)}}$ to avoid
          trivial brute-force
            \item[\RedCross]  Easy to break with LLL if \alert{short} =
              Euclidean norms
            \item Alternative: \alert{short} = Hamming weight in binary
        \end{itemize}
    \end{block}
\end{frame}

\begin{frame}    
    \frametitle{Hamming Weights and Mersenne Primes
\begin{block}{Hamming Weights}
\item Brief reminder: Hamming weight of a vector is number of non-zero
  elements
\end{block}
\begin{block}{Mersenne Primes}
\item Natural representation: modulo Mersenne numbers.
\item Need prime to avoid decomposition attacks???
\item[\RedCross] Very limited choice of moduli
\end{block}
\end{frame}

\begin{frame}\frametitle{Ramstake Description}

yadda yadda go a bit into the reconciliation method
\end{frame}

\begin{frame}\frametitle{Parameter Sizes}
\end{frame}

\begin{frame}\frametitle{Attacks (Part 1)}

\end{frame}


\end{document}
