% hack to deal with beamer/pgf bug
\RequirePackage{atbegshi}

% "draft" makes compilation faster; remove for final version
% shadow = put drop shadows behind blocks by default
% xcolor: svgnames imports lots of color names
% table allows for multi-color table rows
% pdftex is the compiler
\documentclass[shadow,xcolor=pdftex,svgnames,table,t]{beamer}

\usepackage{amsmath,amssymb,amsfonts}
\usepackage{array}
% need to load tikz defs before beamer defs, for some reason
% additional libraries for tikz and pgf
\usepackage{tikz}
\usetikzlibrary{matrix}
\usetikzlibrary{calc}
\usetikzlibrary{shadows}
\usetikzlibrary{scopes}
\usetikzlibrary{chains}
\usetikzlibrary{positioning}
\usetikzlibrary{fit}
\usetikzlibrary{backgrounds}
\usetikzlibrary{decorations.pathmorphing}
\usetikzlibrary{decorations.pathreplacing}
\usetikzlibrary{shapes.geometric}
\usetikzlibrary{shapes.symbols}
\usetikzlibrary{shapes.misc}
\usetikzlibrary{trees}

% color aliases
%\colorlet{Lattice}{Green}
%\colorlet{Set}{Red!10}

% global styles
\tikzset{
  % make drop shadows a bit softer
  every shadow/.style={opacity=.4,shadow xshift=.3ex,shadow yshift=-.3ex},
  % use \& to separate cells in matrices, avoiding fragility in beamer frames
  every matrix/.style={ampersand replacement=\&},
  % edges and joins with arrows and thicker
  every edge/.append style={->,thick},
  every join/.style={->,thick},
  % distance between nodes
  % node distance=1ex and 2ex,
}

% styles for drawing in Rn, especially lattices
\tikzset{
  % lattice transform must include "cm" dimension to work properly,
  % otherwise the x and y transforms become dependent
  Ztrans/.style={x={(1.5cm,1cm)},y={(.5cm,1.5cm)}},
  Zdual/.style={x={(1.5cm,-.5cm)},y={(.5cm,1cm)}},
  latttrans/.style={x={(1.8cm,.5cm)},y={(0.4cm,1.3cm)}},
  dualtrans/.style={x={(1.3cm,-.4cm)},y={(-.5cm,1.8cm)}},
  lattB/.style={x={(1.8cm,.5cm)},y={(0.4cm,1.4cm)}},
  dualB/.style={x={(1.4cm,-.4cm)},y={(-.5cm,1.8cm)}},
  shortB/.style={x={(.5cm,.2cm)},y={(-.1cm,1.7cm)}},
  latt2B/.style={x={(1cm,.1cm)},y={(.1cm,1.2cm)}},
  ulattB/.style={x={(2.0cm,.5cm)},y={(2.1cm,-.2cm)}},
  openball/.style={fill=structure.fg!20},
  ball/.style={openball,thin,draw=structure.fg},
  latt/.style={Lattice,fill},
  dist/.style={Brown,densely dashed,font=\footnotesize},
  offlabel/.style={Brown,inner sep=2pt,font=\tiny},
  axes/.style={->,Gray,style=very thin},
  fundamental/.style={Gray,fill opacity=.2},
  target/.style={fill,Brown},
  project/.style={thin,dashed},
  help lines/.style={Gray,very thin,step=.5cm},
  % expanding waves representing noise
  noise/.style={
    Lattice,
    opacity=.5,
    decorate,decoration={
      expanding waves,
      segment length=5pt,angle=10,
      post=moveto,post length=3pt
    },
  },
}

% lattice that can be dropped into any path.
% it would be sensible to set a default argument here, but
% commands with optional args don't parse correctly inside a
% tikz path
\newcommand*{\latt}[1]{%
  \foreach \a in {-4, ..., 4} {
    \foreach \b in {-4, ..., 4} {
      (\a,\b) circle (#1)
    }
  }
}

% style for functions mapping one point to another
\tikzset{map/.style={->,semithick}}

% style for an offset rounded box
\tikzset{showoff/.style={draw,fill=White,rounded corners,drop
    shadow,inner sep=6pt}}

% style for all algorithms
\tikzset{algorithm/.style={
    draw=#1!50!Black!60,solid,
    top color=White,bottom color=#1!50!Black!15,
    very thick,rounded corners,drop shadow,
    minimum size=.9cm,inner sep=6pt,outer sep=4pt
  }
}

% style for honest, malicious, trusted parties
\tikzset{honest/.style={algorithm=Green}}
\tikzset{oracle/.style={algorithm=Green}}
\tikzset{malicious/.style={algorithm=Red}}
\tikzset{trusted/.style={algorithm=Purple,inner sep=10pt}}

% style for notions
\tikzset{notion/.style={
    draw=#1!50!Black!60,solid,
    top color=White,bottom color=#1!50!Black!15,
    very thick,rounded corners,drop shadow,
    minimum height=6.5ex,inner sep=6pt,
  }
}

\tikzset{app/.style={notion=Blue}}
\tikzset{prim/.style={notion=Green}}
\tikzset{math/.style={notion=Red}}

% % style for a common reference string
\tikzset{crs/.style={draw,fill=White,cloud,cloud puffs=13,cloud ignores aspect,drop shadow}}

%% BOILERPLATE PRESENTATION STUFF

% an underlining package
\usepackage{ulem}
% use normal italics for emphasis
\normalem

% my custom theme
\usetheme{CJP}

% dingbats
\usepackage{pifont}

\renewcommand{\Check}{\ding{52}}
\newcommand{\Cross}{\ding{55}}
\newcommand{\Hand}{\ding{43}}
\newcommand{\GreenCheck}{{\color{Green}\Check}}
\newcommand{\RedCross}{{\color{Red}\Cross}}
\newcommand{\Orange}[1]{{\color{Orange} #1}}

% alias for citations
\newcommand{\citationsize}{\footnotesize}

\subject{Theoretical Computer Science}

% suppress ugly QEDs in proofs
\def\qedsymbol{}


% common stuff
% basic notation

\newcommand{\lamperp}{\Lambda^{\perp}}
\newcommand{\lam}{\Lambda}

% lattice
\newcommand{\lat}{\mathcal{L}}
% fundamental region
\newcommand{\piped}{\mathcal{P}}
% smoothing parameter
\newcommand{\smooth}{\eta}
% smoothing w/ epsilon
\newcommand{\smootheps}{\smooth_{\epsilon}}
% ball
\newcommand{\ball}{\mathcal{B}}
% cube
\newcommand{\cube}{\mathcal{C}}
% covering radius symbol
\newcommand{\cover}{\ensuremath{\mu}}
% Gram-Schmidt
\newcommand{\gs}[1]{\ensuremath{\widetilde{#1}}}
% GS minimum
\newcommand{\gsmin}{\ensuremath{\tilde{bl}}}
% volume operation
\DeclareMathOperator{\vol}{vol}
% Hermite normal form
\DeclareMathOperator{\hnf}{HNF}
% rank
\DeclareMathOperator{\rank}{rank}
% distance operator
\DeclareMathOperator{\dist}{dist}
% span operator
\DeclareMathOperator{\spn}{span}
% error function
\DeclareMathOperator{\erf}{erf}

% quantities that show up regularly
\newcommand{\wsln}{\ensuremath{\omega(\sqrt{\log n})}}
\newcommand{\wslm}{\ensuremath{\omega(\sqrt{\log m})}}

% support algorithms
\newcommand{\sampleZ}{\algo{Sample}\Z}
\newcommand{\sampleD}{\algo{SampleD}}
\newcommand{\tobasis}{\algo{ToBasis}}
\newcommand{\invert}{\algo{Invert}}
\newcommand{\genbasis}{\algo{GenBasis}}
\newcommand{\extbasis}{\algo{ExtBasis}}
\newcommand{\extlattice}{\algo{ExtLattice}}
\newcommand{\randbasis}{\algo{RandBasis}}
\newcommand{\gentrap}{\algo{GenTrap}}
\newcommand{\deltrap}{\algo{DelTrap}}
\newcommand{\noisy}{\algo{Noisy}}

% problems related to lattices
\newcommand{\svp}{\problem{SVP}}
\newcommand{\gapsvp}{\problem{GapSVP}}
\newcommand{\cogapsvp}{\problem{coGapSVP}}
\newcommand{\usvp}{\problem{uSVP}}
\newcommand{\sivp}{\problem{SIVP}}
\newcommand{\gapsivp}{\problem{GapSIVP}}
\newcommand{\cvp}{\problem{CVP}}
\newcommand{\gapcvp}{\problem{GapCVP}}
\newcommand{\cvpp}{\problem{CVPP}}
\newcommand{\gapcvpp}{\problem{GapCVPP}}
\newcommand{\bdd}{\problem{BDD}}
\newcommand{\gdd}{\problem{GDD}}
\newcommand{\add}{\problem{ADD}}
\newcommand{\incgdd}{\problem{IncGDD}}
\newcommand{\incivd}{\problem{IncIVD}}
\newcommand{\crp}{\problem{CRP}}
\newcommand{\gapcrp}{\problem{GapCRP}}
% problems on ideal lattices
\newcommand{\igvp}{\problem{IGVP}}
\newcommand{\incigvp}{\problem{IncIGVP}}
% avg-case stuff
\newcommand{\sis}{\problem{SIS}}
\newcommand{\isis}{\problem{ISIS}}
\newcommand{\ilwe}{\problem{ILWE}}
\newcommand{\lwe}{\problem{LWE}}
\newcommand{\rlwe}{\problem{RLWE}}
\newcommand{\lwr}{\problem{LWR}}
\newcommand{\rlwr}{\problem{RLWR}}
\newcommand{\dlwe}{\problem{DLWE}}
\newcommand{\lpn}{\problem{LPN}}
\newcommand{\Psibar}{\ensuremath{\bar{\Psi}}}

% macros for matrices and vectors
\newcommand{\matA}{\ensuremath{\mathbf{A}}}
\newcommand{\matB}{\ensuremath{\mathbf{B}}}
\newcommand{\matC}{\ensuremath{\mathbf{C}}}
\newcommand{\matD}{\ensuremath{\mathbf{D}}}
\newcommand{\matE}{\ensuremath{\mathbf{E}}}
\newcommand{\matF}{\ensuremath{\mathbf{F}}}
\newcommand{\matG}{\ensuremath{\mathbf{G}}}
\newcommand{\matH}{\ensuremath{\mathbf{H}}}
\newcommand{\matI}{\ensuremath{\mathbf{I}}}
\newcommand{\matJ}{\ensuremath{\mathbf{J}}}
\newcommand{\matK}{\ensuremath{\mathbf{K}}}
\newcommand{\matL}{\ensuremath{\mathbf{L}}}
\newcommand{\matM}{\ensuremath{\mathbf{M}}}
\newcommand{\matN}{\ensuremath{\mathbf{N}}}
\newcommand{\matO}{\ensuremath{\mathbf{O}}}
\newcommand{\matP}{\ensuremath{\mathbf{P}}}
\newcommand{\matQ}{\ensuremath{\mathbf{Q}}}
\newcommand{\matR}{\ensuremath{\mathbf{R}}}
\newcommand{\matS}{\ensuremath{\mathbf{S}}}
\newcommand{\matT}{\ensuremath{\mathbf{T}}}
\newcommand{\matU}{\ensuremath{\mathbf{U}}}
\newcommand{\matV}{\ensuremath{\mathbf{V}}}
\newcommand{\matW}{\ensuremath{\mathbf{W}}}
\newcommand{\matX}{\ensuremath{\mathbf{X}}}
\newcommand{\matY}{\ensuremath{\mathbf{Y}}}
\newcommand{\matZ}{\ensuremath{\mathbf{Z}}}
\newcommand{\matzero}{\ensuremath{\mathbf{0}}}

\newcommand{\veca}{\ensuremath{\mathbf{a}}}
\newcommand{\vecalpha}{\ensuremath{\mathbf{\alpha}}}
\newcommand{\vecb}{\ensuremath{\mathbf{b}}}
\newcommand{\vecc}{\ensuremath{\mathbf{c}}}
\newcommand{\vecd}{\ensuremath{\mathbf{d}}}
\newcommand{\vece}{\ensuremath{\mathbf{e}}}
\newcommand{\vecf}{\ensuremath{\mathbf{f}}}
\newcommand{\vecg}{\ensuremath{\mathbf{g}}}
\newcommand{\vech}{\ensuremath{\mathbf{h}}}
\newcommand{\veck}{\ensuremath{\mathbf{k}}}
\newcommand{\vecm}{\ensuremath{\mathbf{m}}}
\newcommand{\vecp}{\ensuremath{\mathbf{p}}}
\newcommand{\vecq}{\ensuremath{\mathbf{q}}}
\newcommand{\vecr}{\ensuremath{\mathbf{r}}}
\newcommand{\vecs}{\ensuremath{\mathbf{s}}}
\newcommand{\vect}{\ensuremath{\mathbf{t}}}
\newcommand{\vecu}{\ensuremath{\mathbf{u}}}
\newcommand{\vecv}{\ensuremath{\mathbf{v}}}
\newcommand{\vecw}{\ensuremath{\mathbf{w}}}
\newcommand{\vecx}{\ensuremath{\mathbf{x}}}
\newcommand{\vecy}{\ensuremath{\mathbf{y}}}
\newcommand{\vecz}{\ensuremath{\mathbf{z}}}
\newcommand{\veczero}{\ensuremath{\mathbf{0}}}

\DeclareMathOperator{\diag}{diag}
\renewcommand{\O}{\mathcal{O}}

% "left-right" pairs of symbols

% inner product
\newcommand{\inner}[1]{\langle{#1}\rangle}
\newcommand{\innerfit}[1]{\left\langle{#1}\right\rangle}
% absolute value
\newcommand{\abs}[1]{\lvert{#1}\rvert}
\newcommand{\absfit}[1]{\left\lvert{#1}\right\rvert}
% a set
\newcommand{\set}[1]{\{{#1}\}}
\newcommand{\setfit}[1]{\left\{{#1}\right\}}
% parens
\newcommand{\parens}[1]{({#1})}
\newcommand{\parensfit}[1]{\left({#1}\right)}
% tuple, alias for parens
\newcommand{\tuple}[1]{\parens{#1}}
\newcommand{\tuplefit}[1]{\parensfit{#1}}
% square brackets
\newcommand{\bracks}[1]{[{#1}]}
\newcommand{\bracksfit}[1]{\left[{#1}\right]}
% rounding off
\newcommand{\round}[1]{\lfloor{#1}\rceil}
\newcommand{\roundfit}[1]{\left\lfloor{#1}\right\rceil}
% floor function
\newcommand{\floor}[1]{\lfloor{#1}\rfloor}
\newcommand{\floorfit}[1]{\left\lfloor{#1}\right\rfloor}
% ceiling function
\newcommand{\ceil}[1]{\lceil{#1}\rceil}
\newcommand{\ceilfit}[1]{\left\lceil{#1}\right\rceil}
% length of some vector, element
\newcommand{\length}[1]{\lVert{#1}\rVert}
\newcommand{\lengthfit}[1]{\left\lVert{#1}\right\rVert}


% blackboard symbols

\newcommand{\C}{\ensuremath{\mathbb{C}}}
\newcommand{\D}{\ensuremath{\mathbb{D}}}
\newcommand{\F}{\ensuremath{\mathbb{F}}}
\newcommand{\G}{\ensuremath{\mathbb{G}}}
\newcommand{\J}{\ensuremath{\mathbb{J}}}
\newcommand{\N}{\ensuremath{\mathbb{N}}}
\newcommand{\Q}{\ensuremath{\mathbb{Q}}}
\newcommand{\R}{\ensuremath{\mathbb{R}}}
\newcommand{\T}{\ensuremath{\mathbb{T}}}
\newcommand{\Z}{\ensuremath{\mathbb{Z}}}

\newcommand{\Zt}{\ensuremath{\Z_t}}
\newcommand{\Zp}{\ensuremath{\Z_p}}
\newcommand{\Zq}{\ensuremath{\Z_q}}
\newcommand{\ZN}{\ensuremath{\Z_N}}
\newcommand{\Zps}{\ensuremath{\Z_p^*}}
\newcommand{\ZNs}{\ensuremath{\Z_N^*}}
\newcommand{\JN}{\ensuremath{\J_N}}
\newcommand{\QRN}{\ensuremath{\mathbb{QR}_N}}

% MARGIN NOTES

\newif\ifnotes\notestrue
%\newif\ifnotes\notesfalse


\ifnotes
\usepackage{color}
\definecolor{mygrey}{gray}{0.50}
\newcommand{\noteby}[2]{{\textcolor{mygrey}{\footnotesize{\bf (#1:} {#2}{\bf ) }}}}
\newcommand{\noteswarning}{{\begin{center} {\Large WARNING: NOTES ON}\end{center}}}

\else

\newcommand{\noteby}[2]{{}}
\newcommand{\noteswarning}{{}}

\fi

\newcommand{\cnote}[1]{{\noteby{Chris}{#1}}}
\newcommand{\jnote}[1]{{\noteby{Jacob}{#1}}}
\newcommand{\dnote}[1]{{\noteby{Daniel}{#1}}}


% trace
\DeclareMathOperator{\trace}{Tr}
\DeclareMathOperator{\lcm}{lcm}
\DeclareMathOperator{\msb}{msb}
\renewcommand{\O}{\mathcal{O}}
\newcommand{\frakp}{\mathfrak{p}}
\newcommand{\Blue}[1]{{\color{Blue}#1}}
\newcommand{\Red}[1]{{\color{Red}#1}}
\newcommand{\Purple}[1]{{\color{Purple}#1}}
\newcommand{\Green}[1]{{\color{Green}#1}}

% fix weird spacing of \pmod, and introduce \pmod* command; see
% http://tex.stackexchange.com/questions/39221/removing-extra-space-with-pmod-command
\makeatletter
\renewcommand{\pod}[1]{\mathchoice
  {\allowbreak \if@display \mkern 18mu\else \mkern 8mu\fi (#1)}
  {\allowbreak \if@display \mkern 18mu\else \mkern 8mu\fi (#1)}
  {\mkern4mu(#1)}
  {\mkern4mu(#1)}
}
\makeatletter
\let\@@pmod\pmod
\DeclareRobustCommand{\pmod}{\@ifstar\@pmods\@@pmod}
\def\@pmods#1{\mkern4mu({\operator@font mod}\mkern 6mu#1)}
\makeatother
%%%% 

\title
{ Practical Bootstrapping in\\Quasilinear Time
}

\author
{ {\Large Jacob Alperin-Sheriff}
  \and
  {\Large Chris Peikert}
}

\institute{School of Computer Science\\Georgia Tech}

\date
{
  CRYPTO 2013\\
  19 August 2013
}

% restricts presentation to specified frames, for faster compilation;
% comment out the line to get the entire presentation

%\includeonlyframes{enhance2}

\begin{document}

%%% put less vertical space around displayed equations
\setlength{\abovedisplayskip}{8pt plus 2pt minus 1pt}
\setlength{\belowdisplayskip}{8pt plus 2pt minus 0pt}
\setlength{\abovedisplayshortskip}{0pt plus 2pt}
\setlength{\belowdisplayshortskip}{8pt plus 2pt minus 1pt}

\begin{frame}[label=title]
  \titlepage
\end{frame}

\begin{frame}[label=fhe]{Fully Homomorphic Encryption {\citationsize
      [RAD'78,Gentry'09]}}
  \begin{itemize}
  \item<+-> FHE lets you do this:
    \begin{center}
      \begin{tikzpicture}
       \node (c) {$\fbox{$\mu$}$};
  
        \node[right=of c,algorithm=Green] (eval)
        {$\text{Eval}\parens*{f}$};

        \node[right=of eval] (out) {$\fbox{$f(\mu)$}$};

        \path
        (c) edge (eval)
        (eval) edge (out)
        ;
      \end{tikzpicture}
    \end{center}
    % where $\abs{f(\mu)}$ and decryption time don't depend on
    % $\abs{f}$.

    \medskip
    A cryptographic ``holy grail'' with countless applications.

    \medskip Solved in {\citationsize [Gentry'09]}, followed by

    \hfill
    {\scriptsize [vDGHV'10,BV'11a,BV'11b,BGV'12,B'12,GSW'13,\ldots]}
  \end{itemize}

  \onslide<+->
  %\bigskip
  \begin{itemize}
  \item ``Naturally occurring'' schemes are \alert<.>{somewhat
      homomorphic}~(SHE):

    can only evaluate functions of an \alert<.>{\textit{a priori}
      bounded} depth.
  \end{itemize}

  \begin{center}
    \begin{tikzpicture}[node distance=.4cm]
      \node (c) {$\Green{\fbox{$\mu$}}$};
      
      \node[right=of c,algorithm=Green] (eval)
      {$\text{Eval}\parens*{f}$};
      
      \node[right=of eval] (cprime) {$\color{Orange} \fbox{$f(\mu)$}$};

      \node[right=of cprime,algorithm=Green] (eval2)
      {$\text{Eval}\parens*{g}$};

      \node[right=of eval2] (cdp) {$\Red{\fbox{$g(f(\mu))$}}$};

      \path
      (c) edge (eval)
      (eval) edge (cprime)
      (cprime) edge (eval2)
      (eval2) edge (cdp)
      ;
    \end{tikzpicture}
  \end{center}
  \begin{itemize}
  \item<+-> Thus far, ``\alert{bootstrapping}'' is required to
    achieve~\alert{unbounded} FHE.
  \end{itemize}
\end{frame}

\begin{frame}[label=bootstrapping]{Bootstrapping: SHE $\to$ FHE {\citationsize [Gentry'09]}}
  \begin{itemize}
  \item<+-> \alert<.>{Homomorphically evaluates the SHE decryption
      function} to ``refresh'' a ciphertext $\Red{\fbox{$\mu$}}$,
    allowing further homomorphic operations.

    \begin{center}
      \begin{tikzpicture}[every node/.style={node distance=.7cm}]
        \node (sk)
        {$\Green{\fbox{$sk$}}$};

        \node[right=of sk,algorithm=Green] (eval)
        {$\text{Eval}\parens*{\; \text{Dec} \parens*{\cdot\;,
            \Red{\fbox{$\mu$}}} \;}$};

        \node[right=of eval] (out)
        {${\color{Orange} \fbox{$\mu$}}$};

        \path
        (sk) edge (eval)
        (eval) edge (out)
        ;

      \end{tikzpicture}
    \end{center}

    % \onslide<+->
    % \begin{itemize}
    % \item \alert<.>{Goal}: Efficiency!  Minimize depth $d$ and
    %   size $s$ of decryption ``circuit.''

    %   \medskip
    % \item Best SHEs {\citationsize [BGV'12]} can evaluate in time
    %   $\tilde{O}(d \cdot s \cdot \lambda)$.
    % \end{itemize}

    \smallskip

  \item<+-> Intensive study {\citationsize
      [G'09,GH'11a,GH'11b,BGV'12,GHS'12b]}, but

    \alert<.>{still very inefficient} -- the main bottleneck in FHE,
    by far.

    \medskip
  \item<+->  Notable past results:

    \medskip
    \renewcommand{\arraystretch}{1.3}
    \begin{tabular}{|c|c|c|}
      \hline
      \textbf{Notes} & \textbf{Runtime} & \textbf{Reference} \\
      \hline
      Seminal work & $\tilde{O}(\lambda^4)$ & {[\citationsize
        Gentry'09]}\\
      \hline
      \alert{One-bit}
      ciphertexts & $\tilde{O}(\lambda^2)$ for $\lambda$ ciphertexts & 
      {\citationsize [BGV'12]}\\
      \hline
      \alert{``Packed''}
      ciphertexts & 
      $\tilde{O}(\lambda)$ & {\citationsize [GHS'12a,GHS'12b]}\\
      \hline    
    \end{tabular}
  \end{itemize}
\end{frame}

\begin{frame}[label=prev]{Bootstrapping in Quasilinear Time {\citationsize[GHS'12a,GHS'12b]}}
  % \cnote{The picture still doesn't make sense... in GHS'12b the
  %   compiler is applied to the entire Dec circuit, not just the
  %   ``maps.''  The Dec circuit is somehow the composition of various
  %   log-depth pieces (including mod $\Phi_{k}$), and the compiler is
  %   applied to the whole. The picture should reflect this... I don't
  %   think we want or need text saying what the Dec circuit looks like
  %   internally, only that it is quite complex, with large constant/log
  %   factors in its depth/size.}
  \begin{center}
    \begin{tikzpicture}[node distance=.5cm]
\node[algorithm=Purple,minimum width=10cm,minimum height=3cm] at (0,0)
{};

\node at (0,1.1) {Homomorphic Compiler \citationsize [GHS'12a]};
      \node at (-3,-0.25) [algorithm=Red,text
        width=2.5cm, minimum height=1.9cm,align=center] (toslots)  {};

 \node at (-3,0.35) {Coeffs to Slots}; 
\node at (-3,-0.5) [algorithm=Purple,align=center] (modPhi)  { ${} \bmod \Phi_{k}(X)$};

   \node at (0.5,-0.2) [algorithm=Green,text width=1cm,align=center] (round)
       {SIMD Round};
% \matrix[algorithm=Red,left=of round, row sep=-2pt] (toCRT) {
  
   \node at (3.55,-0.2) [algorithm=Red,text width=1.2cm,align=center] (invert) 
      {Slots to Coeffs};
    %  };

       \path
      (toslots) edge (round)
      (round) edge (invert)
     ;
  
    \end{tikzpicture}
  \end{center}

  % \begin{enumerate}
 %    \item
 %  % \item<+-> \alert<.>{Map} plaintext coefficients  to slots, reduce
 %  %   mod $\Phi_{k}(X)$ (log-depth)
 %  %   \smallskip
 %  % \item<+-> \alert<.>{Batch-Round} the coefficients using integer
 %  %   rounding circuit (log-depth)
 %  %   \smallskip
  
 %  %    \item<+-> \alert<.>{Reverse-Map} coefficients to polynomial (log-depth)
 % \end{enumerate}

  \begin{itemize}
  \item<+-> Decryption works modulo $k$th cyclotomic polynomial
    $\Phi_{k}(X)$.

  \item<+-> Decryption circuit is \alert<.>{$O(\log \lambda)$ depth},
    \alert<.>{$O(\lambda \log \lambda)$ size}.

    \smallskip Can evaluate it homomorphically in
    \alert<.>{$\tilde{O}(\lambda)$ time} via {\citationsize
      [GHS'12a]} compiler.
     
  \item<+-> \uline{Problems}:
    \begin{itemize}
      \smallskip
    \item<.->[\RedCross] Log-depth mod-$\Phi_{k}(X)$ circuit is
      \alert<.>{complex}, w/large hidden constants.
      \smallskip
    \item<+->[\RedCross\RedCross] {\citationsize [GHS'12a]} compiler is
      \alert{very complex}, w/\alert{very large polylog overhead}.
    \end{itemize}

  \item<+->[$\Rightarrow$] Appears \uline{practically slower} than
    asymptotically worse methods.
  \end{itemize}
\end{frame}



\begin{frame}[label=results]{Our Results}
  \onslide<+->
  \alert<.>{Practical} bootstrapping algorithms with
  \alert<.>{quasilinear} $\tilde{O}(\lambda)$ runtimes:

  \begin{enumerate}
  \item<+-> For ``\alert<.>{unpacked}'' (single-bit) plaintexts:

    \begin{itemize}
    \item[\GreenCheck] Extremely simple!  \uline{New technique}:
      \alert<.>{isolate} the message coefficient.
      
      \medskip
    \item<+->[\GreenCheck] Can easily use \alert<.>{power-of-2
        cyclotomic rings}: fast, easy to implement.

      \medskip
    \item<+-> Cf.~{\citationsize [BGV'12]}: $\tilde{O}(\lambda^2)$
      \alert<.>{amortized} over $\lambda$ ciphertexts, costly to use
      power-of-2 cyclotomics.
    \end{itemize}

    \medskip
  \item<+-> For ``\alert<.>{packed}'' (many-bit) plaintexts:

    \begin{itemize}
    \item<+-> \uline{New technique}: substantial enhancement of
      \alert<.>{ring-switching} {\citationsize [GHPS'12]}.

      \medskip
    \item<+->[\GreenCheck] \alert<.>{Appears very practical}, avoids
      both main inefficiencies of {\citationsize [GHS'12b]}:

      no homomorphic reduction modulo $\Phi_{k}(X)$, no generic
      compilation.

      \medskip
    \item<+->[\GreenCheck] \alert<.>{Completely algebraic description}
      -- no ``circuits.''

      \medskip
    \item<+->[\GreenCheck] Works with \alert<.>{any} SHE cyclotomic
      plaintext ring and modulus.

      % \smallskip
      % (No special algebraic structure required.)
    \end{itemize}
  \end{enumerate}
\end{frame}

\begin{frame}[label=example]{Example SHE Scheme\; {\citationsize [LPR'10,BV'11,BGV'12]}}
  \begin{itemize}
  \item<+-> Let $\Blue{R} \cong \Z[X]/\Phi_{k}(X)$ be the $k$th
    cyclotomic ring; let $\Blue{R_{q}} = \Blue{R/qR}$.

    \medskip
  \item<+-> Ciphertext $c=(c_{0}, c_{1}) \in \Blue{R_{q}^{2}}$ encrypting
    $\mu \in \Blue{R_2}$ under $s \in \Blue{R}$ satisfies
    \begin{equation*}
      v = c_{0} + c_{1} \cdot s\, =\, \tfrac{q}{2} \mu + e \in \Blue{R_q}.  
    \end{equation*}
    \vspace{-12pt}
    \begin{itemize}
    \item<+-> \alert<.>{Unpacked} ciphertext: $\mu \in \Z_{2}
      \subseteq \Blue{R_{2}}$, i.e., just a constant polynomial.

      \smallskip 
    \item<+-> \alert<.>{Packed} ciphertext: $\mu$ may use all of
      $\Blue{R_2}$, e.g., multiple ``slots'' {\citationsize [SV'11]}.
      \smallskip
    \end{itemize}

    \smallskip
  \item<+-> The \alert<.>{decryption function} is
    \[ \text{Dec}_{s}(c) := \round{v} = \mu \in \Blue{R_{2}}. \]
    \onslide<+-> Rounding $\round{\cdot} \colon \Zq \to \Z_{2}$ is
    applied to coefficients of $v$ in a certain \alert{``decryption
      basis''} $\Blue{B}$ of $\Blue{R}$:
    \[v=\sum
    v_j \cdot \Blue{b_j} \quad \mapsto \quad \sum
    \round{v_{j}} \cdot \Blue{b_{j}} \in \Blue{R_{2}}. \]
  \end{itemize}
\end{frame}

\begin{frame}[label=practical]{Overview of Our Bootstrapping
    Algorithm}
  \onslide<+->
  \begin{block}{}
    \begin{itemize}
    \item Want to homomorphically evaluate decryption function 
    \end{itemize}
    \begin{equation*}
      \text{Dec}_{s}(c) := \round{v} = \round{c_{0} +
        c_{1} \cdot s} = \round*{\tfrac{q}{2}\mu+e} \in \Blue{R_{2}}.
    \end{equation*}
  \end{block}
  
  \begin{enumerate}
  \item<+-> \alert<.-.(1)>{Prepare}: view $c$ as an
    encryption of \alert<.-.(1)>{plaintext} $v \in \Blue{R_{q}}$. \\
    {\citationsize (Also switch to larger ciphertext ring and modulus
      for upcoming homom ops.)}
  \end{enumerate}

  \onslide<+-> Evaluate the following \alert<.->{homomorphically} on
  $v$: \smallskip
  \begin{enumerate}
    \setcounter{enumi}{1}
    
  \item<.-> \alert<.>{Isolate} ``message coefficients'' $v_{j}$, zero
    out all others.
    
    \smallskip For unpacked ciphertext: $\sum v_j \cdot \Blue{b_j} \;
    \mapsto\; v_{0} \cdot \Blue{b_{0}}$.
    
    \medskip
  \item<+-> \alert<.>{Round} $\sum v_{j} \cdot \Blue{b_{j}} \;
    \mapsto\; \sum \round{v_{j}} \cdot \Blue{b_{j}} \in \Blue{R_2}$.
    
    \begin{enumerate} 
      \renewcommand{\theenumii}{\alph{enumii}}
    \item<+-> \alert<.>{Map} message coefficients $v_{j}$ to
      ``$\Zq$-slots'' of a certain ring $\Red{S_{q}}$.
      
      \medskip
    \item<+-> \alert<.>{Batch-round} every $v_{j}$ to $\round{v_{j}}
      \in \Z_{2}$, using SIMD ops {\citationsize [SV'11,GHS'12b]}.
      
      \medskip
    \item<+-> \alert<.>{Reverse-map} $\Z_{2}$-slots back to
      $\Blue{B}$-coefficients.
    \end{enumerate}
  \end{enumerate}
\end{frame}

% \begin{frame}[label=unpacked-main]{Unpacked Ciphertexts:
%     \alert{Isolating} Message Coefficients}
%   \begin{itemize}
%   \item<+-> \alert<.-.(1)>{Isolate} message-carrying coefficient 
%     $v_{0}$ of $v(X)$ by homomorphically ``\alert<.-.(1)>{tracing
%       down}'' a tower of cyclotomic rings
%     $\O_{2k}/\O_{k}/\cdots/\O_{4}/\Z$.

%     \smallskip
%   \item<+-> \uline{Key idea}: Sum the two \alert<.>{automorphisms} of
%     $\O_{2i}/\O_{i}$ at each step.
%     %\cnote{Possible to reveal one row of picture at a time?}
  
%     \onslide<+->
%     \begin{center}
%       \begin{tikzpicture}[every node/.style={node distance=.7cm}]
%         \matrix[matrix of nodes, column sep=1cm, row sep=1cm]
%         {
%           %
%           |(V2k)| $v_0 + v_1 X + v_{2} X^{2} + \cdots v_{k-1} X^{k-1}$ \&
%           |(O2k)| $\Z[X]/(X^{k}+1) \cong \O_{2k}$ \\
%           % 
%          |(Vk)| {\uncover<4->{$v_0 + 0 X + v_{2} X^{2} + \cdots + 0 X^{k-1}$ }}\&
%           |(Ok)| {\uncover<4->{$\Z[X^{2}]/(X^{k}+1) \cong \O_{k}$}} \\
%           %
%           % |(V8)|  $v_{0} + v_{k/4} X^{k/4} + \cdots + v_{3k/4} X^{3k/4}$ \&
%           % |(O8)| $\Z[X^{k/4}]/(X^{k}+1) \cong \O_{8}$ \\
%           % 
%           |(V4)| {\uncover<5->{$v_{0} + v_{k/2} X^{k/2}$}} \&
%           |(O4)| {\uncover<5->{$\Z[X^{k/2}]/(X^{k}+1) \cong \O_{4}$}} \\
%           % 
%           |(V2)| {\uncover<6->{$v_{0}$}} \&
%           |(O2)| {\uncover<6->{$\Z \cong \O_{2} \cong \O_{1}$}} \\ 
%         };


%         \uncover<4->{    \path
%           (V2k) edge   node[right] {\citationsize
%             $\tfrac{1}{2}(v(X)+v(-X))$}  (Vk)
%           (O2k) edge[-] (Ok)
%           ;}
        
%         \uncover<5->{    \path
%           (Vk) edge[dotted]   node[right] {\citationsize
%           }  (V4)
%           (Ok) edge[-,dotted] (O4)
%           ;}

%         \uncover<6->{    \path
%           (V4) edge   node[right] {\citationsize
%             $\tfrac{1}{2}(v(X^{k/2})+v(-X^{k/2}))$}  (V2)
%           (O4) edge[-] (O2)
          
%           ;}
%       \end{tikzpicture}
%     \end{center}
 
%   \end{itemize}
% \end{frame}

% \begin{frame}[c,label=main]
%   \begin{center}
%     {\LARGE \structure{Main Result:\\ \medskip Bootstrapping Packed Ciphertexts}}
%   \end{center}
% \end{frame}

\begin{frame}[label=packed]{Packed Ciphertexts: Mapping Coefficients to Slots}
  \begin{itemize}
   \item<+-> Homom ops induce \alert<.>{SIMD operations} on
     plaintext ``slots'' {\citationsize [SV'11]}.
     \begin{center}
       \begin{tikzpicture}[node distance=.75cm]
         \node (Sq) {$\Red{S_q}$};
         \node [right=7.15cm] (Sq2) {$\Red{S_q}$}; 

         {\tikzstyle{every node}=[minimum size=.9cm]
           \matrix [below=.25cm,nodes=draw,row sep=0mm,column sep=0mm] (slotsinit) {
             \node {\scriptsize $v_1$}; \& 
             \node {\scriptsize $v_2$}; \&
             \node {\scriptsize $\ldots$}; \& 
             \node {\scriptsize $v_n$}; \\
           };
         }

         \node [algorithm=Green, right=of slotsinit] (Eval) {$\text{Eval}(f)$};

         {\tikzstyle{every node}=[minimum size=.9cm]
           \matrix [right=of Eval,nodes=draw,row sep=0mm,column sep=0mm] (slotsafter) {
             \node {\scriptsize $f(v_1)$}; \& 
             \node {\scriptsize $f(v_2)$}; \&
             \node {\scriptsize $\ldots$}; \& 
             \node {\scriptsize $f(v_n)$}; \\
           };
         }

         \path  
         (slotsinit) edge (Eval)
         (Eval) edge (slotsafter)
         ;
       \end{tikzpicture}
     \end{center}

     \medskip
  \item<+-> A CRT set $\Red{C} = \set{\Red{c_{j}}}$ has $\Red{c_j}=1$
    in slot $j$, $\Red{c_j}=0$ in other slots.
    % \begin{itemize}
    %   \smallskip
    % \item<.-> 
    %   \smallskip
    % \item<.-> Allows for batch operations on the integer values in
    %   the slots 
    % \end{itemize}
    
    \smallskip
  \item<+-> \uline{Our goal}: homomorphically map 
    \[\sum v_{j} \cdot
    \Blue{b_{j}} \in \Blue{R_{q}} \; \longmapsto\; \sum
    v_{j} \cdot \Red{c_{j}} \in \Red{S_{q}}.\]

    \onslide<+-> \medskip I.e., evaluate the \alert<.>{$\Z$-linear
      map} $L \colon \Blue{R} \to \Red{S}$ defined by
    $L(\Blue{b_{j}})=\Red{c_{j}}.$

    % \begin{tikzpicture}[remember picture,overlay]
    %   \path (current page.south west) node[anchor=south
    %   west,font=\citationsize] (Zlinear) {$^{*}\Z$-linear:
    %   $L(b+b')=L(b)+L(b')$, $L(v \cdot b) = v \cdot L(b)$ for any
    %   $b,b' \in R, v \in \Z$.};
    % \end{tikzpicture}
    % 
    % \vspace{-20pt}

    \smallskip
  \item<+-> \alert<.>{Ring-switching} {\citationsize [GHPS'12]} lets
    us eval any $R'$-linear map
    \[ L \colon \Blue{R} \to R' \] \onslide<+-> \ldots but only for a
    \alert<.>{subring} $R' \subseteq \Blue{R}$.
  \end{itemize}
\end{frame}

\begin{frame}[label=enhance1]{Enhanced Ring-Switching: First Attempt}
  \begin{itemize}
  \item<+-> Let $\Blue{R=\O_{k}}$, $\Red{S=\O_{\ell}}$ where
    $\gcd(k,\ell)=1$, $\Purple{\lcm(k,\ell)=k\ell}$.
  \end{itemize}
  
  \onslide<+->
  \begin{center}
    \begin{tikzpicture}[every node/.style={node distance=1.2cm}]
      \node (R) {$\Blue{R}$};
      \node[above right=of R] (T) {$\Purple{T} = \Blue{R}+\Red{S} = \Purple{\O_{k\ell}}$};
      \node[below right=of R,align=left] (E) {$E = \Blue{R} \cap
        \Red{S} = \Z$};
      \node[below right=of T,Red] (S) {$S$};
      
      \path
      (R) edge[-] (E)
      (E) edge[-] (S)
      (R) edge[->] node[above,sloped] {\citationsize embed} (T)
      ;

      \uncover<3->{
        \path
        (T) edge[->] node[above,sloped] {$\bar{L}$} (S)     
        (R) edge[->,dashed] node[above] {$L$} node[below] {(induced)} (S)
        ;}
    \end{tikzpicture}
  \end{center}
  
  \begin{enumerate}
  \item<.-> Embed ciphertext $c \in \Blue{R}$ in the
    \Purple{compositum} ring $\Purple{T}=\Blue{R}+\Red{S}=\Purple{\O_{k\ell}}$
    \smallskip
    \begin{itemize}
    \item<+-> Lemma: for any $E$-linear $L \colon \Blue{R} \to
      \Red{S}$, there is an $\Red{S}$-linear $\bar{L} \colon
      \Purple{T} \to \Red{S}$ that agrees with $L$ on $\Blue{R}$.
    \end{itemize}

  \item<+-> Homomorphically apply $\Red{S}$-linear $\bar{L} \colon
    \Purple{T} \to \Red{S}$ using ring-switching.
  \end{enumerate}

  \begin{itemize}
  \item<+->[\GreenCheck] We just homomorphically evaluated
    $L(v) = \bar{L}(v)$ !

    \medskip
  \item<+->[\RedCross\RedCross] \uline{Problem}: degree of
    $\Purple{T}$ is \alert<.>{quadratic}, therefore so is runtime \&
    space.
  \end{itemize}
\end{frame}


\begin{frame}[label=enhance2]{Enhanced Ring-Switching, Efficiently}
  \begin{itemize}
  \item<+-> Use a chain of \alert<.>{hybrid rings} $\Purple{H^{(i)}}$
    to keep the compositums $\Purple{T^{(i)}}$ small.

    \smallskip Equivalently, need $E^{(i)} = \gcd(H^{(i-1)}, H^{(i)})$
    to be large.

    \medskip
  \item<+-> Now can only evaluate a sequence of
    \uline{$E^{(i)}$-linear} fcts $\Purple{H^{(i-1)}} \to
    \Purple{H^{(i)}}$.

    \onslide<+-> \medskip Can we evaluate our desired map
    $L(\Blue{b_{j}}) = \Red{c_{j}}$ ?

    \medskip
  \item<+->[\GreenCheck] Yes! \uline{If} $\Blue{B},\Red{C}$ have a
    \alert<.>{product structure} induced by cyclotomic towers.
    \begin{itemize}
    %   \smallskip
    % \item<+-> In each step, map one small factor of $\Blue{B}$ to
    %   corresponding one of $\Red{C}$.

      \smallskip
    \item<+-> \uline{Key point}: naive ``power'' basis $\set{X^{j}}$
      of $\Z[X]/\Phi_{k}(X)$ doesn't work.

      \smallskip Instead, need structured \alert<.>{``powerful''} or
      \alert<.>{``decoding''} bases {\citationsize [LPR'13]}.
    \end{itemize}
  \end{itemize}

  \onslide<1->
\begin{center}
  \begin{tikzpicture}[every node/.style={node distance=1.3cm}]
    \node[Blue] (H0) {$R=H^{(0)}$};
    \node[above right=of H0,Purple] (T1) {$T^{(1)}$};
    \node[below right=of H0] (E1) {$E^{(1)}$};
    \node[below right=of T1,Purple] (H1) {$H^{(1)}$};
    \onslide<4->{
      \node at (.3,-0.55) [rotate=270] (supsetB0B1) {$\supset$};
      \node at (0.07,-1.05) (B0B1) {$\Blue{B=B_0B_1}$};
      \node at (4.15,-0.55) [rotate=270] (supsetB0C1) {$\supset$};
      \node at (4.2,-1.05) (B0C1) {$\Blue{B_0}\Red{C_1}$};
      \node at (7.95,-0.55) [rotate=270] (supsetC0C1) {$\supset$};
      \node at (8.37,-1.05) (C0C1) {$\Red{C_0 C_1 = C}$};
    }
    \node[above right=of H1,Purple] (T2) {$T^{(2)}$};
    \node[below right=of H1] (E2) {$E^{(2)}$};
    \node[below right=of T2,Red] (H2) {$H^{(2)} = S$};

    \path (H0.south east) edge[-] (E1)
    (H0.north east) edge[->] node[above,sloped] {\citationsize embed} (T1)
    (E1) edge[-] (H1)
    (T1) edge[->] (H1)
    (H1) edge[-] (E2)
    (H1) edge[->] node[above,sloped] {\citationsize embed} (T2)
    (E2) edge[-] (H2.south west)
    (T2) edge[->] (H2.north west)
    ;

    \path (H0) edge[->,dashed] node[above]
    {\citationsize $E^{(1)}$-linear}
    % node[below] {\citationsize(induced)}
    (H1)
    % 
    (H1) edge[->,dashed] node[above] {\citationsize
      $E^{(2)}$-linear}
    % node[below] {\citationsize (induced)}
    (H2)
    ;
  \end{tikzpicture}
\end{center}
\end{frame}

\begin{frame}[label=conclusions]{Concluding Thoughts}
  \begin{itemize}
  \item<+-> The ``linear'' steps (isolating coeffs \& ring-switching)
    of our procedure are \alert<.>{fast} and have \alert<.>{small
      noise expansion}.

    \smallskip The bottleneck is the \alert<.>{high-degree rounding}
    $\Zq \to \Z_{2}$.

    \medskip
  \item<+-> Gradually changing $\Blue{B}$ to $\Red{C}$ via hybrid
    rings \& bases is analogous to the FFT's conversion from ``time
    basis'' to ``frequency basis.''

    \medskip \onslide<+-> New ring-switching technique should also be
    useful for other signal-processing transforms having ``sparse
    decompositions.''

    \medskip 
  \item<+-> Practical implementation and evaluation are underway.
  \end{itemize}

  \bigskip
  \onslide<+->
  \begin{center}
    {\Large Thanks!}
  \end{center}
  
\end{frame}

\end{document}

%%% Local Variables: 
%%% mode: latex
%%% TeX-master: t
%%% End: 
